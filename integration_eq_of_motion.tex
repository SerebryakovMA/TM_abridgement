\subsection{Интегрирование уравнения движения систем с одной степенью свободы}
Мы будем интересоваться решением ДУ вида $\ddot{x} = F(t, x, \dot{x}).$\footnote{$x= 1;\; \dim x =1, s =1$}
\subsubsection{Классификация состояний равновесия автономной системы на плоскости}
\begin{equation}
\ddot{x} = F(x, \dot{x}) \Rightarrow \begin{cases}
\dot{x} = f(x, y)\\
\dot{y} = g(x, y)
\end{cases}
\end{equation}
Состояние равновесия
\begin{equation}
\begin{cases}
\dot{x} = 0\\
\dot{y} = 0
\end{cases}
\Leftrightarrow
\begin{cases}
f(x_0, y_0) =0\\
g(x_0, y_0) = 0.
\end{cases}
\end{equation}
Рассматриваемое малое возмущение
\begin{equation}
\begin{cases}
x = x_ 0 + \xi\\
y = y_0 + \eta 
\end{cases}
\Rightarrow
\begin{cases}
\dot{xi} = \alpha \xi + \beta \eta\\
\dot{\eta} = \gamma \xi + \delta \eta,
\end{cases}
\alpha = \left. \pdv{f}{x} \right|_{x_0, y_0}, \beta = \left. \pdv{f}{y} \right|_{x_0, y_0}, \ldots, \delta = \left. \pdv{g}{y} \right|_{x_0, y_0}.\footnote{Рассматриваем линейное состояние равновесия.}
\end{equation}
Будем искать решение в виде 
\begin{gather}
\begin{pmatrix}
\xi\\
\eta
\end{pmatrix} = 
\begin{pmatrix}
a\\
b
\end{pmatrix}
e^{\lambda t} \Rightarrow \begin{cases}
\lambda a = \alpha a + \beta\\
\lambda b = \gamma a + \delta b
\end{cases} \Rightarrow 
\det \begin{pmatrix}
\alpha - \gamma & \beta\\
\gamma & \delta - \lambda
\end{pmatrix}
= 0 \Leftrightarrow\\
\Leftrightarrow (\lambda - \alpha)(\lambda - \delta) = \gamma \beta,\\
\lambda_{1,2} = \frac{\alpha + \beta}{2} \pm \sqrt{\left(\frac{\alpha - \delta}{2}\right)^2 - \gamma \beta}.
\end{gather}
\begin{enumerate}
\item $\lambda = \pm i \omega$ --- состояние равновесия типа центр, устойчивое, не асимптотически устойчивое. Фазовые кривые --- эллипсы.
\item $\lambda p \pm i \omega$ --- состояние равновесия типа фокус, может быть в зависимости от знака $p$ как устойчивым, так и неустойчивым\footnote{Устойчиво при $p < 0$.}, устойчивое равновесие асимптотически устойчиво. Фазовые кривые <<сжимаются>> или <<раскручиваются>>.
\item $\lambda \in \mathbb{R}, \lambda_1 \cdot \lambda_2 > 0$ ---состояние равновесия типа узел --- равновесие типа фокус при очень сильном трении. Асимптотически устойчиво либо неустойчиво.
\item $\lambda \in \mathbb{R}, \lambda_1 \cdot \lambda_2 < 0$ --- состояние равновесия типа седло, неустойчиво.
\end{enumerate}
$H/w$ Как найти направления асимптот?

\begin{ex}
\begin{gather}
\ddot{x} = f(x, v), v =0, m = 1 \\
\begin{cases}
\dot{x} = v\\
\dot{v} = f_0(x) - \mu \cdot v = - \pdv{U(x)}{x} - \mu \cdot x
\end{cases}\\
\pdv{U(x_0)}{x} = 0; \pdv[2]{U(x_0)}{x} = U'' \neq 0; \begin{cases}
x = x_0 + \xi\\
v = \eta
\end{cases}
\Rightarrow\\
\begin{cases}
\dot{\xi} = v\\
\dot{v} = -U'' \cdot \xi - \mu \cdot v
\end{cases} \Rightarrow 
\det \begin{pmatrix}
-\lambda & 1\\
-U'' & -\lambda - \mu
\end{pmatrix} = 0 \Rightarrow \boxed{\lambda^2 + \mu \lambda + U'' = 0} \Rightarrow \\
\lambda_{1, 2} = - \frac{\mu}{2} \pm \sqrt{\frac{\mu^2}{4} - U''}
\end{gather}
Чередуются состояние равновесия типа седло, узел и фокус, центры превращаются в диссипативные состояния равновесия, фокусы отвечают случаям малого трения, при её увеличении превращаются в узлы.
\end{ex}

\subsubsection{Интегрирование уравнения движения консервативных одномерных систем}
Будем рассматривать лагранжеву задачу, в которой существует интеграл энергии.
\[L(x, \dot{x})\; \text{--- уравнения Лагранжа решаются в квадратурах.}\]
\begin{equation}
\pdv{L}{t} = 0 \Rightarrow H(x, \dot{x}) = const\; \text{на решениях}.
\end{equation}
Фазовый портрет --- линии уровня $H(x, \dot{x})$.\index{Фазовый портрет}
\begin{gather}
H(x, \dot{x}) = E = const \Rightarrow \dot{x} = V(x, E)\\
\dv{x}{t} = V(x, E) \Rightarrow \dd{t} = \frac{\dd{x}}{V(x, E)} \Rightarrow \boxed{t= \int^x \frac{\dd{x}}{V(x, E)}},
\end{gather}
аддитивная константа в нижнем пределе интегрирования отвечает заданному $E$, его линии уровня, и такие точки надо различать, поэтому на самом деле везде в решении вместо $V(x, E)$ должно быть $V_i (x, E),$ где $i$ определяет \textit{ветку однозначности}.\index{Ветка однозначности}

\begin{dfn}
$x$ --- ограниченная область --- движение финитное; $x$ --- неограниченная область --- движение инфинитное.
\end{dfn}
\begin{pst}
Всякое финитное движение в одномерной консервативной системе обязательно является периодическим. 
\end{pst}

Частный случай --- натуральная система.
\begin{gather}
L(x, \dot{x}) = \frac{1}{2} m(x) \dot{x}^2 - U(x)\\
H(x, \dot{x}) = \frac{1}{2} m \dot{x}^2 + U(x) = E \Rightarrow \dot{x} = \pm \sqrt{\frac{2}{m}\left(E- U(x)\right)}\\
\boxed{t = \pm \int^x \sqrt{\frac{m(x)}{2} \frac{\dd{x}}{\sqrt{E - U(x)}}}},
\end{gather}
знак в начале определяется начальными условиями, меняется в точке разворота, где знаменатель обращается в нуль, то есть $U(x) = E$. 
\begin{gather}
p(x) = \pdv{L}{\dot{x}} = m(x) \dot{x} = \pm \sqrt{2m (E - U)}\\
\pdv{p}{E} = \sqrt{\frac{m}{2}}\frac{1}{\sqrt{E- U}} \Rightarrow t = \pm \pdv{E} \int^x p (x, E) \dd{x},
\end{gather}
а интеграл --- площадь под графиком $p(x)$.
\begin{dfn}
$U(x) < E $ --- область возможного движения, иначе мы переходим в комплексное время, что запрещено в классической механике.
\end{dfn}
\begin{dfn}
$U(x) = E$ --- точки разворота.
\end{dfn}

Фазовый портрет обладает зеркальной симметрией относительно оси $x$.
$[x_3, \infty]$ --- инфинитное движение.

\[\dot{x} \sim \sqrt{E - U(x)} \sim \pm \sqrt{x - x_1}\]
$\dot{x} \sim \pm \sqrt{(x - x_0)^2} = \pm \abs{x-x_0}$ --- в состоянии равновесия с асимптотами $\left(\pdv{U(x_0)}{x} =0\right)$.

\subsubsection{Диссипативные одномерные системы}
В общем случае способа решения в квадратурах для диссипативных систем нет. И оказывается, что все известные интегрируемые диссипативные системы на самом деле не по-настоящему диссипативные. Поясним, о чём идёт речь. Вспомним консервативную задачу $L = \frac{1}{2} m(x) \dot{x}^2 - U(x)$ и посмотрим, какому уравнению движения такой лагранжиан соответствует (а интересует нас уравнение вида $\ddot{x} = F(t, x, \dot{x})$).
\begin{equation}
L = \frac{1}{2} m(x) \dot{x}^2 - U(x) \Rightarrow m\ddot{x} + \frac{1}{2} \pdv{m}{x} \dot{x}^2 = -  \pdv{U}{x} \Rightarrow \begin{cases}
\ddot{x} = f(x) - \mu (x) \dot{x}^2,\; \text{где}\\
f(x) = - \frac{1}{m(x)} \pdv{U}{x}, \; \text{а} \\
\mu({x}) = \frac{1}{2m} \pdv{m}{x}, 
\end{cases}
\end{equation}
и мы можем получить решение в квадратурах для любых $f(x)$ и $\mu(x)$:
\begin{align} 
\mu({x}) &= \frac{1}{2m} \pdv{m}{x} \Rightarrow \mu = \pdv{x}(\ln \sqrt{m}) \Rightarrow m = \left( e^{\int \mu \dd{x}}\right)^2,\\
f(x) &= - \frac{1}{m(x)} \pdv{U}{x} \Rightarrow U = - \int m f \dd{x}.
\end{align}
Задачи вида $\ddot{x} + \omega_0^2 x + \mu \dot{x}^n =0,$ где $n = 2k$, допускают решение в квадратурах, в системе <<псевдотрение>>. При $n = 2k +1$ трение уже истинное, поскольку <<штука>> $\dot{x} \cdot \mu \dot{x}^n$ знакоопределена, а она управляет скоростью вытекания энергии:
\begin{equation}
 H_0 = \frac{1}{2} \dot{x}^2 + \frac{\omega^2}{2}x^2 \Rightarrow \dv{H_0}{t} = - \mu \dot{x}^{n+1}.
 \end{equation} 
Но можно и при $n = 2k$ сделать трение: достаточно добавить сигнум $\ddot{x} + \omega_0^2 x + \mu \sign \dot{x} \cdot \dot{x}^n = 0,$ при $n = 2k$ получаем трение, при $n = 2k +1$ --- псевдотрение.
\begin{ex}[n= 0]
Как выглядит фазовый потрет системы $\ddot{x} + \omega_0^2 x + \mu = 0$? А если добавить сигнум, как увидеть, что задача стала диссипативной?
Как выглядит настоящее диссипативное решение? Сшиваем по непрерывности два решения. Зона застоя.

Когда рисуем сигнум скорости, говорим, что фазовой фазовый портрет системы получается из фазового портрета системы без сигнума путём склейки: разрезаем по горизонтальной прямой, с помощью сигнума разворачиваем и склеиваем по непрерывности.

\emph{H/w} $\sign \dot{x} \cdot \dot{x}$ --- задача с трением --- нарисовать фазовый портрет, зная ФП консервативной системы.
\end{ex}

\paragraph{Тривиальные случаи}
$\dot{x} = F(v) \Rightarrow \dot{v} = F(v) \Rightarrow v(t) \Rightarrow x(t) \int v \dd{t}$

\subsubsection{Качественный анализ на фазовой плоскости}
...
\begin{ex}[Линейное трение в системе можно учитывать с помощью чисто лагранжевой задачи с нестационарным лагранжианом]
\begin{align}
L &= \frac{1}{2} \dot{x}^2 - U(x) &  &\longrightarrow &  \ddot{x} &= - \pdv{U}{x} = f(x),\\
\intertext{и в эту задачу мы можем добавить линейное трение:}
L &= \left(\frac{1}{2} \dot{x}^2 - U\right) e^{2\gamma t} &  &\longrightarrow & p &= \pdv{L}{\dot{x}} = \dot{x} e^{2 \gamma t}\\
& & & & \dot{p} &=(\ddot{x} + 2\gamma \dot{x})e^{2\gamma t} = \pdv{L}{x} = -\pdv{U}{x} e^{2\gamma t} 
\end{align}
И последнее равенство, которое можно сократить на $e^{2\gamma t}$ является следствием одной теоремы\dots
\end{ex}
\begin{pst}
\begin{equation}
\begin{rcases}
\ddot{x} = F(t, x, \dot{x})\\
s =1
\end{rcases}
\Rightarrow
 \exists \left.
\begin{matrix} 
L(t, x, \dot{x})\\
\mu (t, x, \dot{x})\neq 0
\end{matrix}
\right|\; (\ddot{x} - F) \cdot \mu = \dv{t} \pdv{L}{\dot{x}} - \pdv{L}{x}
\end{equation}
\end{pst}

\begin{pst}
\begin{equation}
\ddot{x} = F(x, \dot{x}) \Rightarrow
\exists \;
\begin{matrix} 
L(x, \dot{x})\\
\mu (x, \dot{x})
\end{matrix}
\quad \ldots
\end{equation}
\end{pst}
\begin{ex}
\begin{equation}
\ddot{x} + \omega^2 x + 2\gamma \dot{x} = 0 \Leftarrow L = \frac{\dot{x} + \gamma x}{\gamma x} \arctg \frac{\dot{x} + \gamma x}{\omega x} -\frac{\omega}{2\gamma} \ln \left( \omega^2 x^2 + (\dot{x} + \gamma x)^2 \right)\footnote{\emph{H/w} проверить. См. статью \cite{Shalashov_2017}.}
\end{equation}
\begin{equation}
\begin{cases}
x(t) = A e^{-\gamma t} \sin (\omega t + \gamma)\\
p(t)  = \pdv{L}{\dot{x}}\; \text{уходит в бесконечность за конечное время,}
\end{cases}
\end{equation}
движение инфинитно.
\end{ex}