\subsubsection{Периодическое движение в одномерных консервативных системах}
Консервативная система --- система, которая задаётся лагранжианом $L(x, \dot{x}),$ но мы ограничим круг задач натуральными системами, то есть будем считать, что $L(x, \dot{x}) = \frac{m(x)}{2} \dot{x}^2 - U(x),$ и часть времени будет полагать, что $m$ --- константа. Ранее мы договорились, что в таких системах всякое финитное движение периодическое: это видно по графику потенциальной энергии: движение сосредоточено между точками разворота, где энергия $U(x) < E,$ $E$ --- константа, определяемая начальными условиями задачи . Выпишем формулы, которые у нас получились,
\begin{gather}
U(x_{1,2}) = E,
\end{gather}
\begin{equation}
T(E) = 2 \int_{t_1}^{t_2} \dd{t} = \left| \dd{t} = \frac{\dd{x}}{v(x)} = \sqrt{\frac{m}{2}} \frac{\dd{x}}{\sqrt{E - U(x)}} \right| = \int\limits_{x_1 (E)}^{x_2 (E)} \frac{\sqrt{2m}\dd{x}}{\sqrt{E - U(x)}}\footnote{Не забываем, что $\frac{m \dot{x}^2}{2} + U = E$.}, \label{period_formula}
\end{equation}
и дальше этот раздел будет в основном посвящён анализу формулы \eqref{period_formula}.

Вспомним, как дифференцировать неопределённые интегралы. Сначала надо прокоммутировать интеграл и производную, а потом вспомнить, что пределы зависят от параметра:
\begin{equation}
\pdv{E} \int\limits_{x_1 (E)}^{x_2 (E)} f(x, E) \dd{x} = \int\limits_{x_1}^{x_2} \pdv{E} f(x, E) \dd{x} - \pdv{x_2}{E} f(x_2, E) - \pdv{x_1}{E} f(x_1, E).
\end{equation}

Понять, что несобственный интеграл в формуле \eqref{period_formula} сходится, можно, разложив подынтегральную функцию в ряд, который сходится равномерно. Если же первых член разложения становится равным нулю, то интеграл перестаёт сходиться.

Отметим ещё один интересный факт: $\frac{\sqrt{m}}{\sqrt{E - U}} = 2 \pdv{E} \sqrt{m} \sqrt{E - U},$ и тогда несложно заметить, что
\begin{equation}
 \boxed{T(E) = 2 \pdv{E}  \int\limits_{x_1}^{x_2} \underbrace{\sqrt{2m (E- U)} }_{p(x) \equiv \pdv{L}{\dot{x}} = m\dot{x}}\dd{x}}, \label{T_from_E}
 \end{equation} 
 построив график, получим ещё одну замечательную формулу:
 \begin{equation}
 T(E) = \pdv{E} \oint p(x, E) \dd{x},
 \end{equation}
 то есть выбранный уровень энергии определяет на фазовой плоскости замкнутую кривую, и скорость изменения площади изменения площади внутри этой кривой, в зависимости от энергии, это и есть период.\footnote{$\oint p \dd{x}$ --- адиабатический инвариант. \index{Адиабатический инвариант}}
 
\subsubsection{Теория возмущений для периодического движения}
Пусть есть некоторая задача, про которую мы всё знаем...
Рассмотрим новую задачу в потенциале $U(x)  + \delta U(x),$ где $\delta U(x)$ --- малая (в смысле неисчезновения в ряде Тейлора) поправка. Можем ли мы найти новый период, не решая задачу заново? То есть, как найти часть $\delta T(E)$ в первом неисчезающем приближении по $\delta U(x)$. Прежде, чем ответить на этот вопрос, рассмотрим более простую задачу --- попробуем найти, как изменятся точки разворота. Вспомним, что в исходной задаче $U(x_1) = E,$ а $x(t)$ и $T(E)$ мы знаем. Тогда мы можем сказать, что 
\begin{gather}
U(x_1 + \delta x_1) + \delta U(x_1 + \delta x_1) = E,
\end{gather}
то есть в потенциале с возмущением новая точка разворота, и будем пользоваться малостью, раскладывая слагаемые в левой части:
\begin{gather}
U(x_1 + \delta x_1) = U(x_1) + U'(x_1) \delta x_1 + \ldots,\\
\delta U(x_1 + \delta x_1) = \delta U(x_1) + \ldots,
\end{gather}
но $U(x_1)$ это в точности $E,$ как было упомянуто ранее, поэтому 
\begin{equation}
\var{x_1} \approx - \frac{\var{U(x_1)}}{U'(x_1)},
\end{equation}
но это, очевидно, не работает при $U'(x_1) = 0,$ и нам нужен следующий член в разложении\dots

... Стартуем с выражения \eqref{T_from_E}, которое будем раскладывать, подставив $U(x) + \delta U(x)$ вместо $U$:
\begin{gather}
 T(E) + \delta T(E) = 2 \pdv{E} \int_{x_1 + \delta x_1}^{x_2 + \delta x_2} \sqrt{2m (E- U - \delta U)} \dd{x},\\
 \intertext{и начинаем раскладывать:}
\delta T(E) = 2 \pdv{E} \left\{ \int\limits_{x_1}^{x_2} \sqrt{2m} \left(- \frac{\delta U}{2} \right) \frac{\dd{x}}{\sqrt{E- U}} + \delta x_2 \left. \sqrt{2m (E - U - \delta U)}\right|_{x_2} - \delta x_1 \left. \sqrt{2m (E - U - \delta U)}\right|_{x_1} \right\},
 \end{gather}
 интеграл, как нам бы и хотелось, оказывается пропорционален $\delta U$\footnote{$ \int\limits_{x_1}^{x_2} \sqrt{2m} \left(- \frac{\delta U}{2} \right) \frac{\dd{x}}{\sqrt{E- U}} \sim \var U$, $\delta x_2 \left. \sqrt{2m (E - U - \var U)}\right|_{x_2} = \frac{1}{2} \var x_2 \sqrt{2m \var U|_{x_2}} \sim \var U^{3/2}$}, интегральные подстановки оказываются более высокого порядка малости ... поэтому, если получаем ненулевое первое слагаемое, то подстановки можем выкинуть, тогда
 \begin{equation}
\boxed{ \delta T(E) = \pdv{E} \int\limits_{x_1}^{x_2} \frac{\sqrt{2m}\, \delta U}{\sqrt{E-U}} \dd{x} + o(\var U)}.
 \end{equation}
 При этом $\dfrac{\sqrt{2m}}{\sqrt{E-U}} \dd{x} = 2 \dd{t}$, и 
 \begin{equation}
\Rightarrow \var T(E) = - \pdv{E} \int_{t_1}^{t_2} \var U 2 \dd{t} = - \pdv{E} \int_0^{T(E)} \var U\left(x(t)\right) \dd{t}. 
 \end{equation}
Раскладывая до первого неисчезающего члена, получим
\begin{equation}
\boxed{\var T = \frac{(-1)^n}{n!} \pdv[n]{E} \left(T \expval{\var U^n} \right)}.
\end{equation}

\boxed{Дописать!}
\subsubsection{Колебания математического маятника}
