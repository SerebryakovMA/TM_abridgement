Стартуем с механики Ньютона:
\[m_i \Ddot{\vec{r}}_i = \Vec{F}_i,\; i=\overline{1,N}, \] но беда в том, что для ряда сил мы знаем результат их действия, а не сами силы.
\begin{dfn}
Связи \index{Связи!} --- не вытекающие из уравнения движения ограничения на $\lbrace\vec{r}_i, \vec{v}_i\rbrace$.
\[m_i \Ddot{\vec{r}}_i = \vec{F}_i^{(a)} + \vec{R}_i \] Так, выше указана несвободная система --- на неё наложены связи. $\vec{F}_i^{(a)}$ --- активные силы (их знаем), $\vec{R}_i$ --- силы реакции (знаем связи).
\end{dfn}
\subsection{Связи и их классификация.}
\[f(t, \lbrace\vec{r}_i\rbrace, \lbrace\vec{v}_i \rbrace) = 0,\] связи в виде равенств --- удерживающие связи.
\[f(\dots) \geqslant 0 \text{--- неудерживающиие, ненапряжённые связи.}\] Неудерживающие связи впервые появились только в теории атомного ядра; математически их можно представить в виде удерживающей связи, подставив в степ-функцию.
...\footnote{см. \cite{OlhTM}, c. 204, \cite{GantnakherTM}, с. 201.}
\begin{gather}
f(t, \lbrace\vec{r}_i\rbrace) = 0 \Rightarrow \notag\\
\pdv{f}{t} + \sum_{i=1}^N \pdv{f}{\vec{r}_i} \vec{v}_i = 0. \label{golonom}
\end{gather}
Конечные, дифференцируемы, недифференцируемые, интегрируемые, голономные \eqref{golonom}.
Стационарные (склерономные)($\not t$), нестационарные (реаномные)($t$) связи.
\begin{ex}

\end{ex}


$\boxed{\text{\huge{WILL BE COMPLEMENTED}}}$