Стартуем с механики Ньютона:
\[m_i \Ddot{\vec{r}}_i = \Vec{F}_i (\vec{r}_1, \ldots, \vec{r}_N, \vec{v}_1, \ldots, \vec{r}_N, t),\; i=\overline{1,N}, \] но беда в том, что для ряда сил мы знаем результат их действия, а не сами силы.
\begin{dfn}
Связи \index{Связи!} --- не вытекающие из уравнения движения ограничения на положения точек $\lbrace\vec{r}_i, \vec{v}_i\rbrace$.
\[m_i \Ddot{\vec{r}}_i = \vec{F}_i^{(a)} + \vec{R}_i \] Так, выше указана несвободная система \index{Несвободная система} --- на неё наложены связи. $\vec{F}_i^{(a)}$ --- активные силы (их знаем), $\vec{R}_i$ --- силы реакции (знаем связи)\footnote{Силы, с которыми  тела, осуществляющие связи, действуют на точки системы называются реакциями связей.}.
\end{dfn}
\subsection{Связи и их классификация.}
Различают голономные и неголономные, удерживающие и неудерживающие, стационарные и нестационарные связи.
\begin{dfn}
Голономными (или интегрируемыми) связями называют связи, уравнения которых всегда можно свести к уравнениям вида
\begin{equation}
f(\vb{r}_1, \ldots, \vb{r}_N, t) = 0,
\end{equation}
где $f$ является функцией только координат точек и времени. Эти связи накладывают ограничения не только на положение, но и на скорости и ускорения точек системы.
\end{dfn}
\begin{dfn}
Неголономными  (неинтегрируемыми) связями называют связи, уравнения которых нельзя свести к уравнениям, содержащим только координаты точек и время. Неголономной, например, является связь, налагаемая на шар, катящийся по шероховатой поверхности.
\end{dfn}

\[f(t, \lbrace\vec{r}_i\rbrace, \lbrace\vec{v}_i \rbrace) = 0,\] связи в виде равенств --- удерживающие связи. \index{Связи! удерживающие}
\[f(\dots) \geqslant 0 \text{--- неудерживающиие, ненапряжённые связи.}\]  Неудерживающие связи впервые появились только в теории атомного ядра; математически их можно представить в виде удерживающей связи, подставив в степ-функцию.
...\footnote{см. \cite{OlhTM}, c. 204, \cite{GantnakherTM}, с. 201.}
\begin{gather}
f(t, \lbrace\vec{r}_i\rbrace) = 0 \Rightarrow \notag\\
\pdv{f}{t} + \sum_{i=1}^N \pdv{f}{\vec{r}_i} \vec{v}_i = 0. \label{golonom}
\end{gather}
Конечные, дифференцируемые, недифференцируемые, интегрируемые, голономные \eqref{golonom}.
Стационарные (склерономные)($\not t$), нестационарные (реаномные)($t$) связи.
\begin{ex}

\end{ex}

\subsection{Основная задача механики. Идеальные связи}
\subsection{Идеальные связи и уравнения Лагранжа первого рода}
Есть $\{\vb{r}_i\}$. Договорились, что 
\begin{equation}
m_i \ddot{\vb{r}}_i = \vb{F}_i + \vb{R}, \quad i = \overline{1, N},
\end{equation}
то есть поделили на активные силы и силы реакции связей, но плохо то, что знаем не все силы правой части. Из-за $R_i$-ых возникают $3N$ новых величин, связи дают лишь $K$ величин, и мы хотим выяснить, когда у нас задача согласована.
Вспомним про голономные связи, которые могут быть представлены в виде \begin{equation}
f_j (\{\vb{r}_i\}, t) = 0, \quad j = \overline{1, K}.
\end{equation} 
\begin{dfn}
Возможное перемещение --- произвольное бесконечно малое перемещение точек системы, которое согласовано со связями. Формально
\begin{gather}
\left. \sum_i \pdv{f_j}{\vb{r}_i}\vb{v}_i + \pdv{f_j}{t} \equiv 0 \right| \cdot \dd{t},\\
\intertext{то есть умножаем на $\dd{t}$ интегрируемую связь, тогда}
\sum_i \pdv{f_j}{\vb{r}_i} \dd{\vb{r}_i} + \pdv{f_j}{t} \equiv 0,
\end{gather}
и решение этой системы $K$ уравнений является совокупностью всех возможных перемещений.
\end{dfn}

\begin{dfn}
Действительное перемещение --- бесконечно малое перемещение, совместимое со связями и уравнениями движения (и оно единственно как решение задачи Коши).
\end{dfn}

Если <<заморозить>> время, то есть <<забыть>> про частную производную $\pdv{f}{t}$,  то 
\[\left. \sum_i \pdv{f_j}{\vb{r}_i} \vb{v}_i = 0 \right| \cdot \dd{t} \quad \Rightarrow \sum_i \pdv{f_j}{\vb{r}_i} \var \vb{r}_i = 0 \footnote{$\var \vb{r}_i = \vb{v}_i \dd{t}$ --- дифференциал при замороженном времени.}\] 

\begin{dfn}
Виртуальное перемещение --- бесконечно малое перемещение, совместимое со связями при замороженном времени.
\end{dfn}

Виртуальному перемещению можем сопоставить виртуальную работу:
\[\{\var \vb{r}_i\} \longrightarrow \var A = \sum_i \vb{F}_i \var \vb{r}_i = \sum_i \vb{F}_i^{(a)} \var \vb{r}_i + \sum_i \vb{R}_i \var \vb{r}_i.\]
И оказывается, что почти в любом идеализированном механизме без трения $\var A_R = 0$.

\begin{dfn}
Связь называется идеальной, если $\var A_R = 0 \quad \forall \{\var \vb{r}_i\},$ то есть если виртуальная работа сил реакции связей равна нулю при любом виртуальном перемещении.
\end{dfn}
\noindent Примером идеальной связи может служить движение по гладкой неподвижной поверхности.


\textit{Эмпирическое утверждение.} Почти все связи в механике являются идеальными. Но трение  (попытка учесть немеханическое явление в механике) разрушает идеальность, и мы не знаем, как устроены связи, то есть мы обычно идеализируем задачи и почти всегда угадываем, но при этом сядем в лужу, если уйдём в очень большие масштабы --- в космологию, или в очень малые...

\begin{thm}
Пусть дана идеальная связь:
\begin{equation}
\begin{cases}
\sum_i \vb{R}_i \var \vb{r}_i = 0,\\
\sum_i \pdv{f_j}{\vb{r}_i} \var \vb{r}_i = 0 
\end{cases} \Longleftrightarrow
\exists\; \lambda_j (t): \; \vb{R}_i = \sum_{j=1}^K \lambda_j \pdv{f_j}{\vb{r}_i},
\end{equation}
то есть решили основную задачу механики.
\end{thm}
Перед доказательством применим сформулированную теорему, рассмотрим следствие из неё. 
\begin{cns} Посмотрим, как будут записываться уравнения движения.
\begin{equation}
\begin{cases}
m_i \ddot{r}_i = \vb{F}_i^{(a)} + \sum_{j=1}^K \lambda_j \pdv{f_j}{\vb{r}_i}\\
f_j (t, \{\vb{r}_i\}) = 0,
\end{cases}
\end{equation}
и эта задача уже математически корректна: в ней число переменных соответствует числу уравнений: $3N+K$ неизвестных, $3N+K$ соотношений. Уравнения движения в такой форме для несвободной системы называют \textit{уравнениями Лагранжа первого рода}. \index{Уравнения Лагранжа! первого рода}

И как эти уравнения можно решать?
Метод Лагранжа.
\begin{enumerate}
\item $\forall \lambda(t) \Rightarrow \vb{r}(t, \lambda)$ из уравнений движения.
\item $f(\{\vb{r}(t, \lambda\}, t) \equiv 0 \Rightarrow \lambda.$
\end{enumerate}
\end{cns}
\begin{rmk}
$\pdv{f}{t} = 0 \Rightarrow \pdv{\lambda}{f} = 0$
\end{rmk}
