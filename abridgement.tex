\documentclass[12pt]{article}
\usepackage[utf8]{inputenc}
 \usepackage[T1, T2A]{fontenc}
\usepackage[english, russian]{babel}
\usepackage{caption}
\usepackage{indentfirst}
\usepackage{graphicx, xcolor}
\usepackage{cmap}

\usepackage[unicode, pdftex]{hyperref}
\hypersetup{linkcolor=blue, urlcolor=blue, colorlinks=true}
\usepackage{hyphenat}
\hyphenation{объект}
\usepackage{wrapfig}
\usepackage[left=1.4cm,right=1.4cm,top=1.5cm,bottom=1.5cm,bindingoffset=0cm]{geometry}
\usepackage{tocloft}    
\usepackage{titlesec} \titlelabel{\thetitle.\quad} 
\frenchspacing
\makeatletter
\renewcommand{\@biblabel}[1]{#1.} % Заменяем библиографию с квадратных скобок на точку:
\makeatother
\makeindex

\usepackage[]{mathtools}
\renewcommand{\theequation}{\thesection.\arabic{equation}}
    
\parindent=1.25cm
%\parskip=0.1cm

\usepackage{physics} 
\usepackage{siunitx} % typesets numbers with units very nicely
\usepackage{amssymb,amsfonts,amsmath,mathtext,cite,enumerate,float}
\usepackage{amsthm}

%\usepackage[dvips]{graphicx}

\usepackage{multicol}

\bibliographystyle{unsrt}


\begin{document}
\renewcommand{\cftsecaftersnum}{.}
\renewcommand{\cftsubsecaftersnum}{.}

\renewcommand\refname{Список литературы}

\theoremstyle{plain}
\newtheorem{thm}{Теорема}[section]
\newtheorem{lem}[thm]{Лемма}
\newtheorem{pst}{Постулат}[section]

\theoremstyle{definition}
\newtheorem{dfn}{Определение}[section]
\newtheorem{cns}[thm]{Следствие}

\theoremstyle{remark}
\newtheorem{task}{Задача}[section]
\newtheorem{ex}{Пример}[subsection]
\newtheorem{cex}[ex]{Контрпример}
\newtheorem{rmk}{Замечание}[subsection]

\newcommand*{\eqdef}{\stackrel{\mathrm{def}}{=}}
\newcommand*{\is}[1]{\stackrel{\mathrm{\eqref{#1}}}{=}}
\newcommand*{\eqq}[1]{\stackrel{\mathrm{#1}}{=}}
\newcommand*{\hlf}{\frac{1}{2}}

\columnseprule = 0.4pt

\mathtoolsset{showonlyrefs=true}
\mathtoolsset{showmanualtags=true}

\begin{center}
\Huge{\textbf{Теоретическая механика}}
\end{center}
\tableofcontents
\newpage
\input{section1.tex}
\section{Механика Лагранжа}
Стартуем с механики Ньютона:
\[m_i \Ddot{\vec{r}}_i = \Vec{F}_i (\vec{r}_1, \ldots, \vec{r}_N, \vec{v}_1, \ldots, \vec{r}_N, t),\; i=\overline{1,N}, \] но беда в том, что для ряда сил мы знаем результат их действия, а не сами силы.
\begin{dfn}
Связи \index{Связи!} --- не вытекающие из уравнения движения ограничения на положения точек $\lbrace\vec{r}_i, \vec{v}_i\rbrace$.
\[m_i \Ddot{\vec{r}}_i = \vec{F}_i^{(a)} + \vec{R}_i \] Так, выше указана несвободная система \index{Несвободная система} --- на неё наложены связи. $\vec{F}_i^{(a)}$ --- активные силы (их знаем), $\vec{R}_i$ --- силы реакции (знаем связи)\footnote{Силы, с которыми  тела, осуществляющие связи, действуют на точки системы называются реакциями связей.}.
\end{dfn}
\subsection{Связи и их классификация.}
Различают голономные и неголономные, удерживающие и неудерживающие, стационарные и нестационарные связи.
\begin{dfn}
Голономными (или интегрируемыми) связями называют связи, уравнения которых всегда можно свести к уравнениям вида
\begin{equation}
f(\vb{r}_1, \ldots, \vb{r}_N, t) = 0,
\end{equation}
где $f$ является функцией только координат точек и времени. Эти связи накладывают ограничения не только на положение, но и на скорости и ускорения точек системы.
\end{dfn}
\begin{dfn}
Неголономными  (неинтегрируемыми) связями называют связи, уравнения которых нельзя свести к уравнениям, содержащим только координаты точек и время. Неголономной, например, является связь, налагаемая на шар, катящийся по шероховатой поверхности.
\end{dfn}

\[f(t, \lbrace\vec{r}_i\rbrace, \lbrace\vec{v}_i \rbrace) = 0,\] связи в виде равенств --- удерживающие связи. \index{Связи! удерживающие}
\[f(\dots) \geqslant 0 \text{--- неудерживающиие, ненапряжённые связи.}\]  Неудерживающие связи впервые появились только в теории атомного ядра; математически их можно представить в виде удерживающей связи, подставив в степ-функцию.
...\footnote{см. \cite{OlhTM}, c. 204, \cite{GantnakherTM}, с. 201.}
\begin{gather}
f(t, \lbrace\vec{r}_i\rbrace) = 0 \Rightarrow \notag\\
\pdv{f}{t} + \sum_{i=1}^N \pdv{f}{\vec{r}_i} \vec{v}_i = 0. \label{golonom}
\end{gather}
Конечные, дифференцируемые, недифференцируемые, интегрируемые, голономные \eqref{golonom}.
Стационарные (склерономные)($\not t$), нестационарные (реаномные)($t$) связи.
\begin{ex}

\end{ex}

\subsection{Основная задача механики. Идеальные связи}
\subsection{Идеальные связи и уравнения Лагранжа первого рода}
Есть $\{\vb{r}_i\}$. Договорились, что 
\begin{equation}
m_i \ddot{\vb{r}}_i = \vb{F}_i + \vb{R}, \quad i = \overline{1, N},
\end{equation}
то есть поделили на активные силы и силы реакции связей, но плохо то, что знаем не все силы правой части. Из-за $R_i$-ых возникают $3N$ новых величин, связи дают лишь $K$ величин, и мы хотим выяснить, когда у нас задача согласована.
Вспомним про голономные связи, которые могут быть представлены в виде \begin{equation}
f_j (\{\vb{r}_i\}, t) = 0, \quad j = \overline{1, K}.
\end{equation} 
\begin{dfn}
Возможное перемещение --- произвольное бесконечно малое перемещение точек системы, которое согласовано со связями. Формально
\begin{gather}
\left. \sum_i \pdv{f_j}{\vb{r}_i}\vb{v}_i + \pdv{f_j}{t} \equiv 0 \right| \cdot \dd{t},\\
\intertext{то есть умножаем на $\dd{t}$ интегрируемую связь, тогда}
\sum_i \pdv{f_j}{\vb{r}_i} \dd{\vb{r}_i} + \pdv{f_j}{t} \equiv 0,
\end{gather}
и решение этой системы $K$ уравнений является совокупностью всех возможных перемещений.
\end{dfn}

\begin{dfn}
Действительное перемещение --- бесконечно малое перемещение, совместимое со связями и уравнениями движения (и оно единственно как решение задачи Коши).
\end{dfn}

Если <<заморозить>> время, то есть <<забыть>> про частную производную $\pdv{f}{t}$,  то 
\[\left. \sum_i \pdv{f_j}{\vb{r}_i} \vb{v}_i = 0 \right| \cdot \dd{t} \quad \Rightarrow \sum_i \pdv{f_j}{\vb{r}_i} \var \vb{r}_i = 0 \footnote{$\var \vb{r}_i = \vb{v}_i \dd{t}$ --- дифференциал при замороженном времени.}\] 

\begin{dfn}
Виртуальное перемещение --- бесконечно малое перемещение, совместимое со связями при замороженном времени.
\end{dfn}

Виртуальному перемещению можем сопоставить виртуальную работу:
\[\{\var \vb{r}_i\} \longrightarrow \var A = \sum_i \vb{F}_i \var \vb{r}_i = \sum_i \vb{F}_i^{(a)} \var \vb{r}_i + \sum_i \vb{R}_i \var \vb{r}_i.\]
И оказывается, что почти в любом идеализированном механизме без трения $\var A_R = 0$.

\begin{dfn}
Связь называется идеальной, если $\var A_R = 0 \quad \forall \{\var \vb{r}_i\},$ то есть если виртуальная работа сил реакции связей равна нулю при любом виртуальном перемещении.
\end{dfn}
\noindent Примером идеальной связи может служить движение по гладкой неподвижной поверхности.


\textit{Эмпирическое утверждение.} Почти все связи в механике являются идеальными. Но трение  (попытка учесть немеханическое явление в механике) разрушает идеальность, и мы не знаем, как устроены связи, то есть мы обычно идеализируем задачи и почти всегда угадываем, но при этом сядем в лужу, если уйдём в очень большие масштабы --- в космологию, или в очень малые...

\begin{thm}
Пусть дана идеальная связь:
\begin{equation}
\begin{cases}
\sum_i \vb{R}_i \var \vb{r}_i = 0,\\
\sum_i \pdv{f_j}{\vb{r}_i} \var \vb{r}_i = 0 
\end{cases} \Longleftrightarrow
\exists\; \lambda_j (t): \; \vb{R}_i = \sum_{j=1}^K \lambda_j \pdv{f_j}{\vb{r}_i},
\end{equation}
то есть решили основную задачу механики.
\end{thm}
Перед доказательством применим сформулированную теорему, рассмотрим следствие из неё. 
\begin{cns} Посмотрим, как будут записываться уравнения движения.
\begin{equation}
\begin{cases}
m_i \ddot{r}_i = \vb{F}_i^{(a)} + \sum_{j=1}^K \lambda_j \pdv{f_j}{\vb{r}_i}\\
f_j (t, \{\vb{r}_i\}) = 0,
\end{cases}
\end{equation}
и эта задача уже математически корректна: в ней число переменных соответствует числу уравнений: $3N+K$ неизвестных, $3N+K$ соотношений. Уравнения движения в такой форме для несвободной системы называют \textit{уравнениями Лагранжа первого рода}. \index{Уравнения Лагранжа! первого рода}

И как эти уравнения можно решать?
Метод Лагранжа.
\begin{enumerate}
\item $\forall \lambda(t) \Rightarrow \vb{r}(t, \lambda)$ из уравнений движения.
\item $f(\{\vb{r}(t, \lambda\}, t) \equiv 0 \Rightarrow \lambda.$
\end{enumerate}
\end{cns}
\begin{rmk}
$\pdv{f}{t} = 0 \Rightarrow \pdv{\lambda}{f} = 0$
\end{rmk}

\newpage
\subsection{Линейные колебания в лагранжевых системах}
Начнём потихонечку искать решения уравнений Лагранжа.
\subsubsection{Одномерное движение}
Пусть нам дана одномерная (s =1) натуральная лагранжева система. Будем временно использовать букву $x $ вместо $q$. Имеем полином не выше второй степени (в силу натуральности системы):
\[L(t, x, \dot{x})  = \frac{1}{2} \alpha(x) \dot{x}^2 + \beta(x) \dot{x} - U(x),\]
уже есть некоторая нестыковка, потому что $\alpha, \beta$ могут зависеть от времени --- сделаем упрощения.
\begin{enumerate}
\item $\pdv{L}{t} = 0.$
\item Линейный по скорости член --- гироскопическая сила, но в одномерном случае никаких гироскопических сил не существует, поэтому можем этот член выкинуть, поскольку всегда найдётся $\beta,$ т. ч.
\[\beta \dot{x} = \dv{t} \int \beta \dd{x}.\]
\begin{dfn}[Состояние равновесия]
Состояние равновесия --- решение уравнения движения --- тождественная константа.
$x = x_0 = const\; (\dot{x} = \ddot{x} = \dots = 0).$
\end{dfn}
\item Диссипативных сил нет, то есть $\dv{t} \pdv{L}{\dot{x}} = \pdv{L}{x}$.
\end{enumerate}
Тогда 
\[\pdv{L}{\dot{x}} = \alpha(x) \dot{x} \Rightarrow \alpha \ddot{x} + \alpha' \dot{x}^2 = \pdv{L}{x} = \frac{1}{2} \alpha' \dot{x}^2 - U' \Rightarrow \]
\begin{gather}
 \alpha(x) \dot{x} + \hlf \pdv{\alpha}{x} \dot{x}^2 = - \pdv{U}{x} \Rightarrow \pdv{U(x_0)}{x} =0.\\ \intertext{Используем ещё существования стационарного решения в точке $x_0$:}
 \pdv{U(x_0)}{x} = 0\\
 \intertext{Опишем формально двжиение в окрестности точки $x_0$:}
x = x_0 + q \Rightarrow \alpha(x) = \alpha(x_0) + \dots; \quad \alpha(x_0) = m\\
U(x) = U(x_0) + U'(x_0)q + \hlf U'' (x_0)q^2 + \approx \hlf k q^2; \quad k = U'' (x_0) \Rightarrow\\
 L = \hlf m \dot{q}^2 - \hlf kq^2\\
 \intertext{Возникает вопрос, что значит <<мало>>, когда говорим о малости отклонения от положения равновесия. Поэкспериментировав, можно проверить, что неважно, где делать разложение: в функции Лагранжа или в уравнениях движения. Если будем учитывать следующие поправки, то у нас буду появляться следующие поправки к силе, которая уже учтена, и они по сравнению с ней должны быть малы. Второе замечание: коэфффициенты $k$ и $q$ должны быть невырожденными, иначе должны учитывать следующие члены в разложении, но тогда колебания уже будут нелинейными. Перепишем лагранжиан в эквивалентной форме:}
L = \hlf m \dot{q}^2 - \hlf kq^2 = \frac{m}{2} (\dot{q}^2 - \omega^2 q^2), \quad \omega^2 = k/m,\\
\intertext{соответствующее уранение движения}
\ddot{q} + \omega^2 q =0\\
\intertext{ линейное ОДУ с постоянными коэффициентами, порождаемое квадратичной формой. Есть стандартный способ решения таких уравнений:}
q = Ce^{\lambda t} \Rightarrow \lambda^2 C e^{\lambda t} + \omega^2 C e^{\lambda  t} = 0\\ \Rightarrow \lambda^2 + \omega^2 = 0 \Leftrightarrow \lambda = \pm i \omega \Rightarrow\\
q = C_1 e^{i \omega t} + C_2 e^{-i \omega t}\\  
\intertext{потребуем, чтобы}
q \in \mathbb{R} \Rightarrow C_2 = C_1^* \Leftrightarrow q = C_1 e^{i \omega t} + \text{к. с.} = 2\Re C_1 e^{i\omega t} = \Re C e^{i\omega t}, \quad C \in \mathbb{C}\\
C = c e^{i \varphi} \text{ --- комплексная амплитуда} \Rightarrow q = c \cos(\omega t + \varphi)
\end{gather}
\begin{thm}
Пусть $\Hat{D}$ --- дифференциальный оператор, имеем
\[\Hat{D} q = 0, \quad \Hat{D} \in \mathbb{R},\]
тогда мы всегда можем рассмотреть некое решение $X \in \mathbb{C},\; \Hat{D} X = 0$, автоматически
\begin{equation}
\begin{cases}
\Hat{D} (\Re X) = 0\\
\Hat{D} (\Im X) = 0
\end{cases}
\end{equation}
\end{thm}
\begin{proof}
\begin{align}
\Hat{D}& (\Re X + i \Im X) = 0 \Rightarrow\\
\Hat{D}& (\Re X) + i \Hat{D} (\Im X) = 0
\end{align}
\end{proof}
\begin{gather}
q= Ce^{\lambda t}\\
\ddot{q} + \omega^2 q = Ce^{\lambda t} (\lambda^2 + \omega^2) = 0 \Rightarrow q = Ce^{i \omega t} \Rightarrow q = \Re C e^{i \omega t}\\
\omega^2 > 0\quad q = c \cos (\omega t + \varphi)  = a \sin \omega t + b \cos \omega t,\\
\intertext{ через комплексные амплитуды:}
C = c e^{i  \varphi}.
\intertext{Квадрат $\omega$ больше нуля, когда $k$ больше нуля, потому что $m$ мы получаем из законов Ньютона и оно больше нуля, а вот $k$ <<выползает>> из потенциала, и может быть меньше нуля. Положительный квадрат частоты отвечает минимуму потенциальной энергии. }
\omega^2 < 0 \quad \lambda = \pm \sqrt{\abs{\omega}} \in \mathbb{R}\\
q = c_1 e^{\lambda t} + c_2 e^{- \lambda t} = a \sh \lambda t + b \ch \lambda t \underset{H/w}{\eqq{?}} c \sh(\lambda t + \varphi) \underset{H/w}{\eqq{?}} \tilde{c} \ch (\lambda t + \varphi)\\
\omega = 0 \quad q = c_1 t + c_2 \quad \ddot{q} = 0
\end{gather}
Продемонстрируем всю мощь метода комплексных амплитуд. В этих нескольких примерах будем выходить за рамки консервативных ($L = \frac{m}{2} (\dot{q}^2 - \omega^2 q^2),\; H = \frac{m}{2}(\dot{q} + \omega^2 q^2) = const$) систем. 
\begin{ex}[Осциллятор с трением.]
\begin{gather}
\ddot{q} + \omega_0 q + 2\gamma \dot{q} = 0,\\ \intertext{то есть рассматриваем линейный осциллятор с трением.}
q = C e^{\lambda t} \Rightarrow (\underbrace{\lambda^2 + \omega_0^2 + 2\gamma \lambda}_{0}) C e^{\lambda t} = 0\\
\lambda = -\gamma \pm \sqrt{\gamma^2 - \omega_0^2} = -\gamma \pm i\sqrt{\omega_0^2 - \gamma^2},
\intertext{в комплексном виде записали, чтобы был переход к незатухающему осциллятору.}
q = c e^{-\lambda t} \cos(\sqrt{\omega_0^2 - \gamma^2}t + \varphi),\\
\intertext{получили общее решение. $H/w\; \omega_0 = \gamma,\; \omega_0 < \gamma,\; \omega_0 > \gamma\, \text{(движение в меду).}$}
\end{gather}
\end{ex}
\begin{ex}[Осциллятор, на который действует внешняя сила.]
\begin{gather}
\ddot{q} + \omega^2 q = f(t)\\
\dot{q} + i\omega t = a(t) e^{i \omega t},\label{pl_h_1}\\
 a(t) \in \mathbb{C} \Rightarrow q(t) = \frac{1}{\omega} \Im a e^{i \omega t}\\
\begin{cases}
\dv{t}\left(\dot{q} + i\omega t \right) = (\dot{a} +  i \omega a)e^{i \omega t}\\
\eqref{pl_h_1}* i\omega: \quad - i\omega \dot{q} + \omega^2 q = -i \omega a e^{i \omega t}
\end{cases}
\Rightarrow \ddot{q} + \omega^2 q = \dot{a} e^{i \omega t} = f(t) \Rightarrow a(t) = \int^t f(t) e^{-i \omega t} \dd{t}\\
\intertext{Обратим внимание, что $a(t)$ очень похоже на преобразование Фурье.}
\end{gather}
\end{ex}

\begin{thm}[Появляющаяся сила.]
Пусть в некоторый момент на осцилллятор подействовала сила с конечным спектром (см. рисунок) $\int\limits_{-\infty}^{+\infty} f e^{-i \omega t} \dd{t} = F,$ тогда $H(+\infty) - H(-\infty) = \frac{m}{2} \abs{F}^2,$\\ где~$F$~--- спектральная компонента силы.
\end{thm}
\begin{proof}
H/w
\end{proof}
\begin{task}
\begin{gather}
\ddot{q} + 2\gamma \dot{q} + \omega_0^2 q = f(t)
\end{gather}
В частности, когда сила сама осциллирует: $f (t) = A \cos \omega t \Rightarrow q(t) = ?$.
\end{task}

\subsubsection{Многомерные системы}
\begin{gather}
L = \sum_{i, j} \hlf \alpha_{ij}(x) \dot{x}_i \dot{x}_j + \sum_i \beta_i(x)\dot{x}_i - U(x)
\end{gather}
Построим уравнение движения. Скажем, что $x$ --- тождественная константа --- отвечает случаю локального экстремума функции $U(x)$:
\[x = x_0 \equiv const \Leftrightarrow \pdv{U(x_0)}{x_i} = 0, \quad \forall i =\overline{1,s}\]
\begin{align}
x = x_0 + q \Rightarrow\; & \alpha_{ij}(x) \approx m_{ij} = \alpha_{ij} (x_0) \\
&U(x) \approx \sum_{i, j} \hlf k_{ij} q_i q_j; \quad k_{ij} = \pdv{U(x_0)}{x_i}{x_j}
\end{align}
Что можем сказать про коэффициенты $m_{ij}, k_{ij}$?
\begin{enumerate}
\item $m_{ij}$ симметричная ($m_{ij} = m{ji}$) и положительно определённая, $\alpha$ --- положительно определённая квадратичная форма, потому что произошла из кинетической энергии, и матрица постоянных коэффициентов $m$ унаследовала эти свойства.
\item $k_{ij} = k_{ji}:$ появилась по определению как смешанная производная, положительная определённость не гарантируется (достигается в случае локального минимума потенциала). 
\end{enumerate}
Осталось рассмотреть гироскопические силы. К полной производной, как в одномерном случае они сводиться не обязаны. Заметим, что 
\begin{gather}
\sum_i \beta_i(x_0) \dot{x}_i = \dv{t} \left(\sum \beta_i (x_0) x_i\right),\; \text{поэтому}\\
\beta_i(x) \approx \not{\beta_i(x_0)} + \sum_j \pdv{\beta_i(x_0)}{x_j} q_j; \quad g_{ij} = \pdv{\beta_i(x_0)}{x_j},
\end{gather}
подставим это всё в лагранжиан:
\begin{equation}
\boxed{L = \sum_{i, j = 1}^s \left\lbrace \hlf m_{ij} \dot{q}_i \dot{q}_j + g_{ij} q_j \dot{q}_i - \hlf k_{ij} q_i q_j \right\rbrace}. \label{Lagr_osc}
\end{equation}
\begin{ex}[$g_{ij} = 0$]
Рассмотрим лагранжиан \eqref{Lagr_osc} в частном случае, когда нет гиротропных сил, то есть $g_{ij} = 0$:
\begin{equation}
L = \sum_{i, j = 1} \left\lbrace \hlf m_{ij} \dot{q}_i \dot{q}_j  - \hlf k_{ij} q_i q_j \right\rbrace.
\end{equation}
Его можно рассматривать как две квадратичные формы, соответствующие двум слагаемым: первая симметричная и положительно определённая, вторая симметричная. Теорема из линейной алгебры утверждает, что такие кв. формы диагонализируемы одновременно.
\begin{thm}
$$
\left\{
\begin{array}{rcl}
\text{симм.$(+)$}\\
\text{симм.}
\end{array}
\right.
\Rightarrow \text{всегда диагонализуемы одновременно!}
$$
То есть $\exists\; \text{линейное преобразование}\; a_{ik}\; |\; q_i = \sum_i a_{ik} \theta_k, \dot{q}_i = \sum a_{ik} \dot{\theta}_k.$
\end{thm}
 тогда
\begin{equation}
L = \sum^s_k \lbrace \hlf m_k \dot{\theta}_k^2 - \hlf k_k \theta_k^2 \rbrace = \sum_k \frac{m_k}{2} \lbrace \dot{\theta}_k^2 - \omega^2_k \theta_k^2 \rbrace, \label{Lagr_ex_osc}
\end{equation}
где $\omega_k^2 = k_k / m_k,$ то есть система распадается на $s$ штук невзаимодействующих подсистем, каждая из которых есть  одномерный гармонический осциллятор.
Уравнение движения соответствующее \eqref{Lagr_ex_osc}:
\begin{gather}
\ddot{\theta}_k + \omega^2_k \theta_k = 0 \Rightarrow\\
\theta(t) = C_k \cos (\omega t + \varphi_k)  \approx \Re C_k e^{i \omega_ kt}
\end{gather}
\begin{dfn}
$\{\omega_k\}$ --- спектр нормальных частот $\omega_k,\; k = \overline{1, s}.$
\end{dfn}
\begin{dfn}
$\{\theta_k\}$ --- нормальные координаты.
\end{dfn}
Как выглядит решение? 
\begin{dfn}
Частное решение при $\theta_k = 0,$ кроме $k = k^* \Rightarrow$
\begin{equation}
q_j = a_{jk^*} \theta_{k^*}(t) 
\end{equation}
называют нормальными колебаниями.
\end{dfn}
Общее решение:
\begin{equation}
q_j (t) = \sum_{k=1}^s a_{jk} \theta_k (t).
\end{equation}
\begin{rmk}
Если мы возбудили одно нормальное колебание, то каждая степень свободы колеблется в одной и той же фазе. То есть, если у нас есть сложная многомерная система, и одна степень свободы проходит через ноль или экстремум, то остальные степени свободы тоже проходят через ноль или экстремум соответственно.
\end{rmk}
\begin{rmk}
Если мы живём на дне потенциального рельефа, то в \eqref{Lagr_ex_osc} две положительно определённые квадратичные формы, значит, $\omega_k^2 > 0\; \forall k,$ и у нас действительно колебания, то есть можем получить решения в виде синусов и косинусов, а не только экспонент.
\end{rmk}
\end{ex}
Построим уравнение движения для лагранжиана \eqref{Lagr_ex_osc}:
\begin{gather}
\pdv{L}{\dot{q}_k} =  \{ \frac{1}{2} m_{ik} \dot{q}_i + \frac{1}{2} m_{kj} \dot{q}_j + g_{kj}q_j \} = \sum_i \{ m_{ik} \dot{q}_i + g_{ki} q_i\}, \label{pdv_L_dotq_osc}\\
\pdv{L}{q_k} = \sum \{ g_{ik} \dot{q}_i - k_{ik} q_i \} \stackrel{!!!!!}{\Rightarrow} \label{pdv_L_qk}\\
\dv{t}\pdv{L}{\dot{q}_j} - \pdv{L}{q_j} = 0 \Rightarrow \sum_i \{ m_{ij} \ddot{q}_i + (g_{ij} - g_{ji})\dot{q}_i + k_{ij} q_i \} = 0.
\end{gather}
\begin{rmk}
$G_{ij} = -G_{ji}.$
\end{rmk}
Ищем решение в виде
\begin{equation}
q_i = \Re C_i e^{\lambda t}, 
\end{equation}
тогда
\begin{gather}
\Re \sum_i \{m_{ij} \lambda^2 + G_{ij} \lambda + k_{ij} \} C_i e^{\lambda t} = 0 \Leftrightarrow\\
\sum_i \{m_{ij} \lambda^2 + G_{ij} \lambda + k_{ij} \} C_i = 0. \label{eq_lambda}\\
\intertext{Эта линейная однородная алгебраическая система имеет невырожденное решение, когда детерминант матрицы её коэффициентов равен нулю:}
\det \left( m_{ij} \lambda^2 + G_{ij} \lambda + k_{ij} \right) = 0\; \text{--- характеристическое уравнение.}
\end{gather}
Поразмышляем о структуре решения. По размерности $P_{2s} (\lambda) =0,$ плюс, если $\lambda$ --- корень, то $\lambda^*$~--- тоже корень, так как $P_{2s} \in \mathbb{R},$ то есть все коэффициенты этого полинома действительные. На самом деле, уравнение движения консервативной системы накладывает ещё одно ограничение, и если расписать детерминант, то  можно получить, что решение имеет вид
$P_s (\lambda^2) = 0.$ Свойство чётности степеней --- свойство обратимости времени. Покажем, что решения действительно идут парами. Пусть $\lambda$ --- корень исходного характеристического уравнения, рассмотрим это же уравнение относительно $-\lambda$:
\begin{gather}
\det \left(m_{ij} (-\lambda)^2 +  G_{ij}(- \lambda) + k_{ij} \right) \Leftrightarrow\\
\det \left(m_{ij} \lambda^2 +  G_{ji} \lambda + k_{ij} \right) \Leftrightarrow\\
\det \left(m_{ji} \lambda^2 +  G_{ji} \lambda + k_{ji} \right) \label{new_det},\\
\intertext{поскольку $m_{ji}$ и $k_{ji}$ симметричные, и мы получили уравнение, выполняющееся тождественно, потому что $\lambda$ --- корень --- свойство антисимметричности члена, отвечающшего за гиротропию. А это означает, что $-\lambda$ --- тоже корень, что в точности и означает, что характеристическое уравнение имеет вид}
P_s(\lambda^2) =0.
\end{gather}
Для консервативной системы каждый корень порождает ещё три:
\begin{equation}
\lambda \longrightarrow \lambda^*, -\lambda, -\lambda^*.
\end{equation}

Поразмышляем, при каких условиях реализуются устойчивые колебания, а не какие-то экспоненты, описывающие неустойчивые состояния равновесия. Вернёмся к \eqref{eq_lambda}. Для анализа таких уравнений существует стандартный приём: умножим каждое уравнение на $C_j^*$, учтём, что
\begin{gather}
C_i C_j^* = (c_i' + i c_i'')(c_j' - i c_j'') = (\underbrace{c_i' c_j' + c_i''c_j''}_{S_{ij}}) + i(\underbrace{c_i'' c_j' - c_i' c_j''}_{A_{ij}}),\\
\intertext{•:}
\sum_{i, j} \boxed{m_{ij} S_{ij} \lambda^2 + iG_{ij} A_{ij} \lambda + k_{ij} S_{ij} = 0}
\end{gather}
\begin{enumerate}
\item гиротропии нет $k_{ij} (+); G_{iJ} = 0 \Rightarrow \lambda^2 = - \frac{k_{ij} S_{ij}}{m_{ij} S_{iJ}} < 0 \Rightarrow \lambda = \pm i\omega$ --- ситуация, когда существуют нормальные частоты и нормальные колебания в смысле именно колебаний.
\item Нет $(+) k_{ij} \Rightarrow \lambda^2 > 0 \Rightarrow \pm \lambda \Rightarrow c_1 e^{\lambda t} + c_2 e^{-\lambda t}$ --- состояние равновесия типа седло.
\item Можно показать, что гиротропия не может разрушить устойчивое состояние равновесия. $k_{ij}(+) \& G_{iо} \neq 0 \Rightarrow \lambda^2 < 0,$ то есть колебания устойчивые.
\end{enumerate}

Допустим, что мы научились решать характеристическое уравнение. Получим общее решение уравнения движения лагранжевой системы вблизи положения равновесия.
\begin{align}
P_s (\lambda^2) = 0 \Rightarrow \{&\lambda^2_k\} k=\overline{1, s}\\
&\lambda^2_k < 0 \Rightarrow \lambda_k = \pm i\omega_k\\
q_j^{(k)} &= \Re C_{jk} e^{i\omega_k t}\\
\end{align}
\begin{align}
\sum_{i=1}^s \left( m_{ij} \lambda^2 + G_{ij} \lambda + k_{ij} \right) C_i = 0 \quad j=\overline{1, s} \Rightarrow C_{jk} = a_{jk} e^{i \varphi_{jk}} &B_k\\
&\forall B_k = b_k e^{i\varphi_{0_k}}
\end{align}
Строим общее решение для $q,$ которое есть сумма всех нормальных колебаний:
\begin{align}
q_ j= \sum_{k=1}^s q_j^{(k)} &= \sum_k \Re b_k a_{jk} e^{i\omega_k t + i\varphi_{jk} + i \varphi_{0_k}}\\
\intertext{$\varphi_{jk} = 0,$ если $G = 0$ (нет гиротропии), тогда}
q_j &= \sum_k a_{jk} \underbrace{\Re B_k e^{i \omega_k t + i \varphi_{0_k}}}_{\theta_k (t)},\\
\intertext{то есть свели ответ к предыдущему.}
\end{align}
Вообще,
\begin{equation}
q_j = \sum_k \Re \left\{B_k C_{jk} e^{i\omega_k t} \right\}.
\end{equation}
Рассмотрим пару простых примеров.
\begin{ex}[Чашечка]
Пусть у нас есть движение в поле тяжести в окрестности минимума какой-то ямки $z = h(x, y).$ Заметим, что если $x =y = 0$ отвечают $\min h(x, y),$ то 
\begin{gather}
z = h(x, y) = \frac{x^2}{2\rho_1^2} + \frac{y^2}{2\rho_2^2} + \dots \quad \rho_{1, 2}\;\text{--- главные радиусы кривизны.}\\
\intertext{Составим лагранжиан:}
L = T - U = \frac{m}{2} \left(\dot{x}^2 + \dot{y}^2 + \dot{z}^2\right) - mg\left( \frac{x^2}{2\rho_1^2} + \frac{y^2}{2\rho_2^2}\right), \label{lagrangian_yamka}\\
\intertext{$\dot{z}^2$ нас не интересует, потому что речь идёт о малых колебаниях, поэтому \eqref{lagrangian_yamka} можно переписать в виде}
L = \frac{m}{2} \left\{\dot{x}^2 - \Omega_1^2 x^2 + \dot{y}^2 - \Omega_2^2 y^2 \right\}, \quad \Omega_{1, 2} = \frac{g}{\rho_{1, 2}^2}.\\
x = a \cos (\Omega_1 t + \varphi_1),\\
y = b \cos (\Omega_2 t + \varphi_2), \text{и эти колебания независимые.}
\end{gather}
\end{ex}
\begin{ex}[Вращающаяся чашечка]
Перейдём в систему координат $x, y$, которая прибита к чашке, и в ней уравнение чашки не изменится, но при этом в подвижной системе координат появятся дополнительные члены, связанные с вращением:
\begin{gather}
\vb{v}_{co} = [\vb*{\Omega}, \vb{r}] \Rightarrow\\
\begin{cases}
v_x = \dot{x} - \Omega y\\
v_y = \dot{y} + \Omega x
\end{cases}
\Rightarrow L = \frac{m}{2} \left\{(\dot{x} - \Omega y)^2 + (\dot{y} + \Omega x)^2 - \Omega_1^2 x^2 - \Omega_2^2 y^2 \right\},
\intertext{этот лагранжиан квадратичен по всем координатам и скоростям, и он содержит гироскопически члены (вида произведение координаты на скорость), а мы его перепишем:}
L = \frac{m}{2} \big\{ \dot{x}^2 + \dot{y}^2 + 2\Omega (x\dot{y} - y\dot{x}) - (\underbrace{\Omega_1^2 - \Omega^2}_{\widetilde{\Omega}_1^2})x^2 - (\underbrace{\Omega_2^2 - \Omega^2}_{\widetilde{\Omega}_2^2})y^2\big\}.\\
\begin{rcases}
\dv{t}\pdv{L}{\dot{x}}= m\ddot{x} - m\Omega\dot{y} = \pdv{L}{x} = m\Omega \dot{y} - m\widetilde{\Omega}_1^2 x\\
\dv{t}\pdv{L}{\dot{y}} = m\ddot{y} + m\Omega\dot{x} = \pdv{L}{y} = - m\Omega \dot{x} - m \widetilde{\Omega}_2^2 y
\end{rcases}
\Rightarrow\\
\begin{cases}
\ddot{x} - 2\Omega \dot{y} + \widetilde{\Omega}_1^2 x = 0\\
\ddot{y} + 2\Omega \dot{x} +  \widetilde{\Omega}_2^2 y =0
\end{cases}\\
x = C_1 e^{i \Omega t}\\
y = C_2 e^{i\Omega t}\\
\begin{cases}
\left(-\omega^{2}+\widetilde{\Omega}_1^2\right) C_{1}-2 \Omega i \omega C_{2}=0\\
2 \Omega i \omega  C_{1}+\left(-\omega^{2} + \widetilde{\Omega}_2^2 \right) C_{2}=0, \label{C1_C2_sys}
\end{cases}
\intertext{система \eqref{C1_C2_sys} имеет нетривиальное решение, когда определитель матрицы коэффициентов перед искомыми $C_1$ и $C_2$ равен нулю, тогда}
\boxed{ \left(\omega^{2}-\widetilde{\Omega}_{1}^{2}\right)\left(\omega^{2}-\widetilde{\Omega}_{2}^{2}\right)=4 \Omega^{2} \omega^{2}}.
\end{gather}
 Получили биквадратное уравнение относительно нормальных частот $\omega$. Проанализируем случай, когда
 \[\omega \ll \widetilde{\Omega}_1, \widetilde{\Omega}_2.\]
\end{ex}

До сих пор у нас был консервативный случай, и обобщённая энергия сохранялась:
\[H = const \quad H = \sum \frac{1}{2} \left\{m_{iJ} \dot{q}_i \dot{q}_j + k_{ij} q_i q_j\right\}.\]
\[H/w \quad \dv{H}{t} = 0 \Leftrightarrow P_s(\lambda^2) = 0\; \text{или}\; \lambda\; \text{и}\; -\lambda\; \text{--- корни одновременно.} \]

\subsubsection{Малые колебания в диссипативных системах}
\paragraph{Диссипативная функция Рэлея.} \index{Диссипативная функция Рэлея}
Линейное трение в лагранжевых системах обычно вводится следующим образом:
\[\vb{F}_i = -\sum_j \mu_{ij} \vb{v}_j,\]
и такие силы можно пересчитать в обобщённые силы, которые войдут в уравнение Лагранжа, с помощью этакого потенциала в пространстве скоростей, с помощью функции Рэлея $R(t, \{\vb{r}_i\}, \{\vb{v}_i\},$ например, такой функции:
\[\vb{F}_i = -\sum_j \mu_{ij} \vb{v}_j = - \pdv{R}{\vb{r}_i} \quad R = \frac{1}{2} \sum_{i, j} \mu_{ij} \vb{v}_i \vb{v}_j = \frac{1}{2} \gamma_{iJ} \dot{q}_i \dot{q}_j,\]
и в этом случае
\[Q_j = \sum_{i=1}^N \vb{F}_i \pdv{\vb{r}_i}{q_j} = - \sum_{i=1}^N \pdv{R}{\vb{v}_i} \pdv{\vb{v}_i}{\dot{q}_j} = - \pdv{R}{\dot{q}_j} = - \sum_{i=1}^n \gamma_{ij} \dot{q}_i,\]
и отличие от гироскопических сил только в том, что матрица коэффициентов здесь симметричная (по построению):
\[\gamma_{ij} = \gamma_{ji}.\]
А уравнение Лагранжа выглядит следующим образом:
\[\boxed{\frac{d}{d t} \frac{\partial L}{\partial \dot{q}_{j}}-\frac{\partial L}{\partial q_{j}}+\frac{\partial R}{\partial \dot{q}_{j}}=0}.\]
\[H/w \Rightarrow \sum_{j=1}^{s}\left\{m_{i j} \dot{q}_{j}+\left(G_{i j}+\gamma_{i j}\right) \dot{q}_{j}k_{ij} q_j\right\} = 0, \]
причём $G_{ij}$ --- антисимметричная часть (порождается функцией Лагранжа), гарантирует,\\ 
что $P_s (\lambda^2) =0;\; \dv{H}{t} =0;$ $\lambda_{ij}$ --- симметричная часть (порождается функцией Рэлея, и $P_{2s}(\lambda) =0$ --- есть нечётные степени, диссипация, и направления времени не эквиваленты, диссипация работает в обе стороны, система не может двигаться <<по кругу>>, $\dv{H}{t} = \sum Q_j \dot{q}_j = - \sum \pdv{R}{\dot{q}_j} \dot{q}_j = -2R,$ то есть физический смысл функции Рэлея в том, что она отвечает мощности потерь на соответствующей силе, которую она определяет, и в этом случае мы будем получать решения, как для осциллятора с трением, в виде
\begin{gather}
e^{\lambda t};\; \lambda = \lambda' + i\lambda'' \Rightarrow\\
e^{\lambda' t} \cos (\lambda'' t + \varphi),
\end{gather}
действительная часть корня характеристического уравнения описывает затухание в случае диссипации, а мнимая часть --- действительную часть частоты, свойство одновременной принадлежности к корням $\lambda$ и $\lambda^*$ сохраняется (потому что это свойство действительности коэффициентов), а $\lambda$ и $-\lambda$ --- нет.
\subsection{Вариационная форма механики Лагранжа}
\subsubsection{Введение в принцип наименьшего действия}
Раньше, получая уравнения Ньютона, мы исходили из принципа малых шажков. Математически это выражалось тем, что мы рассматривали дифференциальные уравнения Ньютона, и получали решения (уравнения движения как бесконечную сумму бесконечно малых шажков). Введение принципа уравнения Лагранжа и понятия идеальных связей позволили нам исключить из уравнений Ньютона силы реакции связей, которые мы не знаем, и мы получили уравнения Лагранжа второго рода. Это был дифференциальный подход.

Вариационная формулировка подразумевает рассмотрение траекторий как некое целое --- "интегральный подход".

\[L(t, q, \dot{q}, q = (q_1, \dots, q_s), s= 3N-k,\]
и нет непотенциальных обобщённых сил
\[Q^{\text{НП}} =0, \]
то есть наша система полностью описывается уравнением
\[\left(\frac{d}{d t} \frac{\partial L}{\partial \dot{q}_{j}} - \frac{\partial L}{\partial q_{j}}=0\right).\]
\begin{dfn}
$\{q\}$ --- конфигурационное пространство.
\end{dfn}
Линия $q(t)$ в конфигурационном пространстве --- то, что мы ищем --- траектория системы, эволюция её состояний во времени. Принцип наименьшего действия позволяет отсортировать настоящие и ненастоящие траектории.  Истинная траектория --- <<прямой путь>>.
\begin{gather}
q_1 = q(t_1),\\
q_2 = q(t_2).
\end{gather}
Как отличить истинную траекторию, которая отвечает уравнения Лагранжа, от всех остальных? В фазовом пространстве траектории не пересекаются нигде, кроме особых точек, пересечения (самопересечения) кривых в конфигурационном пространстве ничему не противоречат.

\subsubsection{Вариационный принцип Гамильтона для обобщенно-потенциальных систем}
Давайте каждой из траекторий по какому-то закону припишем число, а потом скажем, что истинной траектории отвечает конкретное число.

Отображение функций в числа \[q(t) \stackrel{S}{\rightarrow} \mathbb{R}\] называют функционалом.Чтобы не путать с функциями, пишут аргумент в квадратных скобках
\[S[q(t)] \rightarrow \mathbb{R}.\]

Длина кривой не позволяет выделять истинные траектории. Рассмотрим функционал
\begin{equation}
 \boxed{S[q(t)] = \int\limits L(t, q, \dot{q}(t)) \dd{t}}, \label{deystv}
 \end{equation}
 этот функционал называют функционалом \index{Функционал действия} действия (иногда просто действием). Конфигурационное пространство является общим для большого семейства систем с разными лагранжианами, но общими обобщёнными координатами.
 
 \begin{pst}
 Пусть
 \begin{enumerate}
 \item $q(t_1) = q_1, q(t_2) = q_2,$
 \item $\exists M:\; \abs{q_1 - q_2} < M$, 
 \end{enumerate}
 тогда $q(t),$ удовлетворяющее уравнению движения, отвечает наименьшему значению функционала действия $S[q(t)] \rightarrow min$ --- аналог того, что первая производная равна нулю, вторая производная знакоопределена. 
 \end{pst}
 
 \begin{pst}[Вариационный принцип Гамильтона]
 Пусть $q(t_1) = q_1, q(t_2) = q_2,$ тогда между этими положениями система движется так, что ФЛ принимает стационарные значения $S[q(t)] \rightarrow stat$ --- аналог того, что первая производная равна нулю.
 \end{pst}

Давайте рассмотрим малое возмущение (бесконечно близкую траекторию с невозмущённым  концами) $q(t) + \delta q(t),$ где $\delta q(t_1) =0,\; \delta q(t_2) = 0,\; \forall \delta q(t)$.
\begin{gather}
S[q(t)+\delta q(t)]-S[q(t)] \geqslant 0
\end{gather}
Мы не умеем находить экстремумы в пространстве функций, но умеем в пространстве чисел. Предположим, что
\begin{gather}
\delta q(t) = \alpha \cdot h(t),\\
h(t_1) = h(t_2) = 0,
\end{gather}
введём понятие 
\begin{equation}
S(\alpha) = S[q(t) + \alpha \cdot h(t) ] \geqslant,
\end{equation}
последнее неравенство ... то есть переформулировали ПНД в терминах задачи на экстремум в обычных переменных. На физическом уровне строгости напишем необходимое условие существования экстремума в точке $\alpha = 0$
\begin{gather}
S(\alpha) \rightarrow \pdv{S}{\alpha} = 0\; |_{\alpha = 0}.
\end{gather}
Распишем
\begin{gather}
\pdv{\alpha} \int\limits_{t_1}^{t_2} L (t, q + \alpha h, \dot{q} + \alpha \dot{h}) \dd{t} = \int\limits_{t_1}^{t_2} \left\{ \pdv{L}{q} h + \pdv{L}{\dot{q}}\dot{h} \right\}_{|_{\alpha = 0}} \dd{t} = \\
= \int \limits_{t_1}^{t_2} \left\{ \pdv{L}{q} - \dv{t} \left(\pdv{L}{\dot{q}}\right) \right\} h(t) \dd{t} +  \underbrace{\left. \pdv{L}{\dot{q}}\, h \right| _{t_1}^{t_2}}_{=0} = \\
=\int\limits_{t_1}^{t_2} \sum_{j=1}^{s}\left\{\frac{\partial L}{\partial q_{j}}-\dv{t} \left(\frac{\partial L}{\partial \dot{q}_j}\right)\right\} h_{j}(t) d t=0 \quad \text{по ПНД для $\forall h_j (t)$.} \Rightarrow\\
\forall\, j =1,s \quad \boxed{\dv{t} \left(\pdv{L}{\dot{q}_j} \right) - \pdv{L}{q_j}=0},
\end{gather}
то есть мы получили, что принцип наименьшего действия гарантирует, что движение по истинным траекториям удовлетворяет уравнению Лагранжа (или, что то же самое, что необходимое условие экстремума --- в точности то же, что уравнение Лагранжа).

Займёмся тем же, что проделали только что, однако не переходя к дифференцированию в числах,
\begin{equation}
S[q + \delta q] - S[q] = \int \limits_{t_1}^{t_2} \left\{ L(t, q + \delta q, \dot{q} + \delta \dot{q}) - L(t, q, \delta{q}) \right\} \dd{t},
\end{equation}
зафиксируем момент времени и применим разложение в ряд Тейлора, тогда
\begin{gather}
S[q + \delta q] - S[q] = \int\limits_{t_1}^{t_2} \left\{\pdv{L}{q} \delta q + \pdv{L}{\dot{q}} \delta \dot{q} + \ldots \right\} \dd{t} =,\\
\intertext{и проинтегрируем по частям}
= \int\limits_{t_1}^{t_2} \left\{ \pdv{L}{q} - \dv{t} \pdv{L}{\dot{q}} \right\} \delta q \dd{t} + \ldots .
\end{gather}
Написанное нами слагаемое называют \textit{первой вариацией функционала действия,} \index{Вариация} обозначают как $\var{S[q, \delta q]}$.


Согласно ВПГ
\begin{equation}
\var{S} = 0  \Longleftrightarrow \; \text{ур-ю Лагранжа II рода при фиксированных концах траектории.}
\end{equation}


А ПНД говорит, что
\begin{equation}
\text{ур-e Лагранжа II рода} \Longrightarrow \begin{cases}
\var S = 0,\\
\var^2{S} \geqslant 0 
\end{cases}
\text{при фиксированных концах, близости}\; q_1, q_2.
\end{equation}

\begin{rmk}[Обобщение на случай систем, не являющихся обобщённо-потенциальными]
Хотим, чтобы первая вариация приводила к уравнению $\frac{d}{d t} \frac{\partial T}{\partial \dot{q}}-\frac{\partial T}{\partial q}=Q$. Первые два слагаемые получаются из кинетической энергии $T(t, q, \dot{q}),$ второе --- из $\var{S} = \int\limits_{t_1}^{t_2} Q \dd{t},$ при этом \[Q = \sum_{j=1}^s Q_j \delta q_j = \sum_{i=1}^N \vb{F}_i^{(a)} \delta \vb{r}_i = \delta A^{(a)},\] значит,
\begin{equation}
S[q(t)] = \int\limits_{t_1}^{t_2} \left(T + A^{(a)}\right) \dd{t}.
\end{equation}

Частный случай, когда мы имеем дело с обобщённо-потенциальными силами в натуральных системах, в качестве работы активных сил может выступать обобщённый потенциал, то есть
\[S = \int\limits_{t_1}^{t_2} (T- U) \dd{t}.\] \textit{H/w показать, что, варьируя такой функционал, получим то же, что при варьировании такого функционала с функцией с Лагранжа в качестве аргумента.}
\end{rmk}

\subsubsection{Вариационный принцип для гармонического осциллятора}
Почему выбрали именно гармонический осциллятор? Задача на равенство нулю первой производной гораздо проще задачи на определение знака второй производной...

Пусть у нас одномерная система, которая задана лагранжианом
\[L = \frac{1}{2} (\dot{q}^2 - \omega^2 q.\]
Для это этой системы мы всё знаем:
\begin{gather}
\ddot{q} = -\omega^2 q,\\
q = a\sin (\omega t + \varphi).
\end{gather}
Рассмотрим малое возмущение $q+\delta q,$ посчитаем
\begin{equation}
S[q+\delta q] - S[q]. \label{functional_1}
\end{equation}
Будем считать $t_1 = 0,$ $t_2 = \tau,$ и если $\tau \neq n T/2,$ то по $q_1$ и $q_2,$ соответствующим заданным моментам времени мы траекторию однозначно определяем, иначе, при времени $\tau = n T/2$ ($T = \frac{2\pi}{\omega}$), называемом \textit{кинетическим фокусом} \index{Кинетический фокус},  получим бесконечно много решений, и все они будут удовлетворять уравнению движения. 
\begin{dfn}
Кинетический фокус сопряжённый некой начальной точке --- точка, в которую сходятся  две любые бесконечно близкие (пущенные с разными скоростями) траектории.
\end{dfn}

\begin{ex}[Кинетический фокус на сфере]
На глобусе между двумя точками есть наикратчайшее расстояние, но если мы рассмотрим две точки на диаметрально противоположные точки , то таких траекторий с наименьшей длиной будет бесконечно много, и диаметрально противоположная точка будет кинетическим фокусом.
\end{ex}

Нам надо определить
\begin{equation}
\begin{cases}
\delta q(0) =0\\
\delta q(\tau) = 0,
\forall \delta q(t)
\end{cases}
\Rightarrow \delta q(t) = \sum_{n=1}^\infty a_n \sin \left(n \frac{\pi}{\tau} t \right), \; \forall a_n \in \mathbb{R}.
\end{equation}
Вернёмся к \eqref{functional_1}
\begin{gather}
S[q+\delta q] - S[q] = \int\limits_0^\tau \frac{1}{2} \left( (\dot{q} + \delta \dot{q})^2 - \omega^2 (q +\delta q)^2  - \dot{q}^2 + \omega^2 q\right) \dd{t} =\\
=\int\limits_0^\tau \frac{1}{2} \left\lbrace \underbrace{2\dot{q} \delta \dot{q} - 2\omega^2 q \delta q}_{\dot{q}\delta \dot{q} + \ddot{q} \delta q = \dv{t}(\dot{q} \delta q)} + \delta \dot{q}^2 - \omega^2 \delta q^2 \right\rbrace \dd{t} =\\
= \int\limits_0^\tau \frac{1}{2} \sum_{n=1}^\infty a_n^2  \left\lbrace (n \frac{\pi}{\tau})^2 \cos^2 (n \frac{\pi}{\tau} t) - \omega^2 \sin^2 (\frac{n \pi}{\tau} t) \right\rbrace \dd{t} =\\
= \frac{\pi}{4} \sum_{n=1}^\infty a_n^2 (n^2 \Omega^2 - \omega^2), 
 \end{gather}
 и вся эта штука отвечает минимуму, если $\Omega > \omega,$ что равносильно тому, что $\tau < T/2.$ То есть, если траектория не содержит кинетического фокуса, то реализуется принцип наименьшего действия\footnote{Подробнее, но без доказательств, см. \cite{Atherman}, \S 5, Гл. 7.}.

Повторим общее утверждение. Если на действительной траектории не лежит кинетический фокус исходной точки, то функционал действия отвечает наибольшему (или наименьшему\footnote{Зависит от того, какой знак мы поставили перед лагранжианом.}) значению.

Если мы рассматриваем траекторию с кинетическим фокусом, то её можно разбить на кусочки без кинетических фокусов, и на каждом куске будет реализован минимум.

Если проводить вариацию только по траекториям, проходящим через кинетический фокус, то будет принцип наименьшего действия.

\subsubsection{Следствия вариационного принципа}
Чем же так хорош вариационный принцип?
У вариационного принципа есть два аспекта, которые сделали его в некоторой мере центральным для современной теоретической физики.

Сформулировав вариационный принцип, мы сумели абстрагироваться от системы координат --- технической вспомогательной конструкции, которая нужна уже при решении уравнений движения. Неизменность уравнений Лагранжа при изменении координат --- следствие вариационного принципа.

\begin{enumerate}
\item Пусть у нас есть $L(t, q, \dot{q})$, рассмотрим $L' = L + \dv \Phi (t, q)$. $L$ порождает функционал действия
\[S[q] = \int_{t_1}^{t_2} L \dd{t} \rightarrow \var{S [q, \delta q]} = 0 \Leftrightarrow \text{ур. Лагранжа на $q(t).$}\]
А теперь для $L'$
\[S'[q] = \int_{t_1}^{t_2} L' \dd{t} = S[q] + \left. \Phi (t, q) \right|_{t_1}^{t_2} \rightarrow \var{S'} = \var{S},\]
поскольку последняя подстановка не даёт вклада в вариацию, и вариации зануляются на одних и тех же значения $q$ и $t$.

\item Рассмотрим преобразование координат в уравнении Лагранжа.
\begin{gather}
L(t, q, \dot{q}) \rightarrow\\
\dv{t} \frac{\partial L}{\partial \dot{q}}-\frac{\partial L}{\partial q}=0 \stackrel{q^* = \varphi(t, q)}{\longrightarrow} \text{как будет выглядеть уравнение $q^* (t)$?}\\
\left(\dv{\tilde{\varphi}}{t} = \pdv{\tilde{\varphi}}{t} + \pdv{\tilde{\varphi}}{q^*}\dot{q}^*, \quad q = \tilde{\varphi}(t, q^*) \right)\\
S[q] = \int_{t_1}^{t_2} L \dd{t} \equiv \int_{t_1}^{t_2} \underbrace{L \left(t, \tilde{\varphi}(t, q^*), \dv{\tilde{\varphi}}{t}\right)}_{L^* (t, q^*, \dot{q}^*)} \dd{t} = S^* [q^* (t)]\\
\delta S[q] = \delta S^* [q^*] =0
\end{gather}
\begin{rmk}
Если система была натуральной $L = L_0 + L_1 + L_2,$ то она останется натуральной при замене координат.
\end{rmk}

\item Замена координат и времени. Была Лагранжева задача
\begin{equation}
L(t, q, \dot{q}) \rightarrow \text{ур. Лагранжа} \rightarrow \begin{cases}
q^* = \varphi(t, q)\\
t^* = \psi(t, q)
\end{cases}
\rightarrow \text{? ур-я $q^*(t^*)$}
\end{equation}
\begin{equation}
\begin{cases}
q=\tilde{\varphi}\left(t^{*}, q^{*}\right)\\
t = \widetilde{\psi}(t^*, q^*)
\end{cases}
\end{equation}

\begin{equation}
S[q(t)] = \int_{t_1}^{t_2} L(t, q, \dot{q}) \dd{t} = \int_{t_1^*}^{t_2^*} \underbrace{L\left(\widetilde{\psi}, \tilde{\varphi}, \dv{\tilde{\varphi}}{\widetilde{\psi}}\right) \dv{\widetilde{\psi}}{t^*}}_{L^* \left(t^*, q^*, \dv{q^*}{t^*}\right)} \dd{t^*} = S^* [q^*(t^*)],
\end{equation}
но этот функционал тождественно связан с исходным, поэтому
\[\var{S} \equiv \var{S^*},\]
следовательно, $L^{*}$ порождает уравнения Лагранжа, которые в точности равны исходным, в которых произведена замена <<$*$>>.
\begin{equation}
\begin{rcases}
\dd{q}=\pdv{\tilde{\varphi}}{t^*} \dd{t^{*}}+\pdv{\tilde{\varphi}}{q^*}  \dd{q^{*}}\\
\dd{t} = \pdv{\widetilde{\psi}}{t^*} \dd{t^*} + \pdv{\widetilde{\psi}}{q^*}  \dd{q^{*}}
\end{rcases}
\Rightarrow
L^{*} = L \left( \widetilde{\psi}, \tilde{\varphi}, \frac{\pdv{\tilde{\varphi}}{t^*} +\pdv{\tilde{\varphi}}{q^*} \dv{q^*}{t^*}}{\pdv{\widetilde{\psi}}{t^*} + \pdv{\widetilde{\psi}}{q^*} \dv{q^*}{t^*}}\right) \cdot \Bigg(\underbrace{\pdv{\widetilde{\psi}}{t^*} + \pdv{\widetilde{\psi}}{q^*} \dv{q^*}{t^*}}_{\dv{t}{t^*}} \Bigg) \label{monster}
\end{equation}
\begin{rmk}
Система не остаётся натуральной!
\end{rmk}
\end{enumerate}

\begin{ex}
\begin{multicols}{2}
Натуральная система (Ньютоновская механика).
\[\vb{p}= m \vb{v} = \pdv{L}{\vb{v}} \Rightarrow L = \frac{mv^2}{2}.\]
\begin{equation}
\begin{rcases}
H = \vb{p} \cdot \vb{v} -L = L\\
H = T 
\end{rcases}
\Rightarrow L = T\; (\text{натуральность})
\end{equation}

Не натуральная система (релятивистская механика). $H/w$
\begin{gather}
\vb{p} = \frac{m\vb{v}}{\sqrt{1- v^2/c^2}} = \pdv{L}{\vb{v}} \Rightarrow L = - mc^2 \sqrt{1 - v^2/c^2}\\
H = \vb{p} \cdot \vb{v} - L = \frac{mv^2 + mc^2}{\sqrt{\ldots}} = \frac{mc^2}{\sqrt{1 - v^2/c^2}}\\
H = T \Rightarrow L \neq T\; (\text{ненатур.})
\end{gather}
\end{multicols}
\end{ex}

\subsubsection{Вариационный принцип и законы сохранения (теорема Нётер)} \index{Теорема! Нётер}
Связь уравнений, которые следуют из вариационного принципа, с законами сохранения --- вторая глобальная фишка, делающая ВП центральным в современной теоретической физике, первая --- уравнения, следующие из ВП ковариантны, то есть не зависят от ввода системы координат. 

Существует неразрывная связь симметрий в системе и уравнений движения. Теорема Нётер позволяет связать симметрии и следующие их них интегралы движения способом, не зависящим от способа ввода координат.

\begin{thm}[Теорема Нётер]
Каждому преобразованию координат и времени, оставляющему функционал действия инвариантным, отвечает первый интеграл уравнения движения.

Математически, если дано преобразование координат и времени
\begin{gather}
q^* = \varphi(t, q, \alpha)\\
t^* = \psi (t, q, \alpha),
\end{gather}
которое удовлетворяет следующим трём пунктам  условий,
\begin{enumerate}
\item $q = q^*,$ $t = t^*$ при $\alpha = 0.$

\item При $\abs{\alpha} < \varepsilon$ существуют обратные преобразования $\tilde{\varphi}(t^*, q^*, \alpha) = q,$ $\widetilde{\psi}(t^*, q^*, \alpha) = t.$

\item  (Симметрии.) $\Phi$-инвариантность\footnote{Используемая дальше функция $L^*$ определяется уравнением \eqref{monster}.} $L^* \left(t^*, q^*, \dv{q^*}{t^*}\right) = L \left(t^*, q^*, \dv{q^*}{t^*}\right) + \dv{t^*} \Phi(t^*, q^*, \alpha)$. \footnote{Эквивалентное утверждение для действия $S[q] = S^*[q^*] = S[q^*] + \left.\Phi \right|_{t_1^*}^{t_2^*}$.}
\end{enumerate}

то на решениях системы существует первый интеграл уравнения Лагранжа
\[I(t, q, \dot{q}) \left\{ \sum_{j=1}^s \pdv{L}{\dot{q}_j} \pdv{\varphi_j}{\alpha} - H \pdv{\psi}{\alpha} + \pdv{\Phi}{\alpha} \right\}_{\alpha = 0} = const.\]
\end{thm}

\begin{ex}[Преобразование сдвига]
\begin{equation}
x^* = x + \alpha, \Phi = 0 \Rightarrow I = \pdv{L}{\dot{x}} \left(\pdv{X^*}{x}\right)_{\alpha = 0} = p_x.
\end{equation}
\end{ex}
\begin{ex}[Поворот]
\begin{equation}
\begin{cases}
x^* = x \cos \alpha + y \sin \alpha\\
y^* = -x\sin \alpha + y \cos \alpha,
\end{cases}
\Phi = 0 \Rightarrow I = p_x \pdv{x^*}{\alpha} + p_y \pdv{y^*}{\alpha} = yp_x - xp_y = M_z
\end{equation}
\end{ex}
\begin{ex}[Винтовая симметрия]
К повороту из предыдущего примера добавим сдвиг $z^* = z + \frac{h}{2\pi} \alpha$, тогда
$I = M_z + p_z \frac{h}{2\pi}.$
\end{ex}
\begin{ex}
\[t^* = t + \alpha \Rightarrow I = -H.\]
\end{ex}
\begin{ex}[Преобразования Галилея]
\begin{equation}
\begin{cases}
x^* = x - \alpha t \quad(\alpha = u)\\
t^* = t
\end{cases}
\Rightarrow \pdv{x^*}{\alpha} = -t,
\end{equation}
при этом 
\begin{gather}
L = \frac{m}{2} \dot{x}^2,\; \dot{x} = \dot{x}^* + \alpha,\\
L^* = L(\dot{x}^* + \alpha) = \underbrace{\frac{m}{2} \dot{x}^*}_{L^*(\dot{x}^*)} + \underbrace{m \alpha \dot{x}^* + \frac{m}{2} \alpha^2}_{\dv{t}\left(\alpha mx^* + \frac{m\alpha^2}{2}t\right)} \Rightarrow\\
\Rightarrow I = p_x (-t) + mx = const\; (= 0),\\
p_x = m \frac{x}{t}.
\end{gather}

\end{ex}



\begin{proof}[Доказательство теоремы Нётер]
Попробуем свойства $\Phi$-инвариантности записать в пределе $\alpha \to 0$. Разложим в ряды преобразования координат  и времени
\begin{align}
q^* &= \varphi(t, q, \alpha) = q + \alpha \cdot Q + \ldots & Q_j &= \left. \pdv{\varphi_j}{\alpha} \right|_{\alpha = 0}\\
t^* &= \psi (t, q, \alpha) =  t + \alpha \cdot T + \ldots & T &= \left. \pdv{\psi}{\alpha} \right|_{\alpha = 0}.
\end{align}
Заметим, что 
\begin{equation}
\dv{q^*}{t^*} = \frac{\dot{q} + \alpha \dot{Q} + \ldots}{1 + \alpha \dot{T} + \ldots} = \dot{q} + \alpha (\dot{Q} - \dot{q} \dot{T}) + \ldots.
\end{equation}
По формуле \eqref{monster} получим выражение для $L^*$
\begin{gather}
L^* = L\left(t^* - \alpha T, q^* - \alpha Q, \dv{q^*}{t^*} - \alpha(\dot{Q} - \dot{q} \dot{T}) \right) \cdot \left(1 - \alpha \dot{T} \right) =\\
= L(t^*, q^*, \dv{q^*}{t^*}) - \alpha \bigg\{ \underbrace{\pdv{L}{t}}_{= - \dot{H} T} T + \underbrace{\pdv{L}{q} Q}_{\dot{p} Q} + \pdv{L}{\dot{q}} \left(\dot{Q} - \dot{q} \dot{T}\right) + L\dot{T}  \bigg\} =\\
= L(*)  - \alpha \left\{-\dv{t} (HT) + \dv{t}(pQ)\right\} + \ldots, \label{pre_phi_inv}
\end{gather}
так как $\pdv{L}{\dot{q}} \dot{Q} = p \dot{Q},$ а $-\pdv{L}{\dot{q}} \dot{q} \dot{T} + L\dot{T} = \dot{T} (L - \dot{q} p) = - H\dot{T}$. А дальше воспользуемся $\Phi$-инвариантностью, и перепишем полученное для $L^*$ выражение \eqref{pre_phi_inv}
\begin{equation}
 L^* = L(*) + \underbrace{\dv{t^*} \Phi(t^*, q^*, \alpha)}_{\dv{t}\left(\alpha \left. \pdv{\Phi}{\alpha}\right|_{\alpha = 0} + \ldots\right)}.
 \end{equation} 
 Приравняем одинаковые порядки, тогда
 \begin{equation}
 \dv{t} \left\{ pQ - HT + \left. \pdv{\Phi}{\alpha} \right|_{\alpha = 0} \right\} = 0,
 \end{equation}
а дифференцируемое выражение в точности и есть сохраняющийся интеграл движения.
\end{proof}

\subsubsection{Как вариационная формулировка работает в приближенных к реальным условиях?}
\[L(t, q, \dot{q}) \Rightarrow S[q(t)] = \int_{t_1}^{t_2} L \left(t, q(t), \dot{q}(t)\right) \dd{t}\]
\begin{gather}
\text{ур. Лагранжа}\; \Leftrightarrow \begin{cases}
\var{S} = 0\\
\var{q(t_1)} = \var{q(t_2)} = 0
\end{cases}
\text{(ВПГ)}\quad
\oplus \quad
\begin{cases}
q_1, q_2\; \text{близки так, что нет кин. фокусов,}\\
\text{сопряжённых с началом (концом) траектории}\\
S \to min\; (max)
\end{cases}
\end{gather}

Что мы выигрываем, зная, что уравнения Лагранжа следуют из решения вариационной экстремальной задачи?
\begin{enumerate}
\item Ковариантность относительно замены координат.
\item Теорема Нётер. 
\begin{gather}
\begin{cases}
q^* = q + \alpha \cdot Q(t, q) + O(\alpha^2)\\
t^* = t + \alpha \cdot T(t, q) + O(\alpha^2)
\end{cases}
+ \text{$\Phi$-инфариантность}\; L^{*} = L \left(t^*, q^*, \dv{q^*}{t^*}\right) + \dv{t^*} \Phi (t^*, q^*, \alpha),\\
\left(t\; \text{и}\;q\; \text{не равны нулю одновременно,}\;\Phi = \Phi_0 + \alpha \cdot \Xi + O(\alpha^2)\right),
\end{gather}
тогда существует интеграл движения
\begin{gather}
\pdv{L}{\dot{q}}Q + H \cdot T + \Xi = const\; \text{на реш. ур. Лагранжа.}
\end{gather}
\begin{rmk}
\begin{gather}\begin{cases}
q^* = \varphi(q, t)\\
t^* = \psi(q, t)\\
\ldots
\end{cases} \Rightarrow
\begin{cases}
Q = \left. \pdv{\varphi}{\alpha} \right|_{\alpha= 0},\\
T = \left. \pdv{\psi}{\alpha} \right|_{\alpha = 0},\\
\Xi = \left. \pdv{\Phi}{\alpha} \right|_{\alpha= 0}
\end{cases}
\end{gather}
\end{rmk}
\end{enumerate}

\subsubsection{Лагранжиан свободной материальной точки.}\index{Лагранжиан! свободной материальной точки}
\paragraph{В Ньютоновской механике}\! из дифференциального подхода для свободной материальной точки мы знаем
\begin{equation}
L = \frac{mv^2}{2},
\end{equation}
но можно стартовать с вариационного принципа, и последний не утверждает, что $L = T -U,$ стартуем с того, что система описывается $L(t, q, \dot{q}),$ выведем механику сил.
Воспользуемся теми же принципами, что и для Ньютоновской механики.
\begin{enumerate}
\item Пространство однородно и изотропно. Время однородно. $\Rightarrow L (\not{t}, \not{\vb{r}}, \vb{v}) = L(v^2).$
Зная, что у нас $L(v^2),$ можем доказать первый закон Ньютона:
\begin{gather}
\vb{p} = \pdv{L}{\vb{v}} = \pdv{L}{v^2} \pdv{v^2}{\vb{v}} = \pdv{L}{v^2} \cdot 2\vb{v} = const\; (\text{$\vb{r}$  --- цикл.}) \Rightarrow v =const, \vb{v} = const\\
H = \vb{p}\cdot \vb{v} - L = 2v^2 \pdv{L(v^2)}{v^2} - L = const\; (\text{$t$  --- цикл.}),
\end{gather}
двумя способами получили :)
\item Чтобы воспользоваться теоремой Нётер, надо придумать какое-то преобразование, оставляющее уравнение движения инвариантным, вспомним про преобразование Галилея, которое тоже как бы свойство пространства-времени, делающее эквивалентными все инерциальные системы отсчёта в Ньютоновской механике. 
\begin{gather}
\begin{cases}
t^* = t\\
r^* = \vb{r} - \vb{u} \cdot t;\; \vb{u} = \alpha \cdot \vb{n}_{const}
\end{cases}
\stackrel{Th.\, Noether}{\longrightarrow} 
\begin{cases}
T = 0\\
Q = - \vb{n}\cdot t
\end{cases} \Rightarrow
\boxed{-\pdv{L}{v^2} 2(\vb{n} \cdot \vb{v})t = F (t, \vb{r})}\dv{t} \rightarrow\\
\pdv{L}{v^2} \cdot 2(\vb{n} \cdot \vb{v}) = \pdv{F(t, \vb{r})}{t} + \pdv{F(t, \vb{r})}{\vb{r}}\vb{v},\\
\intertext{но левая часть полученного равенства не зависит от времени и координат, поэтому правая тоже от них не зависит, значит,}
\pdv{L}{v^2} \cdot 2(\vb{n} \cdot \vb{v}) = \pdv{F(t, \vb{r})}{t} + \pdv{F(t, \vb{r})}{\vb{r}}\vb{v} = a + (\vb{b}, \vb{n}) \stackrel{\vb{b} =\vb{n}\cdot m}{=} m (\vb{n}, \vb{v})\\
\left(H/w\quad (\vb{n}, \vb{v}) = a + (\vb{b}, \vb{v}) \Rightarrow a =0, \vb{b} = \vb{n} \right)
\end{gather}
и мы, хитро подобрав константы, получили то, что нам надо в ответе:
\[\pdv{L}{v^2} = \frac{m}{2} \Rightarrow L = \frac{mv^2}{2} + const.\]
Ландау и Лифшиц дальше для несвободной материальной точки постулируют силы, то есть к лагранжиану свободной материальной точки аддитивно добавляют взаимодействие с внешними полями
\[L = \frac{mv^2}{2} - U(t, \vb{r}, \vb{v}) \ldots,\]
возможность так делать постулируется.
\end{enumerate}

\paragraph{В СТО}\! то же, что в предыдущем пункте, но с преобразованиями Лоренца
\begin{equation}
\begin{cases}
\vb{r}^* = \frac{\vb{r}-\vb{u} t}{\sqrt{1-u^{2} / c^{2}}}\\
t^* = \frac{t-(\vb{u}, \vb{r}) / c^{2}}{\sqrt{1-u^{2} / c^2}}.
\end{cases}
\stackrel{\vb{u} = \alpha \cdot \vb{n}}{\Longrightarrow}
\boxed{
\begin{cases}
\vb{r}^* = \vb{r} - \alpha \vb{n} \cdot t + \ldots\\
t^* = t - \alpha (\vb{n}, \vb{r})/c^2 + \ldots
\end{cases}}.
\end{equation}
По-прежнему время и пространство однородны, пространство изотропно. Значит, как и раньше,
\[\vb{p} = 2\vb{v} \pdv{L}{t^2}; H = 2v^2 \pdv{L}{v^2} - L \Rightarrow \vb{p} = const, \vb{v} = const, v = const.\]
Чтобы узнать форму $L(v^2)$, воспользуемся теоремой Нётер
\begin{gather}
\vb{p} Q - H \cdot T = - F(t, \vb{r})\\
-2(\vb{n}, \vb{v}) \pdv{L}{v^2} \cdot t + \left(2v^2 \pdv{L}{v^2} - L\right) \frac{(\vb{n}, \vb{r})}{c^2} =  F(t, \vb{r})\; \left| \dv{t} \right.\\ 
(\vb{n}, \vb{v}) \left\{ \left(2v^2 \pdv{L}{v^2} - L\right)\frac{1}{c^2} - 2\pdv{L}{v^2}\right\} = \underbrace{\pdv{F}{t}}_{=0} + \underbrace{\pdv{F}{\vb{r}}\vb{v}}_{\parallel \vb{n}} = (\vb{n}, \vb{v}) \cdot \underbrace{\frac{L_0}{c^2}}_{const}\\
-\pdv{L}{v^2}\left(1-\frac{v^{2}}{c^{2}}\right)=\frac{1}{2 c^{2}}\left(L-L_{0}\right)
\end{gather}
\begin{gather}
\pdv{\ln (L-L_0)}{v^2} = \hlf \frac{1}{v^2 - c^2} = \hlf \pdv{v^2} \ln \abs{v^2 - c^2}\\
L = L_0 + A \sqrt{c^2 - v^2} \to \frac{mv^2}{2}\; \text{при $v \to 0$}\; \stackrel{cA= - mc^2}{\Rightarrow} \boxed{L = - mc^2 \sqrt{1 - v^2 / c^2} + mc^2} \Rightarrow\\
H = \frac{mc^2}{\sqrt{1- v^2/c^2}} \neq L,
\end{gather}
то есть $L \neq T,$ система ненатуральная; $\vb{p} = \frac{m\vb{v}}{\sqrt{1- v^2/c^2}}.$

\subsection{Электромеханические аналогии}
\textbf{(Смешанные электромеханические системы в механике Лагранжа)}
\begin{ex}
\begin{equation}
U = \mathcal{E},\; U = \frac{q}{C},\; \mathcal{E} = -L\dot{I} = - L \ddot{q} \Rightarrow
\end{equation}
\begin{gather}
L \ddot{I} + \frac{1}{C} q = 0\; \text{ --- гармонический осциллятор с $\omega^2 = \frac{1}{LC},$}\\
L \ddot{I} + \frac{1}{C} I = 0\; \left| \cdot CL \right. \Longrightarrow CL^2 \ddot{I} + L I = 0.
\end{gather}
Будем рассматривать заряд на обкладках в качестве обобщённой координаты, тогда, рассматривая индуктивность в роли массы, запишем функцию Лагранжа
\begin{equation}
\mathcal{L} = \hlf L\dot{q}^2 - \frac{1}{2C} q^2 = T -U: 
\begin{cases}
T = \hlf LI^2\; \text{--- энергия магнитного поля в катушке,}\\
U = \frac{1}{2C} q^2\; \text{--- энергия энергия электрического поля в конденсаторе.}
\end{cases}
\end{equation}
Точно так же в качестве обобщённой координаты можно рассматривать ток (тогда роль массы играет выражение $CL^2$):
\begin{equation}
\mathcal{L} = \hlf CL^2 \dot{I}^2 - \hlf LI^2 = T - U,
\begin{cases}
T = \hlf CU^2,\\
U = \frac{1}{2} LI^2,
\end{cases}
\end{equation}
и, когда контур замкнутый, разницы никакой нет. Разница возникает, если контур разомкнуть. $H/w$
\end{ex}

\begin{ex}
\begin{gather}
\mathcal{L}_{\text{мех}} = \frac{m \dot{x}^{2}}{2}-\frac{k x^{2}}{2}-\operatorname{mg} x\\
\mathcal{L}_{\text{эл}} = \frac{1}{2} L(x) \dot{q}^{2}-\frac{1}{2 C(x)} q^{2},\\
\left(C(x) = \frac{S_C}{4\pi x}, L(x)  = 4\pi \frac{N^2 S_L}{L-x} \right)\\
\mathcal{L} = \mathcal{L}_{\text{мех}} + \mathcal{L}_{\text{эл}}
\end{gather}
Получим уравнения Лагранжа.
\begin{equation}
\dv{t} \pdv{\mathcal{L}}{\dot{x}} = m\ddot{x} = \pdv{\mathcal{L}}{x} = - kx -mg + \frac{1}{2}\pdv{L}{x} \cdot \dot{q}^2 - {\frac{q^2}{2} \pdv{x} \frac{1}{C}}\footnote{$-\frac{q^2}{2} \frac{4\pi}{S_C} = -2\pi \sigma q = Eq$ --- в точности электрическая сила взаимодействия двух пластин конденсатора. Можно показать, что в точности совпадает с силой сопротивления сжатию соленоида предыдущий член, да и все слагаемые могут быть получены из первых принципов.}
\end{equation}
\end{ex}
У нас теперь две степени свободы. Найти $q$  и $\dot{q}$ можно из второго уравнения движения:
\begin{equation}
\dv{t} \pdv{\mathcal{L}}{\dot{q}} = \dv{t} (L(x) \dot{q}) = L \ddot{q} + \pdv{L}{x} \dot{x} \dot{q} = \pdv{\mathcal{L}}{q} = \frac{q}{C}.
\end{equation}

\section{Интегрируемые задачи механики}
\subsection{Интегрирование уравнения движения систем с одной степенью свободы}
Мы будем интересоваться решением ДУ вида $\ddot{x} = F(t, x, \dot{x}).$\footnote{$x= 1;\; \dim x =1, s =1$}
\subsubsection{Классификация состояний равновесия автономной системы на плоскости}
\begin{equation}
\ddot{x} = F(x, \dot{x}) \Rightarrow \begin{cases}
\dot{x} = f(x, y)\\
\dot{y} = g(x, y)
\end{cases}
\end{equation}
Состояние равновесия
\begin{equation}
\begin{cases}
\dot{x} = 0\\
\dot{y} = 0
\end{cases}
\Leftrightarrow
\begin{cases}
f(x_0, y_0) =0\\
g(x_0, y_0) = 0.
\end{cases}
\end{equation}
Рассматриваемое малое возмущение
\begin{equation}
\begin{cases}
x = x_ 0 + \xi\\
y = y_0 + \eta 
\end{cases}
\Rightarrow
\begin{cases}
\dot{xi} = \alpha \xi + \beta \eta\\
\dot{\eta} = \gamma \xi + \delta \eta,
\end{cases}
\alpha = \left. \pdv{f}{x} \right|_{x_0, y_0}, \beta = \left. \pdv{f}{y} \right|_{x_0, y_0}, \ldots, \delta = \left. \pdv{g}{y} \right|_{x_0, y_0}.\footnote{Рассматриваем линейное состояние равновесия.}
\end{equation}
Будем искать решение в виде 
\begin{gather}
\begin{pmatrix}
\xi\\
\eta
\end{pmatrix} = 
\begin{pmatrix}
a\\
b
\end{pmatrix}
e^{\lambda t} \Rightarrow \begin{cases}
\lambda a = \alpha a + \beta\\
\lambda b = \gamma a + \delta b
\end{cases} \Rightarrow 
\det \begin{pmatrix}
\alpha - \gamma & \beta\\
\gamma & \delta - \lambda
\end{pmatrix}
= 0 \Leftrightarrow\\
\Leftrightarrow (\lambda - \alpha)(\lambda - \delta) = \gamma \beta,\\
\lambda_{1,2} = \frac{\alpha + \beta}{2} \pm \sqrt{\left(\frac{\alpha - \delta}{2}\right)^2 - \gamma \beta}.
\end{gather}
\begin{enumerate}
\item $\lambda = \pm i \omega$ --- состояние равновесия типа центр, устойчивое, не асимптотически устойчивое. Фазовые кривые --- эллипсы.
\item $\lambda p \pm i \omega$ --- состояние равновесия типа фокус, может быть в зависимости от знака $p$ как устойчивым, так и неустойчивым\footnote{Устойчиво при $p < 0$.}, устойчивое равновесие асимптотически устойчиво. Фазовые кривые <<сжимаются>> или <<раскручиваются>>.
\item $\lambda \in \mathbb{R}, \lambda_1 \cdot \lambda_2 > 0$ ---состояние равновесия типа узел --- равновесие типа фокус при очень сильном трении. Асимптотически устойчиво либо неустойчиво.
\item $\lambda \in \mathbb{R}, \lambda_1 \cdot \lambda_2 < 0$ --- состояние равновесия типа седло, неустойчиво.
\end{enumerate}
$H/w$ Как найти направления асимптот?

\begin{ex}
\begin{gather}
\ddot{x} = f(x, v), v =0, m = 1 \\
\begin{cases}
\dot{x} = v\\
\dot{v} = f_0(x) - \mu \cdot v = - \pdv{U(x)}{x} - \mu \cdot x
\end{cases}\\
\pdv{U(x_0)}{x} = 0; \pdv[2]{U(x_0)}{x} = U'' \neq 0; \begin{cases}
x = x_0 + \xi\\
v = \eta
\end{cases}
\Rightarrow\\
\begin{cases}
\dot{\xi} = v\\
\dot{v} = -U'' \cdot \xi - \mu \cdot v
\end{cases} \Rightarrow 
\det \begin{pmatrix}
-\lambda & 1\\
-U'' & -\lambda - \mu
\end{pmatrix} = 0 \Rightarrow \boxed{\lambda^2 + \mu \lambda + U'' = 0} \Rightarrow \\
\lambda_{1, 2} = - \frac{\mu}{2} \pm \sqrt{\frac{\mu^2}{4} - U''}
\end{gather}
Чередуются состояние равновесия типа седло, узел и фокус, центры превращаются в диссипативные состояния равновесия, фокусы отвечают случаям малого трения, при её увеличении превращаются в узлы.
\end{ex}

\subsubsection{Интегрирование уравнения движения консервативных одномерных систем}
Будем рассматривать лагранжеву задачу, в которой существует интеграл энергии.
\[L(x, \dot{x})\; \text{--- уравнения Лагранжа решаются в квадратурах.}\]
\begin{equation}
\pdv{L}{t} = 0 \Rightarrow H(x, \dot{x}) = const\; \text{на решениях}.
\end{equation}
Фазовый портрет --- линии уровня $H(x, \dot{x})$.\index{Фазовый портрет}
\begin{gather}
H(x, \dot{x}) = E = const \Rightarrow \dot{x} = V(x, E)\\
\dv{x}{t} = V(x, E) \Rightarrow \dd{t} = \frac{\dd{x}}{V(x, E)} \Rightarrow \boxed{t= \int^x \frac{\dd{x}}{V(x, E)}},
\end{gather}
аддитивная константа в нижнем пределе интегрирования отвечает заданному $E$, его линии уровня, и такие точки надо различать, поэтому на самом деле везде в решении вместо $V(x, E)$ должно быть $V_i (x, E),$ где $i$ определяет \textit{ветку однозначности}.\index{Ветка однозначности}

\begin{dfn}
$x$ --- ограниченная область --- движение финитное; $x$ --- неограниченная область --- движение инфинитное.
\end{dfn}
\begin{pst}
Всякое финитное движение в одномерной консервативной системе обязательно является периодическим. 
\end{pst}

Частный случай --- натуральная система.
\begin{gather}
L(x, \dot{x}) = \frac{1}{2} m(x) \dot{x}^2 - U(x)\\
H(x, \dot{x}) = \frac{1}{2} m \dot{x}^2 + U(x) = E \Rightarrow \dot{x} = \pm \sqrt{\frac{2}{m}\left(E- U(x)\right)}\\
\boxed{t = \pm \int^x \sqrt{\frac{m(x)}{2} \frac{\dd{x}}{\sqrt{E - U(x)}}}},
\end{gather}
знак в начале определяется начальными условиями, меняется в точке разворота, где знаменатель обращается в нуль, то есть $U(x) = E$. 
\begin{gather}
p(x) = \pdv{L}{\dot{x}} = m(x) \dot{x} = \pm \sqrt{2m (E - U)}\\
\pdv{p}{E} = \sqrt{\frac{m}{2}}\frac{1}{\sqrt{E- U}} \Rightarrow t = \pm \pdv{E} \int^x p (x, E) \dd{x},
\end{gather}
а интеграл --- площадь под графиком $p(x)$.
\begin{dfn}
$U(x) < E $ --- область возможного движения, иначе мы переходим в комплексное время, что запрещено в классической механике.
\end{dfn}
\begin{dfn}
$U(x) = E$ --- точки разворота.
\end{dfn}

Фазовый портрет обладает зеркальной симметрией относительно оси $x$.
$[x_3, \infty]$ --- инфинитное движение.

\[\dot{x} \sim \sqrt{E - U(x)} \sim \pm \sqrt{x - x_1}\]
$\dot{x} \sim \pm \sqrt{(x - x_0)^2} = \pm \abs{x-x_0}$ --- в состоянии равновесия с асимптотами $\left(\pdv{U(x_0)}{x} =0\right)$.

\subsubsection{Диссипативные одномерные системы}
В общем случае решений в квадратурах для диссипативных систем нет. $\ddot{x} = F(t, x, \dot{x})$
\begin{equation}
L = \frac{1}{2} m(x) \dot{x}^2 - U(x) \Rightarrow m\ddot{x} + \frac{1}{2} \pdv{m}{x} \dot{x}^2 = -  \pdv{U}{x} \Rightarrow \begin{cases}
\ddot{x} = f(x) - \mu (x) \dot{x}^2\\
f(x) = - \frac{1}{m(x)} \pdv{U}{x} \Rightarrow U = - \int m f \dd{x}\\
\mu{x} = \frac{1}{2m} \pdv{m}{x} \Rightarrow \mu = \pdv{x} (\ln \sqrt{m}) \Rightarrow m = \left(\mu \dd{x}\right)^2
\end{cases}
\end{equation}
\documentclass[12pt, a4paper]{article}
\usepackage[utf8]{inputenc}
 \usepackage[T1, T2A]{fontenc}
\usepackage[english, russian]{babel}
\usepackage{caption}
\usepackage{indentfirst}
\usepackage{graphicx, xcolor}
\usepackage{cmap}

\usepackage[unicode, pdftex]{hyperref}
\hypersetup{linkcolor=blue, urlcolor=blue, colorlinks=true}
\usepackage{hyphenat}
\hyphenation{объект}
\usepackage{wrapfig}
\usepackage[left=1.4cm,right=1.4cm,top=1.5cm,bottom=1.5cm,bindingoffset=0cm]{geometry}
\usepackage{tocloft}    
\usepackage{titlesec} \titlelabel{\thetitle.\quad} 
\frenchspacing
\makeatletter
\renewcommand{\@biblabel}[1]{#1.} % Заменяем библиографию с квадратных скобок на точку:
\makeatother
\makeindex

\usepackage[]{mathtools}
\renewcommand{\theequation}{\thesection.\arabic{equation}}
    
\parindent=1.25cm
%\parskip=0.1cm

\usepackage{physics} 
\usepackage{siunitx} % typesets numbers with units very nicely
\usepackage{amssymb,amsfonts,amsmath,mathtext,cite,enumerate,float}
\DeclareMathOperator{\sign}{sgn}
\usepackage{amsthm}

%\usepackage[dvips]{graphicx}

\usepackage{multicol}

\bibliographystyle{unsrt}


\begin{document}
\renewcommand{\cftsecaftersnum}{.}
\renewcommand{\cftsubsecaftersnum}{.}

\renewcommand\refname{Список литературы}

\theoremstyle{plain}
\newtheorem{thm}{Теорема}[section]
\newtheorem{lem}[thm]{Лемма}
\newtheorem{pst}{Постулат}[section]

\theoremstyle{definition}
\newtheorem{dfn}{Определение}[section]
\newtheorem{cns}[thm]{Следствие}

\theoremstyle{remark}
\newtheorem{task}{Задача}[section]
\newtheorem{ex}{Пример}[subsection]
\newtheorem{cex}[ex]{Контрпример}
\newtheorem{rmk}{Замечание}[subsection]

\newcommand*{\eqdef}{\stackrel{\mathrm{def}}{=}}
\newcommand*{\is}[1]{\stackrel{\mathrm{\eqref{#1}}}{=}}
\newcommand*{\eqq}[1]{\stackrel{\mathrm{#1}}{=}}
\newcommand*{\hlf}{\frac{1}{2}}

\columnseprule = 0.4pt

\mathtoolsset{showonlyrefs=true}
\mathtoolsset{showmanualtags=true}

\begin{center}
\Huge{\textbf{Теоретическая механика}}
\end{center}
\tableofcontents
\newpage
\input{section1.tex}
\section{Механика Лагранжа}
Стартуем с механики Ньютона:
\[m_i \Ddot{\vec{r}}_i = \Vec{F}_i (\vec{r}_1, \ldots, \vec{r}_N, \vec{v}_1, \ldots, \vec{r}_N, t),\; i=\overline{1,N}, \] но беда в том, что для ряда сил мы знаем результат их действия, а не сами силы.
\begin{dfn}
Связи \index{Связи!} --- не вытекающие из уравнения движения ограничения на положения точек $\lbrace\vec{r}_i, \vec{v}_i\rbrace$.
\[m_i \Ddot{\vec{r}}_i = \vec{F}_i^{(a)} + \vec{R}_i \] Так, выше указана несвободная система \index{Несвободная система} --- на неё наложены связи. $\vec{F}_i^{(a)}$ --- активные силы (их знаем), $\vec{R}_i$ --- силы реакции (знаем связи)\footnote{Силы, с которыми  тела, осуществляющие связи, действуют на точки системы называются реакциями связей.}.
\end{dfn}
\subsection{Связи и их классификация.}
Различают голономные и неголономные, удерживающие и неудерживающие, стационарные и нестационарные связи.
\begin{dfn}
Голономными (или интегрируемыми) связями называют связи, уравнения которых всегда можно свести к уравнениям вида
\begin{equation}
f(\vb{r}_1, \ldots, \vb{r}_N, t) = 0,
\end{equation}
где $f$ является функцией только координат точек и времени. Эти связи накладывают ограничения не только на положение, но и на скорости и ускорения точек системы.
\end{dfn}
\begin{dfn}
Неголономными  (неинтегрируемыми) связями называют связи, уравнения которых нельзя свести к уравнениям, содержащим только координаты точек и время. Неголономной, например, является связь, налагаемая на шар, катящийся по шероховатой поверхности.
\end{dfn}

\[f(t, \lbrace\vec{r}_i\rbrace, \lbrace\vec{v}_i \rbrace) = 0,\] связи в виде равенств --- удерживающие связи. \index{Связи! удерживающие}
\[f(\dots) \geqslant 0 \text{--- неудерживающиие, ненапряжённые связи.}\]  Неудерживающие связи впервые появились только в теории атомного ядра; математически их можно представить в виде удерживающей связи, подставив в степ-функцию.
...\footnote{см. \cite{OlhTM}, c. 204, \cite{GantnakherTM}, с. 201.}
\begin{gather}
f(t, \lbrace\vec{r}_i\rbrace) = 0 \Rightarrow \notag\\
\pdv{f}{t} + \sum_{i=1}^N \pdv{f}{\vec{r}_i} \vec{v}_i = 0. \label{golonom}
\end{gather}
Конечные, дифференцируемые, недифференцируемые, интегрируемые, голономные \eqref{golonom}.
Стационарные (склерономные)($\not t$), нестационарные (реаномные)($t$) связи.
\begin{ex}

\end{ex}

\subsection{Основная задача механики. Идеальные связи}
\subsection{Идеальные связи и уравнения Лагранжа первого рода}
Есть $\{\vb{r}_i\}$. Договорились, что 
\begin{equation}
m_i \ddot{\vb{r}}_i = \vb{F}_i + \vb{R}, \quad i = \overline{1, N},
\end{equation}
то есть поделили на активные силы и силы реакции связей, но плохо то, что знаем не все силы правой части. Из-за $R_i$-ых возникают $3N$ новых величин, связи дают лишь $K$ величин, и мы хотим выяснить, когда у нас задача согласована.
Вспомним про голономные связи, которые могут быть представлены в виде \begin{equation}
f_j (\{\vb{r}_i\}, t) = 0, \quad j = \overline{1, K}.
\end{equation} 
\begin{dfn}
Возможное перемещение --- произвольное бесконечно малое перемещение точек системы, которое согласовано со связями. Формально
\begin{gather}
\left. \sum_i \pdv{f_j}{\vb{r}_i}\vb{v}_i + \pdv{f_j}{t} \equiv 0 \right| \cdot \dd{t},\\
\intertext{то есть умножаем на $\dd{t}$ интегрируемую связь, тогда}
\sum_i \pdv{f_j}{\vb{r}_i} \dd{\vb{r}_i} + \pdv{f_j}{t} \equiv 0,
\end{gather}
и решение этой системы $K$ уравнений является совокупностью всех возможных перемещений.
\end{dfn}

\begin{dfn}
Действительное перемещение --- бесконечно малое перемещение, совместимое со связями и уравнениями движения (и оно единственно как решение задачи Коши).
\end{dfn}

Если <<заморозить>> время, то есть <<забыть>> про частную производную $\pdv{f}{t}$,  то 
\[\left. \sum_i \pdv{f_j}{\vb{r}_i} \vb{v}_i = 0 \right| \cdot \dd{t} \quad \Rightarrow \sum_i \pdv{f_j}{\vb{r}_i} \var \vb{r}_i = 0 \footnote{$\var \vb{r}_i = \vb{v}_i \dd{t}$ --- дифференциал при замороженном времени.}\] 

\begin{dfn}
Виртуальное перемещение --- бесконечно малое перемещение, совместимое со связями при замороженном времени.
\end{dfn}

Виртуальному перемещению можем сопоставить виртуальную работу:
\[\{\var \vb{r}_i\} \longrightarrow \var A = \sum_i \vb{F}_i \var \vb{r}_i = \sum_i \vb{F}_i^{(a)} \var \vb{r}_i + \sum_i \vb{R}_i \var \vb{r}_i.\]
И оказывается, что почти в любом идеализированном механизме без трения $\var A_R = 0$.

\begin{dfn}
Связь называется идеальной, если $\var A_R = 0 \quad \forall \{\var \vb{r}_i\},$ то есть если виртуальная работа сил реакции связей равна нулю при любом виртуальном перемещении.
\end{dfn}
\noindent Примером идеальной связи может служить движение по гладкой неподвижной поверхности.


\textit{Эмпирическое утверждение.} Почти все связи в механике являются идеальными. Но трение  (попытка учесть немеханическое явление в механике) разрушает идеальность, и мы не знаем, как устроены связи, то есть мы обычно идеализируем задачи и почти всегда угадываем, но при этом сядем в лужу, если уйдём в очень большие масштабы --- в космологию, или в очень малые...

\begin{thm}
Пусть дана идеальная связь:
\begin{equation}
\begin{cases}
\sum_i \vb{R}_i \var \vb{r}_i = 0,\\
\sum_i \pdv{f_j}{\vb{r}_i} \var \vb{r}_i = 0 
\end{cases} \Longleftrightarrow
\exists\; \lambda_j (t): \; \vb{R}_i = \sum_{j=1}^K \lambda_j \pdv{f_j}{\vb{r}_i},
\end{equation}
то есть решили основную задачу механики.
\end{thm}
Перед доказательством применим сформулированную теорему, рассмотрим следствие из неё. 
\begin{cns} Посмотрим, как будут записываться уравнения движения.
\begin{equation}
\begin{cases}
m_i \ddot{r}_i = \vb{F}_i^{(a)} + \sum_{j=1}^K \lambda_j \pdv{f_j}{\vb{r}_i}\\
f_j (t, \{\vb{r}_i\}) = 0,
\end{cases}
\end{equation}
и эта задача уже математически корректна: в ней число переменных соответствует числу уравнений: $3N+K$ неизвестных, $3N+K$ соотношений. Уравнения движения в такой форме для несвободной системы называют \textit{уравнениями Лагранжа первого рода}. \index{Уравнения Лагранжа! первого рода}

И как эти уравнения можно решать?
Метод Лагранжа.
\begin{enumerate}
\item $\forall \lambda(t) \Rightarrow \vb{r}(t, \lambda)$ из уравнений движения.
\item $f(\{\vb{r}(t, \lambda\}, t) \equiv 0 \Rightarrow \lambda.$
\end{enumerate}
\end{cns}
\begin{rmk}
$\pdv{f}{t} = 0 \Rightarrow \pdv{\lambda}{f} = 0$
\end{rmk}

\newpage
\subsection{Линейные колебания в лагранжевых системах}
Начнём потихонечку искать решения уравнений Лагранжа.
\subsubsection{Одномерное движение}
Пусть нам дана одномерная (s =1) натуральная лагранжева система. Будем временно использовать букву $x $ вместо $q$. Имеем полином не выше второй степени (в силу натуральности системы):
\[L(t, x, \dot{x})  = \frac{1}{2} \alpha(x) \dot{x}^2 + \beta(x) \dot{x} - U(x),\]
уже есть некоторая нестыковка, потому что $\alpha, \beta$ могут зависеть от времени --- сделаем упрощения.
\begin{enumerate}
\item $\pdv{L}{t} = 0.$
\item Линейный по скорости член --- гироскопическая сила, но в одномерном случае никаких гироскопических сил не существует, поэтому можем этот член выкинуть, поскольку всегда найдётся $\beta,$ т. ч.
\[\beta \dot{x} = \dv{t} \int \beta \dd{x}.\]
\begin{dfn}[Состояние равновесия]
Состояние равновесия --- решение уравнения движения --- тождественная константа.
$x = x_0 = const\; (\dot{x} = \ddot{x} = \dots = 0).$
\end{dfn}
\item Диссипативных сил нет, то есть $\dv{t} \pdv{L}{\dot{x}} = \pdv{L}{x}$.
\end{enumerate}
Тогда 
\[\pdv{L}{\dot{x}} = \alpha(x) \dot{x} \Rightarrow \alpha \ddot{x} + \alpha' \dot{x}^2 = \pdv{L}{x} = \frac{1}{2} \alpha' \dot{x}^2 - U' \Rightarrow \]
\begin{gather}
 \alpha(x) \dot{x} + \hlf \pdv{\alpha}{x} \dot{x}^2 = - \pdv{U}{x} \Rightarrow \pdv{U(x_0)}{x} =0.\\ \intertext{Используем ещё существования стационарного решения в точке $x_0$:}
 \pdv{U(x_0)}{x} = 0\\
 \intertext{Опишем формально двжиение в окрестности точки $x_0$:}
x = x_0 + q \Rightarrow \alpha(x) = \alpha(x_0) + \dots; \quad \alpha(x_0) = m\\
U(x) = U(x_0) + U'(x_0)q + \hlf U'' (x_0)q^2 + \approx \hlf k q^2; \quad k = U'' (x_0) \Rightarrow\\
 L = \hlf m \dot{q}^2 - \hlf kq^2\\
 \intertext{Возникает вопрос, что значит <<мало>>, когда говорим о малости отклонения от положения равновесия. Поэкспериментировав, можно проверить, что неважно, где делать разложение: в функции Лагранжа или в уравнениях движения. Если будем учитывать следующие поправки, то у нас буду появляться следующие поправки к силе, которая уже учтена, и они по сравнению с ней должны быть малы. Второе замечание: коэфффициенты $k$ и $q$ должны быть невырожденными, иначе должны учитывать следующие члены в разложении, но тогда колебания уже будут нелинейными. Перепишем лагранжиан в эквивалентной форме:}
L = \hlf m \dot{q}^2 - \hlf kq^2 = \frac{m}{2} (\dot{q}^2 - \omega^2 q^2), \quad \omega^2 = k/m,\\
\intertext{соответствующее уранение движения}
\ddot{q} + \omega^2 q =0\\
\intertext{ линейное ОДУ с постоянными коэффициентами, порождаемое квадратичной формой. Есть стандартный способ решения таких уравнений:}
q = Ce^{\lambda t} \Rightarrow \lambda^2 C e^{\lambda t} + \omega^2 C e^{\lambda  t} = 0\\ \Rightarrow \lambda^2 + \omega^2 = 0 \Leftrightarrow \lambda = \pm i \omega \Rightarrow\\
q = C_1 e^{i \omega t} + C_2 e^{-i \omega t}\\  
\intertext{потребуем, чтобы}
q \in \mathbb{R} \Rightarrow C_2 = C_1^* \Leftrightarrow q = C_1 e^{i \omega t} + \text{к. с.} = 2\Re C_1 e^{i\omega t} = \Re C e^{i\omega t}, \quad C \in \mathbb{C}\\
C = c e^{i \varphi} \text{ --- комплексная амплитуда} \Rightarrow q = c \cos(\omega t + \varphi)
\end{gather}
\begin{thm}
Пусть $\Hat{D}$ --- дифференциальный оператор, имеем
\[\Hat{D} q = 0, \quad \Hat{D} \in \mathbb{R},\]
тогда мы всегда можем рассмотреть некое решение $X \in \mathbb{C},\; \Hat{D} X = 0$, автоматически
\begin{equation}
\begin{cases}
\Hat{D} (\Re X) = 0\\
\Hat{D} (\Im X) = 0
\end{cases}
\end{equation}
\end{thm}
\begin{proof}
\begin{align}
\Hat{D}& (\Re X + i \Im X) = 0 \Rightarrow\\
\Hat{D}& (\Re X) + i \Hat{D} (\Im X) = 0
\end{align}
\end{proof}
\begin{gather}
q= Ce^{\lambda t}\\
\ddot{q} + \omega^2 q = Ce^{\lambda t} (\lambda^2 + \omega^2) = 0 \Rightarrow q = Ce^{i \omega t} \Rightarrow q = \Re C e^{i \omega t}\\
\omega^2 > 0\quad q = c \cos (\omega t + \varphi)  = a \sin \omega t + b \cos \omega t,\\
\intertext{ через комплексные амплитуды:}
C = c e^{i  \varphi}.
\intertext{Квадрат $\omega$ больше нуля, когда $k$ больше нуля, потому что $m$ мы получаем из законов Ньютона и оно больше нуля, а вот $k$ <<выползает>> из потенциала, и может быть меньше нуля. Положительный квадрат частоты отвечает минимуму потенциальной энергии. }
\omega^2 < 0 \quad \lambda = \pm \sqrt{\abs{\omega}} \in \mathbb{R}\\
q = c_1 e^{\lambda t} + c_2 e^{- \lambda t} = a \sh \lambda t + b \ch \lambda t \underset{H/w}{\eqq{?}} c \sh(\lambda t + \varphi) \underset{H/w}{\eqq{?}} \tilde{c} \ch (\lambda t + \varphi)\\
\omega = 0 \quad q = c_1 t + c_2 \quad \ddot{q} = 0
\end{gather}
Продемонстрируем всю мощь метода комплексных амплитуд. В этих нескольких примерах будем выходить за рамки консервативных ($L = \frac{m}{2} (\dot{q}^2 - \omega^2 q^2),\; H = \frac{m}{2}(\dot{q} + \omega^2 q^2) = const$) систем. 
\begin{ex}[Осциллятор с трением.]
\begin{gather}
\ddot{q} + \omega_0 q + 2\gamma \dot{q} = 0,\\ \intertext{то есть рассматриваем линейный осциллятор с трением.}
q = C e^{\lambda t} \Rightarrow (\underbrace{\lambda^2 + \omega_0^2 + 2\gamma \lambda}_{0}) C e^{\lambda t} = 0\\
\lambda = -\gamma \pm \sqrt{\gamma^2 - \omega_0^2} = -\gamma \pm i\sqrt{\omega_0^2 - \gamma^2},
\intertext{в комплексном виде записали, чтобы был переход к незатухающему осциллятору.}
q = c e^{-\lambda t} \cos(\sqrt{\omega_0^2 - \gamma^2}t + \varphi),\\
\intertext{получили общее решение. $H/w\; \omega_0 = \gamma,\; \omega_0 < \gamma,\; \omega_0 > \gamma\, \text{(движение в меду).}$}
\end{gather}
\end{ex}
\begin{ex}[Осциллятор, на который действует внешняя сила.]
\begin{gather}
\ddot{q} + \omega^2 q = f(t)\\
\dot{q} + i\omega t = a(t) e^{i \omega t},\label{pl_h_1}\\
 a(t) \in \mathbb{C} \Rightarrow q(t) = \frac{1}{\omega} \Im a e^{i \omega t}\\
\begin{cases}
\dv{t}\left(\dot{q} + i\omega t \right) = (\dot{a} +  i \omega a)e^{i \omega t}\\
\eqref{pl_h_1}* i\omega: \quad - i\omega \dot{q} + \omega^2 q = -i \omega a e^{i \omega t}
\end{cases}
\Rightarrow \ddot{q} + \omega^2 q = \dot{a} e^{i \omega t} = f(t) \Rightarrow a(t) = \int^t f(t) e^{-i \omega t} \dd{t}\\
\intertext{Обратим внимание, что $a(t)$ очень похоже на преобразование Фурье.}
\end{gather}
\end{ex}

\begin{thm}[Появляющаяся сила.]
Пусть в некоторый момент на осцилллятор подействовала сила с конечным спектром (см. рисунок) $\int\limits_{-\infty}^{+\infty} f e^{-i \omega t} \dd{t} = F,$ тогда $H(+\infty) - H(-\infty) = \frac{m}{2} \abs{F}^2,$\\ где~$F$~--- спектральная компонента силы.
\end{thm}
\begin{proof}
H/w
\end{proof}
\begin{task}
\begin{gather}
\ddot{q} + 2\gamma \dot{q} + \omega_0^2 q = f(t)
\end{gather}
В частности, когда сила сама осциллирует: $f (t) = A \cos \omega t \Rightarrow q(t) = ?$.
\end{task}

\subsubsection{Многомерные системы}
\begin{gather}
L = \sum_{i, j} \hlf \alpha_{ij}(x) \dot{x}_i \dot{x}_j + \sum_i \beta_i(x)\dot{x}_i - U(x)
\end{gather}
Построим уравнение движения. Скажем, что $x$ --- тождественная константа --- отвечает случаю локального экстремума функции $U(x)$:
\[x = x_0 \equiv const \Leftrightarrow \pdv{U(x_0)}{x_i} = 0, \quad \forall i =\overline{1,s}\]
\begin{align}
x = x_0 + q \Rightarrow\; & \alpha_{ij}(x) \approx m_{ij} = \alpha_{ij} (x_0) \\
&U(x) \approx \sum_{i, j} \hlf k_{ij} q_i q_j; \quad k_{ij} = \pdv{U(x_0)}{x_i}{x_j}
\end{align}
Что можем сказать про коэффициенты $m_{ij}, k_{ij}$?
\begin{enumerate}
\item $m_{ij}$ симметричная ($m_{ij} = m{ji}$) и положительно определённая, $\alpha$ --- положительно определённая квадратичная форма, потому что произошла из кинетической энергии, и матрица постоянных коэффициентов $m$ унаследовала эти свойства.
\item $k_{ij} = k_{ji}:$ появилась по определению как смешанная производная, положительная определённость не гарантируется (достигается в случае локального минимума потенциала). 
\end{enumerate}
Осталось рассмотреть гироскопические силы. К полной производной, как в одномерном случае они сводиться не обязаны. Заметим, что 
\begin{gather}
\sum_i \beta_i(x_0) \dot{x}_i = \dv{t} \left(\sum \beta_i (x_0) x_i\right),\; \text{поэтому}\\
\beta_i(x) \approx \not{\beta_i(x_0)} + \sum_j \pdv{\beta_i(x_0)}{x_j} q_j; \quad g_{ij} = \pdv{\beta_i(x_0)}{x_j},
\end{gather}
подставим это всё в лагранжиан:
\begin{equation}
\boxed{L = \sum_{i, j = 1}^s \left\lbrace \hlf m_{ij} \dot{q}_i \dot{q}_j + g_{ij} q_j \dot{q}_i - \hlf k_{ij} q_i q_j \right\rbrace}. \label{Lagr_osc}
\end{equation}
\begin{ex}[$g_{ij} = 0$]
Рассмотрим лагранжиан \eqref{Lagr_osc} в частном случае, когда нет гиротропных сил, то есть $g_{ij} = 0$:
\begin{equation}
L = \sum_{i, j = 1} \left\lbrace \hlf m_{ij} \dot{q}_i \dot{q}_j  - \hlf k_{ij} q_i q_j \right\rbrace.
\end{equation}
Его можно рассматривать как две квадратичные формы, соответствующие двум слагаемым: первая симметричная и положительно определённая, вторая симметричная. Теорема из линейной алгебры утверждает, что такие кв. формы диагонализируемы одновременно.
\begin{thm}
$$
\left\{
\begin{array}{rcl}
\text{симм.$(+)$}\\
\text{симм.}
\end{array}
\right.
\Rightarrow \text{всегда диагонализуемы одновременно!}
$$
То есть $\exists\; \text{линейное преобразование}\; a_{ik}\; |\; q_i = \sum_i a_{ik} \theta_k, \dot{q}_i = \sum a_{ik} \dot{\theta}_k.$
\end{thm}
 тогда
\begin{equation}
L = \sum^s_k \lbrace \hlf m_k \dot{\theta}_k^2 - \hlf k_k \theta_k^2 \rbrace = \sum_k \frac{m_k}{2} \lbrace \dot{\theta}_k^2 - \omega^2_k \theta_k^2 \rbrace, \label{Lagr_ex_osc}
\end{equation}
где $\omega_k^2 = k_k / m_k,$ то есть система распадается на $s$ штук невзаимодействующих подсистем, каждая из которых есть  одномерный гармонический осциллятор.
Уравнение движения соответствующее \eqref{Lagr_ex_osc}:
\begin{gather}
\ddot{\theta}_k + \omega^2_k \theta_k = 0 \Rightarrow\\
\theta(t) = C_k \cos (\omega t + \varphi_k)  \approx \Re C_k e^{i \omega_ kt}
\end{gather}
\begin{dfn}
$\{\omega_k\}$ --- спектр нормальных частот $\omega_k,\; k = \overline{1, s}.$
\end{dfn}
\begin{dfn}
$\{\theta_k\}$ --- нормальные координаты.
\end{dfn}
Как выглядит решение? 
\begin{dfn}
Частное решение при $\theta_k = 0,$ кроме $k = k^* \Rightarrow$
\begin{equation}
q_j = a_{jk^*} \theta_{k^*}(t) 
\end{equation}
называют нормальными колебаниями.
\end{dfn}
Общее решение:
\begin{equation}
q_j (t) = \sum_{k=1}^s a_{jk} \theta_k (t).
\end{equation}
\begin{rmk}
Если мы возбудили одно нормальное колебание, то каждая степень свободы колеблется в одной и той же фазе. То есть, если у нас есть сложная многомерная система, и одна степень свободы проходит через ноль или экстремум, то остальные степени свободы тоже проходят через ноль или экстремум соответственно.
\end{rmk}
\begin{rmk}
Если мы живём на дне потенциального рельефа, то в \eqref{Lagr_ex_osc} две положительно определённые квадратичные формы, значит, $\omega_k^2 > 0\; \forall k,$ и у нас действительно колебания, то есть можем получить решения в виде синусов и косинусов, а не только экспонент.
\end{rmk}
\end{ex}
Построим уравнение движения для лагранжиана \eqref{Lagr_ex_osc}:
\begin{gather}
\pdv{L}{\dot{q}_k} =  \{ \frac{1}{2} m_{ik} \dot{q}_i + \frac{1}{2} m_{kj} \dot{q}_j + g_{kj}q_j \} = \sum_i \{ m_{ik} \dot{q}_i + g_{ki} q_i\}, \label{pdv_L_dotq_osc}\\
\pdv{L}{q_k} = \sum \{ g_{ik} \dot{q}_i - k_{ik} q_i \} \stackrel{!!!!!}{\Rightarrow} \label{pdv_L_qk}\\
\dv{t}\pdv{L}{\dot{q}_j} - \pdv{L}{q_j} = 0 \Rightarrow \sum_i \{ m_{ij} \ddot{q}_i + (g_{ij} - g_{ji})\dot{q}_i + k_{ij} q_i \} = 0.
\end{gather}
\begin{rmk}
$G_{ij} = -G_{ji}.$
\end{rmk}
Ищем решение в виде
\begin{equation}
q_i = \Re C_i e^{\lambda t}, 
\end{equation}
тогда
\begin{gather}
\Re \sum_i \{m_{ij} \lambda^2 + G_{ij} \lambda + k_{ij} \} C_i e^{\lambda t} = 0 \Leftrightarrow\\
\sum_i \{m_{ij} \lambda^2 + G_{ij} \lambda + k_{ij} \} C_i = 0. \label{eq_lambda}\\
\intertext{Эта линейная однородная алгебраическая система имеет невырожденное решение, когда детерминант матрицы её коэффициентов равен нулю:}
\det \left( m_{ij} \lambda^2 + G_{ij} \lambda + k_{ij} \right) = 0\; \text{--- характеристическое уравнение.}
\end{gather}
Поразмышляем о структуре решения. По размерности $P_{2s} (\lambda) =0,$ плюс, если $\lambda$ --- корень, то $\lambda^*$~--- тоже корень, так как $P_{2s} \in \mathbb{R},$ то есть все коэффициенты этого полинома действительные. На самом деле, уравнение движения консервативной системы накладывает ещё одно ограничение, и если расписать детерминант, то  можно получить, что решение имеет вид
$P_s (\lambda^2) = 0.$ Свойство чётности степеней --- свойство обратимости времени. Покажем, что решения действительно идут парами. Пусть $\lambda$ --- корень исходного характеристического уравнения, рассмотрим это же уравнение относительно $-\lambda$:
\begin{gather}
\det \left(m_{ij} (-\lambda)^2 +  G_{ij}(- \lambda) + k_{ij} \right) \Leftrightarrow\\
\det \left(m_{ij} \lambda^2 +  G_{ji} \lambda + k_{ij} \right) \Leftrightarrow\\
\det \left(m_{ji} \lambda^2 +  G_{ji} \lambda + k_{ji} \right) \label{new_det},\\
\intertext{поскольку $m_{ji}$ и $k_{ji}$ симметричные, и мы получили уравнение, выполняющееся тождественно, потому что $\lambda$ --- корень --- свойство антисимметричности члена, отвечающшего за гиротропию. А это означает, что $-\lambda$ --- тоже корень, что в точности и означает, что характеристическое уравнение имеет вид}
P_s(\lambda^2) =0.
\end{gather}
Для консервативной системы каждый корень порождает ещё три:
\begin{equation}
\lambda \longrightarrow \lambda^*, -\lambda, -\lambda^*.
\end{equation}

Поразмышляем, при каких условиях реализуются устойчивые колебания, а не какие-то экспоненты, описывающие неустойчивые состояния равновесия. Вернёмся к \eqref{eq_lambda}. Для анализа таких уравнений существует стандартный приём: умножим каждое уравнение на $C_j^*$, учтём, что
\begin{gather}
C_i C_j^* = (c_i' + i c_i'')(c_j' - i c_j'') = (\underbrace{c_i' c_j' + c_i''c_j''}_{S_{ij}}) + i(\underbrace{c_i'' c_j' - c_i' c_j''}_{A_{ij}}),\\
\intertext{•:}
\sum_{i, j} \boxed{m_{ij} S_{ij} \lambda^2 + iG_{ij} A_{ij} \lambda + k_{ij} S_{ij} = 0}
\end{gather}
\begin{enumerate}
\item гиротропии нет $k_{ij} (+); G_{iJ} = 0 \Rightarrow \lambda^2 = - \frac{k_{ij} S_{ij}}{m_{ij} S_{iJ}} < 0 \Rightarrow \lambda = \pm i\omega$ --- ситуация, когда существуют нормальные частоты и нормальные колебания в смысле именно колебаний.
\item Нет $(+) k_{ij} \Rightarrow \lambda^2 > 0 \Rightarrow \pm \lambda \Rightarrow c_1 e^{\lambda t} + c_2 e^{-\lambda t}$ --- состояние равновесия типа седло.
\item Можно показать, что гиротропия не может разрушить устойчивое состояние равновесия. $k_{ij}(+) \& G_{iо} \neq 0 \Rightarrow \lambda^2 < 0,$ то есть колебания устойчивые.
\end{enumerate}

Допустим, что мы научились решать характеристическое уравнение. Получим общее решение уравнения движения лагранжевой системы вблизи положения равновесия.
\begin{align}
P_s (\lambda^2) = 0 \Rightarrow \{&\lambda^2_k\} k=\overline{1, s}\\
&\lambda^2_k < 0 \Rightarrow \lambda_k = \pm i\omega_k\\
q_j^{(k)} &= \Re C_{jk} e^{i\omega_k t}\\
\end{align}
\begin{align}
\sum_{i=1}^s \left( m_{ij} \lambda^2 + G_{ij} \lambda + k_{ij} \right) C_i = 0 \quad j=\overline{1, s} \Rightarrow C_{jk} = a_{jk} e^{i \varphi_{jk}} &B_k\\
&\forall B_k = b_k e^{i\varphi_{0_k}}
\end{align}
Строим общее решение для $q,$ которое есть сумма всех нормальных колебаний:
\begin{align}
q_ j= \sum_{k=1}^s q_j^{(k)} &= \sum_k \Re b_k a_{jk} e^{i\omega_k t + i\varphi_{jk} + i \varphi_{0_k}}\\
\intertext{$\varphi_{jk} = 0,$ если $G = 0$ (нет гиротропии), тогда}
q_j &= \sum_k a_{jk} \underbrace{\Re B_k e^{i \omega_k t + i \varphi_{0_k}}}_{\theta_k (t)},\\
\intertext{то есть свели ответ к предыдущему.}
\end{align}
Вообще,
\begin{equation}
q_j = \sum_k \Re \left\{B_k C_{jk} e^{i\omega_k t} \right\}.
\end{equation}
Рассмотрим пару простых примеров.
\begin{ex}[Чашечка]
Пусть у нас есть движение в поле тяжести в окрестности минимума какой-то ямки $z = h(x, y).$ Заметим, что если $x =y = 0$ отвечают $\min h(x, y),$ то 
\begin{gather}
z = h(x, y) = \frac{x^2}{2\rho_1^2} + \frac{y^2}{2\rho_2^2} + \dots \quad \rho_{1, 2}\;\text{--- главные радиусы кривизны.}\\
\intertext{Составим лагранжиан:}
L = T - U = \frac{m}{2} \left(\dot{x}^2 + \dot{y}^2 + \dot{z}^2\right) - mg\left( \frac{x^2}{2\rho_1^2} + \frac{y^2}{2\rho_2^2}\right), \label{lagrangian_yamka}\\
\intertext{$\dot{z}^2$ нас не интересует, потому что речь идёт о малых колебаниях, поэтому \eqref{lagrangian_yamka} можно переписать в виде}
L = \frac{m}{2} \left\{\dot{x}^2 - \Omega_1^2 x^2 + \dot{y}^2 - \Omega_2^2 y^2 \right\}, \quad \Omega_{1, 2} = \frac{g}{\rho_{1, 2}^2}.\\
x = a \cos (\Omega_1 t + \varphi_1),\\
y = b \cos (\Omega_2 t + \varphi_2), \text{и эти колебания независимые.}
\end{gather}
\end{ex}
\begin{ex}[Вращающаяся чашечка]
Перейдём в систему координат $x, y$, которая прибита к чашке, и в ней уравнение чашки не изменится, но при этом в подвижной системе координат появятся дополнительные члены, связанные с вращением:
\begin{gather}
\vb{v}_{co} = [\vb*{\Omega}, \vb{r}] \Rightarrow\\
\begin{cases}
v_x = \dot{x} - \Omega y\\
v_y = \dot{y} + \Omega x
\end{cases}
\Rightarrow L = \frac{m}{2} \left\{(\dot{x} - \Omega y)^2 + (\dot{y} + \Omega x)^2 - \Omega_1^2 x^2 - \Omega_2^2 y^2 \right\},
\intertext{этот лагранжиан квадратичен по всем координатам и скоростям, и он содержит гироскопически члены (вида произведение координаты на скорость), а мы его перепишем:}
L = \frac{m}{2} \big\{ \dot{x}^2 + \dot{y}^2 + 2\Omega (x\dot{y} - y\dot{x}) - (\underbrace{\Omega_1^2 - \Omega^2}_{\widetilde{\Omega}_1^2})x^2 - (\underbrace{\Omega_2^2 - \Omega^2}_{\widetilde{\Omega}_2^2})y^2\big\}.\\
\begin{rcases}
\dv{t}\pdv{L}{\dot{x}}= m\ddot{x} - m\Omega\dot{y} = \pdv{L}{x} = m\Omega \dot{y} - m\widetilde{\Omega}_1^2 x\\
\dv{t}\pdv{L}{\dot{y}} = m\ddot{y} + m\Omega\dot{x} = \pdv{L}{y} = - m\Omega \dot{x} - m \widetilde{\Omega}_2^2 y
\end{rcases}
\Rightarrow\\
\begin{cases}
\ddot{x} - 2\Omega \dot{y} + \widetilde{\Omega}_1^2 x = 0\\
\ddot{y} + 2\Omega \dot{x} +  \widetilde{\Omega}_2^2 y =0
\end{cases}\\
x = C_1 e^{i \Omega t}\\
y = C_2 e^{i\Omega t}\\
\begin{cases}
\left(-\omega^{2}+\widetilde{\Omega}_1^2\right) C_{1}-2 \Omega i \omega C_{2}=0\\
2 \Omega i \omega  C_{1}+\left(-\omega^{2} + \widetilde{\Omega}_2^2 \right) C_{2}=0, \label{C1_C2_sys}
\end{cases}
\intertext{система \eqref{C1_C2_sys} имеет нетривиальное решение, когда определитель матрицы коэффициентов перед искомыми $C_1$ и $C_2$ равен нулю, тогда}
\boxed{ \left(\omega^{2}-\widetilde{\Omega}_{1}^{2}\right)\left(\omega^{2}-\widetilde{\Omega}_{2}^{2}\right)=4 \Omega^{2} \omega^{2}}.
\end{gather}
 Получили биквадратное уравнение относительно нормальных частот $\omega$. Проанализируем случай, когда
 \[\omega \ll \widetilde{\Omega}_1, \widetilde{\Omega}_2.\]
\end{ex}

До сих пор у нас был консервативный случай, и обобщённая энергия сохранялась:
\[H = const \quad H = \sum \frac{1}{2} \left\{m_{iJ} \dot{q}_i \dot{q}_j + k_{ij} q_i q_j\right\}.\]
\[H/w \quad \dv{H}{t} = 0 \Leftrightarrow P_s(\lambda^2) = 0\; \text{или}\; \lambda\; \text{и}\; -\lambda\; \text{--- корни одновременно.} \]

\subsubsection{Малые колебания в диссипативных системах}
\paragraph{Диссипативная функция Рэлея.} \index{Диссипативная функция Рэлея}
Линейное трение в лагранжевых системах обычно вводится следующим образом:
\[\vb{F}_i = -\sum_j \mu_{ij} \vb{v}_j,\]
и такие силы можно пересчитать в обобщённые силы, которые войдут в уравнение Лагранжа, с помощью этакого потенциала в пространстве скоростей, с помощью функции Рэлея $R(t, \{\vb{r}_i\}, \{\vb{v}_i\},$ например, такой функции:
\[\vb{F}_i = -\sum_j \mu_{ij} \vb{v}_j = - \pdv{R}{\vb{r}_i} \quad R = \frac{1}{2} \sum_{i, j} \mu_{ij} \vb{v}_i \vb{v}_j = \frac{1}{2} \gamma_{iJ} \dot{q}_i \dot{q}_j,\]
и в этом случае
\[Q_j = \sum_{i=1}^N \vb{F}_i \pdv{\vb{r}_i}{q_j} = - \sum_{i=1}^N \pdv{R}{\vb{v}_i} \pdv{\vb{v}_i}{\dot{q}_j} = - \pdv{R}{\dot{q}_j} = - \sum_{i=1}^n \gamma_{ij} \dot{q}_i,\]
и отличие от гироскопических сил только в том, что матрица коэффициентов здесь симметричная (по построению):
\[\gamma_{ij} = \gamma_{ji}.\]
А уравнение Лагранжа выглядит следующим образом:
\[\boxed{\frac{d}{d t} \frac{\partial L}{\partial \dot{q}_{j}}-\frac{\partial L}{\partial q_{j}}+\frac{\partial R}{\partial \dot{q}_{j}}=0}.\]
\[H/w \Rightarrow \sum_{j=1}^{s}\left\{m_{i j} \dot{q}_{j}+\left(G_{i j}+\gamma_{i j}\right) \dot{q}_{j}k_{ij} q_j\right\} = 0, \]
причём $G_{ij}$ --- антисимметричная часть (порождается функцией Лагранжа), гарантирует,\\ 
что $P_s (\lambda^2) =0;\; \dv{H}{t} =0;$ $\lambda_{ij}$ --- симметричная часть (порождается функцией Рэлея, и $P_{2s}(\lambda) =0$ --- есть нечётные степени, диссипация, и направления времени не эквиваленты, диссипация работает в обе стороны, система не может двигаться <<по кругу>>, $\dv{H}{t} = \sum Q_j \dot{q}_j = - \sum \pdv{R}{\dot{q}_j} \dot{q}_j = -2R,$ то есть физический смысл функции Рэлея в том, что она отвечает мощности потерь на соответствующей силе, которую она определяет, и в этом случае мы будем получать решения, как для осциллятора с трением, в виде
\begin{gather}
e^{\lambda t};\; \lambda = \lambda' + i\lambda'' \Rightarrow\\
e^{\lambda' t} \cos (\lambda'' t + \varphi),
\end{gather}
действительная часть корня характеристического уравнения описывает затухание в случае диссипации, а мнимая часть --- действительную часть частоты, свойство одновременной принадлежности к корням $\lambda$ и $\lambda^*$ сохраняется (потому что это свойство действительности коэффициентов), а $\lambda$ и $-\lambda$ --- нет.
\subsection{Вариационная форма механики Лагранжа}
\subsubsection{Введение в принцип наименьшего действия}
Раньше, получая уравнения Ньютона, мы исходили из принципа малых шажков. Математически это выражалось тем, что мы рассматривали дифференциальные уравнения Ньютона, и получали решения (уравнения движения как бесконечную сумму бесконечно малых шажков). Введение принципа уравнения Лагранжа и понятия идеальных связей позволили нам исключить из уравнений Ньютона силы реакции связей, которые мы не знаем, и мы получили уравнения Лагранжа второго рода. Это был дифференциальный подход.

Вариационная формулировка подразумевает рассмотрение траекторий как некое целое --- "интегральный подход".

\[L(t, q, \dot{q}, q = (q_1, \dots, q_s), s= 3N-k,\]
и нет непотенциальных обобщённых сил
\[Q^{\text{НП}} =0, \]
то есть наша система полностью описывается уравнением
\[\left(\frac{d}{d t} \frac{\partial L}{\partial \dot{q}_{j}} - \frac{\partial L}{\partial q_{j}}=0\right).\]
\begin{dfn}
$\{q\}$ --- конфигурационное пространство.
\end{dfn}
Линия $q(t)$ в конфигурационном пространстве --- то, что мы ищем --- траектория системы, эволюция её состояний во времени. Принцип наименьшего действия позволяет отсортировать настоящие и ненастоящие траектории.  Истинная траектория --- <<прямой путь>>.
\begin{gather}
q_1 = q(t_1),\\
q_2 = q(t_2).
\end{gather}
Как отличить истинную траекторию, которая отвечает уравнения Лагранжа, от всех остальных? В фазовом пространстве траектории не пересекаются нигде, кроме особых точек, пересечения (самопересечения) кривых в конфигурационном пространстве ничему не противоречат.

\subsubsection{Вариационный принцип Гамильтона для обобщенно-потенциальных систем}
Давайте каждой из траекторий по какому-то закону припишем число, а потом скажем, что истинной траектории отвечает конкретное число.

Отображение функций в числа \[q(t) \stackrel{S}{\rightarrow} \mathbb{R}\] называют функционалом.Чтобы не путать с функциями, пишут аргумент в квадратных скобках
\[S[q(t)] \rightarrow \mathbb{R}.\]

Длина кривой не позволяет выделять истинные траектории. Рассмотрим функционал
\begin{equation}
 \boxed{S[q(t)] = \int\limits L(t, q, \dot{q}(t)) \dd{t}}, \label{deystv}
 \end{equation}
 этот функционал называют функционалом \index{Функционал действия} действия (иногда просто действием). Конфигурационное пространство является общим для большого семейства систем с разными лагранжианами, но общими обобщёнными координатами.
 
 \begin{pst}
 Пусть
 \begin{enumerate}
 \item $q(t_1) = q_1, q(t_2) = q_2,$
 \item $\exists M:\; \abs{q_1 - q_2} < M$, 
 \end{enumerate}
 тогда $q(t),$ удовлетворяющее уравнению движения, отвечает наименьшему значению функционала действия $S[q(t)] \rightarrow min$ --- аналог того, что первая производная равна нулю, вторая производная знакоопределена. 
 \end{pst}
 
 \begin{pst}[Вариационный принцип Гамильтона]
 Пусть $q(t_1) = q_1, q(t_2) = q_2,$ тогда между этими положениями система движется так, что ФЛ принимает стационарные значения $S[q(t)] \rightarrow stat$ --- аналог того, что первая производная равна нулю.
 \end{pst}

Давайте рассмотрим малое возмущение (бесконечно близкую траекторию с невозмущённым  концами) $q(t) + \delta q(t),$ где $\delta q(t_1) =0,\; \delta q(t_2) = 0,\; \forall \delta q(t)$.
\begin{gather}
S[q(t)+\delta q(t)]-S[q(t)] \geqslant 0
\end{gather}
Мы не умеем находить экстремумы в пространстве функций, но умеем в пространстве чисел. Предположим, что
\begin{gather}
\delta q(t) = \alpha \cdot h(t),\\
h(t_1) = h(t_2) = 0,
\end{gather}
введём понятие 
\begin{equation}
S(\alpha) = S[q(t) + \alpha \cdot h(t) ] \geqslant,
\end{equation}
последнее неравенство ... то есть переформулировали ПНД в терминах задачи на экстремум в обычных переменных. На физическом уровне строгости напишем необходимое условие существования экстремума в точке $\alpha = 0$
\begin{gather}
S(\alpha) \rightarrow \pdv{S}{\alpha} = 0\; |_{\alpha = 0}.
\end{gather}
Распишем
\begin{gather}
\pdv{\alpha} \int\limits_{t_1}^{t_2} L (t, q + \alpha h, \dot{q} + \alpha \dot{h}) \dd{t} = \int\limits_{t_1}^{t_2} \left\{ \pdv{L}{q} h + \pdv{L}{\dot{q}}\dot{h} \right\}_{|_{\alpha = 0}} \dd{t} = \\
= \int \limits_{t_1}^{t_2} \left\{ \pdv{L}{q} - \dv{t} \left(\pdv{L}{\dot{q}}\right) \right\} h(t) \dd{t} +  \underbrace{\left. \pdv{L}{\dot{q}}\, h \right| _{t_1}^{t_2}}_{=0} = \\
=\int\limits_{t_1}^{t_2} \sum_{j=1}^{s}\left\{\frac{\partial L}{\partial q_{j}}-\dv{t} \left(\frac{\partial L}{\partial \dot{q}_j}\right)\right\} h_{j}(t) d t=0 \quad \text{по ПНД для $\forall h_j (t)$.} \Rightarrow\\
\forall\, j =1,s \quad \boxed{\dv{t} \left(\pdv{L}{\dot{q}_j} \right) - \pdv{L}{q_j}=0},
\end{gather}
то есть мы получили, что принцип наименьшего действия гарантирует, что движение по истинным траекториям удовлетворяет уравнению Лагранжа (или, что то же самое, что необходимое условие экстремума --- в точности то же, что уравнение Лагранжа).

Займёмся тем же, что проделали только что, однако не переходя к дифференцированию в числах,
\begin{equation}
S[q + \delta q] - S[q] = \int \limits_{t_1}^{t_2} \left\{ L(t, q + \delta q, \dot{q} + \delta \dot{q}) - L(t, q, \delta{q}) \right\} \dd{t},
\end{equation}
зафиксируем момент времени и применим разложение в ряд Тейлора, тогда
\begin{gather}
S[q + \delta q] - S[q] = \int\limits_{t_1}^{t_2} \left\{\pdv{L}{q} \delta q + \pdv{L}{\dot{q}} \delta \dot{q} + \ldots \right\} \dd{t} =,\\
\intertext{и проинтегрируем по частям}
= \int\limits_{t_1}^{t_2} \left\{ \pdv{L}{q} - \dv{t} \pdv{L}{\dot{q}} \right\} \delta q \dd{t} + \ldots .
\end{gather}
Написанное нами слагаемое называют \textit{первой вариацией функционала действия,} \index{Вариация} обозначают как $\var{S[q, \delta q]}$.


Согласно ВПГ
\begin{equation}
\var{S} = 0  \Longleftrightarrow \; \text{ур-ю Лагранжа II рода при фиксированных концах траектории.}
\end{equation}


А ПНД говорит, что
\begin{equation}
\text{ур-e Лагранжа II рода} \Longrightarrow \begin{cases}
\var S = 0,\\
\var^2{S} \geqslant 0 
\end{cases}
\text{при фиксированных концах, близости}\; q_1, q_2.
\end{equation}

\begin{rmk}[Обобщение на случай систем, не являющихся обобщённо-потенциальными]
Хотим, чтобы первая вариация приводила к уравнению $\frac{d}{d t} \frac{\partial T}{\partial \dot{q}}-\frac{\partial T}{\partial q}=Q$. Первые два слагаемые получаются из кинетической энергии $T(t, q, \dot{q}),$ второе --- из $\var{S} = \int\limits_{t_1}^{t_2} Q \dd{t},$ при этом \[Q = \sum_{j=1}^s Q_j \delta q_j = \sum_{i=1}^N \vb{F}_i^{(a)} \delta \vb{r}_i = \delta A^{(a)},\] значит,
\begin{equation}
S[q(t)] = \int\limits_{t_1}^{t_2} \left(T + A^{(a)}\right) \dd{t}.
\end{equation}

Частный случай, когда мы имеем дело с обобщённо-потенциальными силами в натуральных системах, в качестве работы активных сил может выступать обобщённый потенциал, то есть
\[S = \int\limits_{t_1}^{t_2} (T- U) \dd{t}.\] \textit{H/w показать, что, варьируя такой функционал, получим то же, что при варьировании такого функционала с функцией с Лагранжа в качестве аргумента.}
\end{rmk}

\subsubsection{Вариационный принцип для гармонического осциллятора}
Почему выбрали именно гармонический осциллятор? Задача на равенство нулю первой производной гораздо проще задачи на определение знака второй производной...

Пусть у нас одномерная система, которая задана лагранжианом
\[L = \frac{1}{2} (\dot{q}^2 - \omega^2 q.\]
Для это этой системы мы всё знаем:
\begin{gather}
\ddot{q} = -\omega^2 q,\\
q = a\sin (\omega t + \varphi).
\end{gather}
Рассмотрим малое возмущение $q+\delta q,$ посчитаем
\begin{equation}
S[q+\delta q] - S[q]. \label{functional_1}
\end{equation}
Будем считать $t_1 = 0,$ $t_2 = \tau,$ и если $\tau \neq n T/2,$ то по $q_1$ и $q_2,$ соответствующим заданным моментам времени мы траекторию однозначно определяем, иначе, при времени $\tau = n T/2$ ($T = \frac{2\pi}{\omega}$), называемом \textit{кинетическим фокусом} \index{Кинетический фокус},  получим бесконечно много решений, и все они будут удовлетворять уравнению движения. 
\begin{dfn}
Кинетический фокус сопряжённый некой начальной точке --- точка, в которую сходятся  две любые бесконечно близкие (пущенные с разными скоростями) траектории.
\end{dfn}

\begin{ex}[Кинетический фокус на сфере]
На глобусе между двумя точками есть наикратчайшее расстояние, но если мы рассмотрим две точки на диаметрально противоположные точки , то таких траекторий с наименьшей длиной будет бесконечно много, и диаметрально противоположная точка будет кинетическим фокусом.
\end{ex}

Нам надо определить
\begin{equation}
\begin{cases}
\delta q(0) =0\\
\delta q(\tau) = 0,
\forall \delta q(t)
\end{cases}
\Rightarrow \delta q(t) = \sum_{n=1}^\infty a_n \sin \left(n \frac{\pi}{\tau} t \right), \; \forall a_n \in \mathbb{R}.
\end{equation}
Вернёмся к \eqref{functional_1}
\begin{gather}
S[q+\delta q] - S[q] = \int\limits_0^\tau \frac{1}{2} \left( (\dot{q} + \delta \dot{q})^2 - \omega^2 (q +\delta q)^2  - \dot{q}^2 + \omega^2 q\right) \dd{t} =\\
=\int\limits_0^\tau \frac{1}{2} \left\lbrace \underbrace{2\dot{q} \delta \dot{q} - 2\omega^2 q \delta q}_{\dot{q}\delta \dot{q} + \ddot{q} \delta q = \dv{t}(\dot{q} \delta q)} + \delta \dot{q}^2 - \omega^2 \delta q^2 \right\rbrace \dd{t} =\\
= \int\limits_0^\tau \frac{1}{2} \sum_{n=1}^\infty a_n^2  \left\lbrace (n \frac{\pi}{\tau})^2 \cos^2 (n \frac{\pi}{\tau} t) - \omega^2 \sin^2 (\frac{n \pi}{\tau} t) \right\rbrace \dd{t} =\\
= \frac{\pi}{4} \sum_{n=1}^\infty a_n^2 (n^2 \Omega^2 - \omega^2), 
 \end{gather}
 и вся эта штука отвечает минимуму, если $\Omega > \omega,$ что равносильно тому, что $\tau < T/2.$ То есть, если траектория не содержит кинетического фокуса, то реализуется принцип наименьшего действия\footnote{Подробнее, но без доказательств, см. \cite{Atherman}, \S 5, Гл. 7.}.

Повторим общее утверждение. Если на действительной траектории не лежит кинетический фокус исходной точки, то функционал действия отвечает наибольшему (или наименьшему\footnote{Зависит от того, какой знак мы поставили перед лагранжианом.}) значению.

Если мы рассматриваем траекторию с кинетическим фокусом, то её можно разбить на кусочки без кинетических фокусов, и на каждом куске будет реализован минимум.

Если проводить вариацию только по траекториям, проходящим через кинетический фокус, то будет принцип наименьшего действия.

\subsubsection{Следствия вариационного принципа}
Чем же так хорош вариационный принцип?
У вариационного принципа есть два аспекта, которые сделали его в некоторой мере центральным для современной теоретической физики.

Сформулировав вариационный принцип, мы сумели абстрагироваться от системы координат --- технической вспомогательной конструкции, которая нужна уже при решении уравнений движения. Неизменность уравнений Лагранжа при изменении координат --- следствие вариационного принципа.

\begin{enumerate}
\item Пусть у нас есть $L(t, q, \dot{q})$, рассмотрим $L' = L + \dv \Phi (t, q)$. $L$ порождает функционал действия
\[S[q] = \int_{t_1}^{t_2} L \dd{t} \rightarrow \var{S [q, \delta q]} = 0 \Leftrightarrow \text{ур. Лагранжа на $q(t).$}\]
А теперь для $L'$
\[S'[q] = \int_{t_1}^{t_2} L' \dd{t} = S[q] + \left. \Phi (t, q) \right|_{t_1}^{t_2} \rightarrow \var{S'} = \var{S},\]
поскольку последняя подстановка не даёт вклада в вариацию, и вариации зануляются на одних и тех же значения $q$ и $t$.

\item Рассмотрим преобразование координат в уравнении Лагранжа.
\begin{gather}
L(t, q, \dot{q}) \rightarrow\\
\dv{t} \frac{\partial L}{\partial \dot{q}}-\frac{\partial L}{\partial q}=0 \stackrel{q^* = \varphi(t, q)}{\longrightarrow} \text{как будет выглядеть уравнение $q^* (t)$?}\\
\left(\dv{\tilde{\varphi}}{t} = \pdv{\tilde{\varphi}}{t} + \pdv{\tilde{\varphi}}{q^*}\dot{q}^*, \quad q = \tilde{\varphi}(t, q^*) \right)\\
S[q] = \int_{t_1}^{t_2} L \dd{t} \equiv \int_{t_1}^{t_2} \underbrace{L \left(t, \tilde{\varphi}(t, q^*), \dv{\tilde{\varphi}}{t}\right)}_{L^* (t, q^*, \dot{q}^*)} \dd{t} = S^* [q^* (t)]\\
\delta S[q] = \delta S^* [q^*] =0
\end{gather}
\begin{rmk}
Если система была натуральной $L = L_0 + L_1 + L_2,$ то она останется натуральной при замене координат.
\end{rmk}

\item Замена координат и времени. Была Лагранжева задача
\begin{equation}
L(t, q, \dot{q}) \rightarrow \text{ур. Лагранжа} \rightarrow \begin{cases}
q^* = \varphi(t, q)\\
t^* = \psi(t, q)
\end{cases}
\rightarrow \text{? ур-я $q^*(t^*)$}
\end{equation}
\begin{equation}
\begin{cases}
q=\tilde{\varphi}\left(t^{*}, q^{*}\right)\\
t = \widetilde{\psi}(t^*, q^*)
\end{cases}
\end{equation}

\begin{equation}
S[q(t)] = \int_{t_1}^{t_2} L(t, q, \dot{q}) \dd{t} = \int_{t_1^*}^{t_2^*} \underbrace{L\left(\widetilde{\psi}, \tilde{\varphi}, \dv{\tilde{\varphi}}{\widetilde{\psi}}\right) \dv{\widetilde{\psi}}{t^*}}_{L^* \left(t^*, q^*, \dv{q^*}{t^*}\right)} \dd{t^*} = S^* [q^*(t^*)],
\end{equation}
но этот функционал тождественно связан с исходным, поэтому
\[\var{S} \equiv \var{S^*},\]
следовательно, $L^{*}$ порождает уравнения Лагранжа, которые в точности равны исходным, в которых произведена замена <<$*$>>.
\begin{equation}
\begin{rcases}
\dd{q}=\pdv{\tilde{\varphi}}{t^*} \dd{t^{*}}+\pdv{\tilde{\varphi}}{q^*}  \dd{q^{*}}\\
\dd{t} = \pdv{\widetilde{\psi}}{t^*} \dd{t^*} + \pdv{\widetilde{\psi}}{q^*}  \dd{q^{*}}
\end{rcases}
\Rightarrow
L^{*} = L \left( \widetilde{\psi}, \tilde{\varphi}, \frac{\pdv{\tilde{\varphi}}{t^*} +\pdv{\tilde{\varphi}}{q^*} \dv{q^*}{t^*}}{\pdv{\widetilde{\psi}}{t^*} + \pdv{\widetilde{\psi}}{q^*} \dv{q^*}{t^*}}\right) \cdot \Bigg(\underbrace{\pdv{\widetilde{\psi}}{t^*} + \pdv{\widetilde{\psi}}{q^*} \dv{q^*}{t^*}}_{\dv{t}{t^*}} \Bigg) \label{monster}
\end{equation}
\begin{rmk}
Система не остаётся натуральной!
\end{rmk}
\end{enumerate}

\begin{ex}
\begin{multicols}{2}
Натуральная система (Ньютоновская механика).
\[\vb{p}= m \vb{v} = \pdv{L}{\vb{v}} \Rightarrow L = \frac{mv^2}{2}.\]
\begin{equation}
\begin{rcases}
H = \vb{p} \cdot \vb{v} -L = L\\
H = T 
\end{rcases}
\Rightarrow L = T\; (\text{натуральность})
\end{equation}

Не натуральная система (релятивистская механика). $H/w$
\begin{gather}
\vb{p} = \frac{m\vb{v}}{\sqrt{1- v^2/c^2}} = \pdv{L}{\vb{v}} \Rightarrow L = - mc^2 \sqrt{1 - v^2/c^2}\\
H = \vb{p} \cdot \vb{v} - L = \frac{mv^2 + mc^2}{\sqrt{\ldots}} = \frac{mc^2}{\sqrt{1 - v^2/c^2}}\\
H = T \Rightarrow L \neq T\; (\text{ненатур.})
\end{gather}
\end{multicols}
\end{ex}

\subsubsection{Вариационный принцип и законы сохранения (теорема Нётер)} \index{Теорема! Нётер}
Связь уравнений, которые следуют из вариационного принципа, с законами сохранения --- вторая глобальная фишка, делающая ВП центральным в современной теоретической физике, первая --- уравнения, следующие из ВП ковариантны, то есть не зависят от ввода системы координат. 

Существует неразрывная связь симметрий в системе и уравнений движения. Теорема Нётер позволяет связать симметрии и следующие их них интегралы движения способом, не зависящим от способа ввода координат.

\begin{thm}[Теорема Нётер]
Каждому преобразованию координат и времени, оставляющему функционал действия инвариантным, отвечает первый интеграл уравнения движения.

Математически, если дано преобразование координат и времени
\begin{gather}
q^* = \varphi(t, q, \alpha)\\
t^* = \psi (t, q, \alpha),
\end{gather}
которое удовлетворяет следующим трём пунктам  условий,
\begin{enumerate}
\item $q = q^*,$ $t = t^*$ при $\alpha = 0.$

\item При $\abs{\alpha} < \varepsilon$ существуют обратные преобразования $\tilde{\varphi}(t^*, q^*, \alpha) = q,$ $\widetilde{\psi}(t^*, q^*, \alpha) = t.$

\item  (Симметрии.) $\Phi$-инвариантность\footnote{Используемая дальше функция $L^*$ определяется уравнением \eqref{monster}.} $L^* \left(t^*, q^*, \dv{q^*}{t^*}\right) = L \left(t^*, q^*, \dv{q^*}{t^*}\right) + \dv{t^*} \Phi(t^*, q^*, \alpha)$. \footnote{Эквивалентное утверждение для действия $S[q] = S^*[q^*] = S[q^*] + \left.\Phi \right|_{t_1^*}^{t_2^*}$.}
\end{enumerate}

то на решениях системы существует первый интеграл уравнения Лагранжа
\[I(t, q, \dot{q}) \left\{ \sum_{j=1}^s \pdv{L}{\dot{q}_j} \pdv{\varphi_j}{\alpha} - H \pdv{\psi}{\alpha} + \pdv{\Phi}{\alpha} \right\}_{\alpha = 0} = const.\]
\end{thm}

\begin{ex}[Преобразование сдвига]
\begin{equation}
x^* = x + \alpha, \Phi = 0 \Rightarrow I = \pdv{L}{\dot{x}} \left(\pdv{X^*}{x}\right)_{\alpha = 0} = p_x.
\end{equation}
\end{ex}
\begin{ex}[Поворот]
\begin{equation}
\begin{cases}
x^* = x \cos \alpha + y \sin \alpha\\
y^* = -x\sin \alpha + y \cos \alpha,
\end{cases}
\Phi = 0 \Rightarrow I = p_x \pdv{x^*}{\alpha} + p_y \pdv{y^*}{\alpha} = yp_x - xp_y = M_z
\end{equation}
\end{ex}
\begin{ex}[Винтовая симметрия]
К повороту из предыдущего примера добавим сдвиг $z^* = z + \frac{h}{2\pi} \alpha$, тогда
$I = M_z + p_z \frac{h}{2\pi}.$
\end{ex}
\begin{ex}
\[t^* = t + \alpha \Rightarrow I = -H.\]
\end{ex}
\begin{ex}[Преобразования Галилея]
\begin{equation}
\begin{cases}
x^* = x - \alpha t \quad(\alpha = u)\\
t^* = t
\end{cases}
\Rightarrow \pdv{x^*}{\alpha} = -t,
\end{equation}
при этом 
\begin{gather}
L = \frac{m}{2} \dot{x}^2,\; \dot{x} = \dot{x}^* + \alpha,\\
L^* = L(\dot{x}^* + \alpha) = \underbrace{\frac{m}{2} \dot{x}^*}_{L^*(\dot{x}^*)} + \underbrace{m \alpha \dot{x}^* + \frac{m}{2} \alpha^2}_{\dv{t}\left(\alpha mx^* + \frac{m\alpha^2}{2}t\right)} \Rightarrow\\
\Rightarrow I = p_x (-t) + mx = const\; (= 0),\\
p_x = m \frac{x}{t}.
\end{gather}

\end{ex}



\begin{proof}[Доказательство теоремы Нётер]
Попробуем свойства $\Phi$-инвариантности записать в пределе $\alpha \to 0$. Разложим в ряды преобразования координат  и времени
\begin{align}
q^* &= \varphi(t, q, \alpha) = q + \alpha \cdot Q + \ldots & Q_j &= \left. \pdv{\varphi_j}{\alpha} \right|_{\alpha = 0}\\
t^* &= \psi (t, q, \alpha) =  t + \alpha \cdot T + \ldots & T &= \left. \pdv{\psi}{\alpha} \right|_{\alpha = 0}.
\end{align}
Заметим, что 
\begin{equation}
\dv{q^*}{t^*} = \frac{\dot{q} + \alpha \dot{Q} + \ldots}{1 + \alpha \dot{T} + \ldots} = \dot{q} + \alpha (\dot{Q} - \dot{q} \dot{T}) + \ldots.
\end{equation}
По формуле \eqref{monster} получим выражение для $L^*$
\begin{gather}
L^* = L\left(t^* - \alpha T, q^* - \alpha Q, \dv{q^*}{t^*} - \alpha(\dot{Q} - \dot{q} \dot{T}) \right) \cdot \left(1 - \alpha \dot{T} \right) =\\
= L(t^*, q^*, \dv{q^*}{t^*}) - \alpha \bigg\{ \underbrace{\pdv{L}{t}}_{= - \dot{H} T} T + \underbrace{\pdv{L}{q} Q}_{\dot{p} Q} + \pdv{L}{\dot{q}} \left(\dot{Q} - \dot{q} \dot{T}\right) + L\dot{T}  \bigg\} =\\
= L(*)  - \alpha \left\{-\dv{t} (HT) + \dv{t}(pQ)\right\} + \ldots, \label{pre_phi_inv}
\end{gather}
так как $\pdv{L}{\dot{q}} \dot{Q} = p \dot{Q},$ а $-\pdv{L}{\dot{q}} \dot{q} \dot{T} + L\dot{T} = \dot{T} (L - \dot{q} p) = - H\dot{T}$. А дальше воспользуемся $\Phi$-инвариантностью, и перепишем полученное для $L^*$ выражение \eqref{pre_phi_inv}
\begin{equation}
 L^* = L(*) + \underbrace{\dv{t^*} \Phi(t^*, q^*, \alpha)}_{\dv{t}\left(\alpha \left. \pdv{\Phi}{\alpha}\right|_{\alpha = 0} + \ldots\right)}.
 \end{equation} 
 Приравняем одинаковые порядки, тогда
 \begin{equation}
 \dv{t} \left\{ pQ - HT + \left. \pdv{\Phi}{\alpha} \right|_{\alpha = 0} \right\} = 0,
 \end{equation}
а дифференцируемое выражение в точности и есть сохраняющийся интеграл движения.
\end{proof}

\subsubsection{Как вариационная формулировка работает в приближенных к реальным условиях?}
\[L(t, q, \dot{q}) \Rightarrow S[q(t)] = \int_{t_1}^{t_2} L \left(t, q(t), \dot{q}(t)\right) \dd{t}\]
\begin{gather}
\text{ур. Лагранжа}\; \Leftrightarrow \begin{cases}
\var{S} = 0\\
\var{q(t_1)} = \var{q(t_2)} = 0
\end{cases}
\text{(ВПГ)}\quad
\oplus \quad
\begin{cases}
q_1, q_2\; \text{близки так, что нет кин. фокусов,}\\
\text{сопряжённых с началом (концом) траектории}\\
S \to min\; (max)
\end{cases}
\end{gather}

Что мы выигрываем, зная, что уравнения Лагранжа следуют из решения вариационной экстремальной задачи?
\begin{enumerate}
\item Ковариантность относительно замены координат.
\item Теорема Нётер. 
\begin{gather}
\begin{cases}
q^* = q + \alpha \cdot Q(t, q) + O(\alpha^2)\\
t^* = t + \alpha \cdot T(t, q) + O(\alpha^2)
\end{cases}
+ \text{$\Phi$-инфариантность}\; L^{*} = L \left(t^*, q^*, \dv{q^*}{t^*}\right) + \dv{t^*} \Phi (t^*, q^*, \alpha),\\
\left(t\; \text{и}\;q\; \text{не равны нулю одновременно,}\;\Phi = \Phi_0 + \alpha \cdot \Xi + O(\alpha^2)\right),
\end{gather}
тогда существует интеграл движения
\begin{gather}
\pdv{L}{\dot{q}}Q + H \cdot T + \Xi = const\; \text{на реш. ур. Лагранжа.}
\end{gather}
\begin{rmk}
\begin{gather}\begin{cases}
q^* = \varphi(q, t)\\
t^* = \psi(q, t)\\
\ldots
\end{cases} \Rightarrow
\begin{cases}
Q = \left. \pdv{\varphi}{\alpha} \right|_{\alpha= 0},\\
T = \left. \pdv{\psi}{\alpha} \right|_{\alpha = 0},\\
\Xi = \left. \pdv{\Phi}{\alpha} \right|_{\alpha= 0}
\end{cases}
\end{gather}
\end{rmk}
\end{enumerate}

\subsubsection{Лагранжиан свободной материальной точки.}\index{Лагранжиан! свободной материальной точки}
\paragraph{В Ньютоновской механике}\! из дифференциального подхода для свободной материальной точки мы знаем
\begin{equation}
L = \frac{mv^2}{2},
\end{equation}
но можно стартовать с вариационного принципа, и последний не утверждает, что $L = T -U,$ стартуем с того, что система описывается $L(t, q, \dot{q}),$ выведем механику сил.
Воспользуемся теми же принципами, что и для Ньютоновской механики.
\begin{enumerate}
\item Пространство однородно и изотропно. Время однородно. $\Rightarrow L (\not{t}, \not{\vb{r}}, \vb{v}) = L(v^2).$
Зная, что у нас $L(v^2),$ можем доказать первый закон Ньютона:
\begin{gather}
\vb{p} = \pdv{L}{\vb{v}} = \pdv{L}{v^2} \pdv{v^2}{\vb{v}} = \pdv{L}{v^2} \cdot 2\vb{v} = const\; (\text{$\vb{r}$  --- цикл.}) \Rightarrow v =const, \vb{v} = const\\
H = \vb{p}\cdot \vb{v} - L = 2v^2 \pdv{L(v^2)}{v^2} - L = const\; (\text{$t$  --- цикл.}),
\end{gather}
двумя способами получили :)
\item Чтобы воспользоваться теоремой Нётер, надо придумать какое-то преобразование, оставляющее уравнение движения инвариантным, вспомним про преобразование Галилея, которое тоже как бы свойство пространства-времени, делающее эквивалентными все инерциальные системы отсчёта в Ньютоновской механике. 
\begin{gather}
\begin{cases}
t^* = t\\
r^* = \vb{r} - \vb{u} \cdot t;\; \vb{u} = \alpha \cdot \vb{n}_{const}
\end{cases}
\stackrel{Th.\, Noether}{\longrightarrow} 
\begin{cases}
T = 0\\
Q = - \vb{n}\cdot t
\end{cases} \Rightarrow
\boxed{-\pdv{L}{v^2} 2(\vb{n} \cdot \vb{v})t = F (t, \vb{r})}\dv{t} \rightarrow\\
\pdv{L}{v^2} \cdot 2(\vb{n} \cdot \vb{v}) = \pdv{F(t, \vb{r})}{t} + \pdv{F(t, \vb{r})}{\vb{r}}\vb{v},\\
\intertext{но левая часть полученного равенства не зависит от времени и координат, поэтому правая тоже от них не зависит, значит,}
\pdv{L}{v^2} \cdot 2(\vb{n} \cdot \vb{v}) = \pdv{F(t, \vb{r})}{t} + \pdv{F(t, \vb{r})}{\vb{r}}\vb{v} = a + (\vb{b}, \vb{n}) \stackrel{\vb{b} =\vb{n}\cdot m}{=} m (\vb{n}, \vb{v})\\
\left(H/w\quad (\vb{n}, \vb{v}) = a + (\vb{b}, \vb{v}) \Rightarrow a =0, \vb{b} = \vb{n} \right)
\end{gather}
и мы, хитро подобрав константы, получили то, что нам надо в ответе:
\[\pdv{L}{v^2} = \frac{m}{2} \Rightarrow L = \frac{mv^2}{2} + const.\]
Ландау и Лифшиц дальше для несвободной материальной точки постулируют силы, то есть к лагранжиану свободной материальной точки аддитивно добавляют взаимодействие с внешними полями
\[L = \frac{mv^2}{2} - U(t, \vb{r}, \vb{v}) \ldots,\]
возможность так делать постулируется.
\end{enumerate}

\paragraph{В СТО}\! то же, что в предыдущем пункте, но с преобразованиями Лоренца
\begin{equation}
\begin{cases}
\vb{r}^* = \frac{\vb{r}-\vb{u} t}{\sqrt{1-u^{2} / c^{2}}}\\
t^* = \frac{t-(\vb{u}, \vb{r}) / c^{2}}{\sqrt{1-u^{2} / c^2}}.
\end{cases}
\stackrel{\vb{u} = \alpha \cdot \vb{n}}{\Longrightarrow}
\boxed{
\begin{cases}
\vb{r}^* = \vb{r} - \alpha \vb{n} \cdot t + \ldots\\
t^* = t - \alpha (\vb{n}, \vb{r})/c^2 + \ldots
\end{cases}}.
\end{equation}
По-прежнему время и пространство однородны, пространство изотропно. Значит, как и раньше,
\[\vb{p} = 2\vb{v} \pdv{L}{t^2}; H = 2v^2 \pdv{L}{v^2} - L \Rightarrow \vb{p} = const, \vb{v} = const, v = const.\]
Чтобы узнать форму $L(v^2)$, воспользуемся теоремой Нётер
\begin{gather}
\vb{p} Q - H \cdot T = - F(t, \vb{r})\\
-2(\vb{n}, \vb{v}) \pdv{L}{v^2} \cdot t + \left(2v^2 \pdv{L}{v^2} - L\right) \frac{(\vb{n}, \vb{r})}{c^2} =  F(t, \vb{r})\; \left| \dv{t} \right.\\ 
(\vb{n}, \vb{v}) \left\{ \left(2v^2 \pdv{L}{v^2} - L\right)\frac{1}{c^2} - 2\pdv{L}{v^2}\right\} = \underbrace{\pdv{F}{t}}_{=0} + \underbrace{\pdv{F}{\vb{r}}\vb{v}}_{\parallel \vb{n}} = (\vb{n}, \vb{v}) \cdot \underbrace{\frac{L_0}{c^2}}_{const}\\
-\pdv{L}{v^2}\left(1-\frac{v^{2}}{c^{2}}\right)=\frac{1}{2 c^{2}}\left(L-L_{0}\right)
\end{gather}
\begin{gather}
\pdv{\ln (L-L_0)}{v^2} = \hlf \frac{1}{v^2 - c^2} = \hlf \pdv{v^2} \ln \abs{v^2 - c^2}\\
L = L_0 + A \sqrt{c^2 - v^2} \to \frac{mv^2}{2}\; \text{при $v \to 0$}\; \stackrel{cA= - mc^2}{\Rightarrow} \boxed{L = - mc^2 \sqrt{1 - v^2 / c^2} + mc^2} \Rightarrow\\
H = \frac{mc^2}{\sqrt{1- v^2/c^2}} \neq L,
\end{gather}
то есть $L \neq T,$ система ненатуральная; $\vb{p} = \frac{m\vb{v}}{\sqrt{1- v^2/c^2}}.$

\subsection{Электромеханические аналогии}
\textbf{(Смешанные электромеханические системы в механике Лагранжа)}
\begin{ex}
\begin{equation}
U = \mathcal{E},\; U = \frac{q}{C},\; \mathcal{E} = -L\dot{I} = - L \ddot{q} \Rightarrow
\end{equation}
\begin{gather}
L \ddot{I} + \frac{1}{C} q = 0\; \text{ --- гармонический осциллятор с $\omega^2 = \frac{1}{LC},$}\\
L \ddot{I} + \frac{1}{C} I = 0\; \left| \cdot CL \right. \Longrightarrow CL^2 \ddot{I} + L I = 0.
\end{gather}
Будем рассматривать заряд на обкладках в качестве обобщённой координаты, тогда, рассматривая индуктивность в роли массы, запишем функцию Лагранжа
\begin{equation}
\mathcal{L} = \hlf L\dot{q}^2 - \frac{1}{2C} q^2 = T -U: 
\begin{cases}
T = \hlf LI^2\; \text{--- энергия магнитного поля в катушке,}\\
U = \frac{1}{2C} q^2\; \text{--- энергия энергия электрического поля в конденсаторе.}
\end{cases}
\end{equation}
Точно так же в качестве обобщённой координаты можно рассматривать ток (тогда роль массы играет выражение $CL^2$):
\begin{equation}
\mathcal{L} = \hlf CL^2 \dot{I}^2 - \hlf LI^2 = T - U,
\begin{cases}
T = \hlf CU^2,\\
U = \frac{1}{2} LI^2,
\end{cases}
\end{equation}
и, когда контур замкнутый, разницы никакой нет. Разница возникает, если контур разомкнуть. $H/w$
\end{ex}

\begin{ex}
\begin{gather}
\mathcal{L}_{\text{мех}} = \frac{m \dot{x}^{2}}{2}-\frac{k x^{2}}{2}-\operatorname{mg} x\\
\mathcal{L}_{\text{эл}} = \frac{1}{2} L(x) \dot{q}^{2}-\frac{1}{2 C(x)} q^{2},\\
\left(C(x) = \frac{S_C}{4\pi x}, L(x)  = 4\pi \frac{N^2 S_L}{L-x} \right)\\
\mathcal{L} = \mathcal{L}_{\text{мех}} + \mathcal{L}_{\text{эл}}
\end{gather}
Получим уравнения Лагранжа.
\begin{equation}
\dv{t} \pdv{\mathcal{L}}{\dot{x}} = m\ddot{x} = \pdv{\mathcal{L}}{x} = - kx -mg + \frac{1}{2}\pdv{L}{x} \cdot \dot{q}^2 - {\frac{q^2}{2} \pdv{x} \frac{1}{C}}\footnote{$-\frac{q^2}{2} \frac{4\pi}{S_C} = -2\pi \sigma q = Eq$ --- в точности электрическая сила взаимодействия двух пластин конденсатора. Можно показать, что в точности совпадает с силой сопротивления сжатию соленоида предыдущий член, да и все слагаемые могут быть получены из первых принципов.}
\end{equation}
\end{ex}
У нас теперь две степени свободы. Найти $q$  и $\dot{q}$ можно из второго уравнения движения:
\begin{equation}
\dv{t} \pdv{\mathcal{L}}{\dot{q}} = \dv{t} (L(x) \dot{q}) = L \ddot{q} + \pdv{L}{x} \dot{x} \dot{q} = \pdv{\mathcal{L}}{q} = \frac{q}{C}.
\end{equation}

\section{Интегрируемые задачи механики}
\subsection{Интегрирование уравнения движения систем с одной степенью свободы}
Мы будем интересоваться решением ДУ вида $\ddot{x} = F(t, x, \dot{x}).$\footnote{$x= 1;\; \dim x =1, s =1$}
\subsubsection{Классификация состояний равновесия автономной системы на плоскости}
\begin{equation}
\ddot{x} = F(x, \dot{x}) \Rightarrow \begin{cases}
\dot{x} = f(x, y)\\
\dot{y} = g(x, y)
\end{cases}
\end{equation}
Состояние равновесия
\begin{equation}
\begin{cases}
\dot{x} = 0\\
\dot{y} = 0
\end{cases}
\Leftrightarrow
\begin{cases}
f(x_0, y_0) =0\\
g(x_0, y_0) = 0.
\end{cases}
\end{equation}
Рассматриваемое малое возмущение
\begin{equation}
\begin{cases}
x = x_ 0 + \xi\\
y = y_0 + \eta 
\end{cases}
\Rightarrow
\begin{cases}
\dot{xi} = \alpha \xi + \beta \eta\\
\dot{\eta} = \gamma \xi + \delta \eta,
\end{cases}
\alpha = \left. \pdv{f}{x} \right|_{x_0, y_0}, \beta = \left. \pdv{f}{y} \right|_{x_0, y_0}, \ldots, \delta = \left. \pdv{g}{y} \right|_{x_0, y_0}.\footnote{Рассматриваем линейное состояние равновесия.}
\end{equation}
Будем искать решение в виде 
\begin{gather}
\begin{pmatrix}
\xi\\
\eta
\end{pmatrix} = 
\begin{pmatrix}
a\\
b
\end{pmatrix}
e^{\lambda t} \Rightarrow \begin{cases}
\lambda a = \alpha a + \beta\\
\lambda b = \gamma a + \delta b
\end{cases} \Rightarrow 
\det \begin{pmatrix}
\alpha - \gamma & \beta\\
\gamma & \delta - \lambda
\end{pmatrix}
= 0 \Leftrightarrow\\
\Leftrightarrow (\lambda - \alpha)(\lambda - \delta) = \gamma \beta,\\
\lambda_{1,2} = \frac{\alpha + \beta}{2} \pm \sqrt{\left(\frac{\alpha - \delta}{2}\right)^2 - \gamma \beta}.
\end{gather}
\begin{enumerate}
\item $\lambda = \pm i \omega$ --- состояние равновесия типа центр, устойчивое, не асимптотически устойчивое. Фазовые кривые --- эллипсы.
\item $\lambda p \pm i \omega$ --- состояние равновесия типа фокус, может быть в зависимости от знака $p$ как устойчивым, так и неустойчивым\footnote{Устойчиво при $p < 0$.}, устойчивое равновесие асимптотически устойчиво. Фазовые кривые <<сжимаются>> или <<раскручиваются>>.
\item $\lambda \in \mathbb{R}, \lambda_1 \cdot \lambda_2 > 0$ ---состояние равновесия типа узел --- равновесие типа фокус при очень сильном трении. Асимптотически устойчиво либо неустойчиво.
\item $\lambda \in \mathbb{R}, \lambda_1 \cdot \lambda_2 < 0$ --- состояние равновесия типа седло, неустойчиво.
\end{enumerate}
$H/w$ Как найти направления асимптот?

\begin{ex}
\begin{gather}
\ddot{x} = f(x, v), v =0, m = 1 \\
\begin{cases}
\dot{x} = v\\
\dot{v} = f_0(x) - \mu \cdot v = - \pdv{U(x)}{x} - \mu \cdot x
\end{cases}\\
\pdv{U(x_0)}{x} = 0; \pdv[2]{U(x_0)}{x} = U'' \neq 0; \begin{cases}
x = x_0 + \xi\\
v = \eta
\end{cases}
\Rightarrow\\
\begin{cases}
\dot{\xi} = v\\
\dot{v} = -U'' \cdot \xi - \mu \cdot v
\end{cases} \Rightarrow 
\det \begin{pmatrix}
-\lambda & 1\\
-U'' & -\lambda - \mu
\end{pmatrix} = 0 \Rightarrow \boxed{\lambda^2 + \mu \lambda + U'' = 0} \Rightarrow \\
\lambda_{1, 2} = - \frac{\mu}{2} \pm \sqrt{\frac{\mu^2}{4} - U''}
\end{gather}
Чередуются состояние равновесия типа седло, узел и фокус, центры превращаются в диссипативные состояния равновесия, фокусы отвечают случаям малого трения, при её увеличении превращаются в узлы.
\end{ex}

\subsubsection{Интегрирование уравнения движения консервативных одномерных систем}
Будем рассматривать лагранжеву задачу, в которой существует интеграл энергии.
\[L(x, \dot{x})\; \text{--- уравнения Лагранжа решаются в квадратурах.}\]
\begin{equation}
\pdv{L}{t} = 0 \Rightarrow H(x, \dot{x}) = const\; \text{на решениях}.
\end{equation}
Фазовый портрет --- линии уровня $H(x, \dot{x})$.\index{Фазовый портрет}
\begin{gather}
H(x, \dot{x}) = E = const \Rightarrow \dot{x} = V(x, E)\\
\dv{x}{t} = V(x, E) \Rightarrow \dd{t} = \frac{\dd{x}}{V(x, E)} \Rightarrow \boxed{t= \int^x \frac{\dd{x}}{V(x, E)}},
\end{gather}
аддитивная константа в нижнем пределе интегрирования отвечает заданному $E$, его линии уровня, и такие точки надо различать, поэтому на самом деле везде в решении вместо $V(x, E)$ должно быть $V_i (x, E),$ где $i$ определяет \textit{ветку однозначности}.\index{Ветка однозначности}

\begin{dfn}
$x$ --- ограниченная область --- движение финитное; $x$ --- неограниченная область --- движение инфинитное.
\end{dfn}
\begin{pst}
Всякое финитное движение в одномерной консервативной системе обязательно является периодическим. 
\end{pst}

Частный случай --- натуральная система.
\begin{gather}
L(x, \dot{x}) = \frac{1}{2} m(x) \dot{x}^2 - U(x)\\
H(x, \dot{x}) = \frac{1}{2} m \dot{x}^2 + U(x) = E \Rightarrow \dot{x} = \pm \sqrt{\frac{2}{m}\left(E- U(x)\right)}\\
\boxed{t = \pm \int^x \sqrt{\frac{m(x)}{2} \frac{\dd{x}}{\sqrt{E - U(x)}}}},
\end{gather}
знак в начале определяется начальными условиями, меняется в точке разворота, где знаменатель обращается в нуль, то есть $U(x) = E$. 
\begin{gather}
p(x) = \pdv{L}{\dot{x}} = m(x) \dot{x} = \pm \sqrt{2m (E - U)}\\
\pdv{p}{E} = \sqrt{\frac{m}{2}}\frac{1}{\sqrt{E- U}} \Rightarrow t = \pm \pdv{E} \int^x p (x, E) \dd{x},
\end{gather}
а интеграл --- площадь под графиком $p(x)$.
\begin{dfn}
$U(x) < E $ --- область возможного движения, иначе мы переходим в комплексное время, что запрещено в классической механике.
\end{dfn}
\begin{dfn}
$U(x) = E$ --- точки разворота.
\end{dfn}

Фазовый портрет обладает зеркальной симметрией относительно оси $x$.
$[x_3, \infty]$ --- инфинитное движение.

\[\dot{x} \sim \sqrt{E - U(x)} \sim \pm \sqrt{x - x_1}\]
$\dot{x} \sim \pm \sqrt{(x - x_0)^2} = \pm \abs{x-x_0}$ --- в состоянии равновесия с асимптотами $\left(\pdv{U(x_0)}{x} =0\right)$.

\subsubsection{Диссипативные одномерные системы}
В общем случае способа решения в квадратурах для диссипативных систем нет. И оказывается, что все известные интегрируемые диссипативные системы на самом деле не по-настоящему диссипативные. Поясним, о чём идёт речь. Вспомним консервативную задачу $L = \frac{1}{2} m(x) \dot{x}^2 - U(x)$ и посмотрим, какому уравнению движения такой лагранжиан соответствует (а интересует нас уравнение вида $\ddot{x} = F(t, x, \dot{x})$).
\begin{equation}
L = \frac{1}{2} m(x) \dot{x}^2 - U(x) \Rightarrow m\ddot{x} + \frac{1}{2} \pdv{m}{x} \dot{x}^2 = -  \pdv{U}{x} \Rightarrow \begin{cases}
\ddot{x} = f(x) - \mu (x) \dot{x}^2,\; \text{где}\\
f(x) = - \frac{1}{m(x)} \pdv{U}{x}, \; \text{а} \\
\mu({x}) = \frac{1}{2m} \pdv{m}{x}, 
\end{cases}
\end{equation}
и мы можем получить решение в квадратурах для любых $f(x)$ и $\mu(x)$:
\begin{align} 
\mu({x}) &= \frac{1}{2m} \pdv{m}{x} \Rightarrow \mu = \pdv{x}(\ln \sqrt{m}) \Rightarrow m = \left( e^{\int \mu \dd{x}}\right)^2,\\
f(x) &= - \frac{1}{m(x)} \pdv{U}{x} \Rightarrow U = - \int m f \dd{x}.
\end{align}
Задачи вида $\ddot{x} + \omega_0^2 x + \mu \dot{x}^n =0,$ где $n = 2k$, допускают решение в квадратурах, в системе <<псевдотрение>>. При $n = 2k +1$ трение уже истинное, поскольку <<штука>> $\dot{x} \cdot \mu \dot{x}^n$ знакоопределена, а она управляет скоростью вытекания энергии:
\begin{equation}
 H_0 = \frac{1}{2} \dot{x}^2 + \frac{\omega^2}{2}x^2 \Rightarrow \dv{H_0}{t} = - \mu \dot{x}^{n+1}.
 \end{equation} 
Но можно и при $n = 2k$ сделать трение: достаточно добавить сигнум $\ddot{x} + \omega_0^2 x + \mu \sign \dot{x} \cdot \dot{x}^n = 0,$ при $n = 2k$ получаем трение, при $n = 2k +1$ --- псевдотрение.
\begin{ex}[n= 0]
Как выглядит фазовый потрет системы $\ddot{x} + \omega_0^2 x + \mu = 0$? А если добавить сигнум, как увидеть, что задача стала диссипативной?
Как выглядит настоящее диссипативное решение? Сшиваем по непрерывности два решения. Зона застоя.

Когда рисуем сигнум скорости, говорим, что фазовой фазовый портрет системы получается из фазового портрета системы без сигнума путём склейки: разрезаем по горизонтальной прямой, с помощью сигнума разворачиваем и склеиваем по непрерывности.

\emph{H/w} $\sign \dot{x} \cdot \dot{x}$ --- задача с трением --- нарисовать фазовый портрет, зная ФП консервативной системы.
\end{ex}

\paragraph{Тривиальные случаи}
$\dot{x} = F(v) \Rightarrow \dot{v} = F(v) \Rightarrow v(t) \Rightarrow x(t) \int v \dd{t}$

\subsubsection{Качественный анализ на фазовой плоскости}
...
\begin{ex}[Линейное трение в системе можно учитывать с помощью чисто лагранжевой задачи с нестационарным лагранжианом]
\begin{align}
L &= \frac{1}{2} \dot{x}^2 - U(x) &  &\longrightarrow &  \ddot{x} &= - \pdv{U}{x} = f(x),\\
\intertext{и в эту задачу мы можем добавить линейное трение:}
L &= \left(\frac{1}{2} \dot{x}^2 - U\right) e^{2\gamma t} &  &\longrightarrow & p &= \pdv{L}{\dot{x}} = \dot{x} e^{2 \gamma t}\\
& & & & \dot{p} &=(\ddot{x} + 2\gamma \dot{x})e^{2\gamma t} = \pdv{L}{x} = -\pdv{U}{x} e^{2\gamma t} 
\end{align}
И последнее равенство, которое можно сократить на $e^{2\gamma t}$ является следствием одной теоремы\dots
\end{ex}
\begin{pst}
\begin{equation}
\begin{rcases}
\ddot{x} = F(t, x, \dot{x})\\
s =1
\end{rcases}
\Rightarrow
 \exists \left.
\begin{matrix} 
L(t, x, \dot{x})\\
\mu (t, x, \dot{x})\neq 0
\end{matrix}
\right|\; (\ddot{x} - F) \cdot \mu = \dv{t} \pdv{L}{\dot{x}} - \pdv{L}{x}
\end{equation}
\end{pst}

\begin{pst}
\begin{equation}
\ddot{x} = F(x, \dot{x}) \Rightarrow
\exists \;
\begin{matrix} 
L(x, \dot{x})\\
\mu (x, \dot{x})
\end{matrix}
\quad \ldots
\end{equation}
\end{pst}
\begin{ex}
\begin{equation}
\ddot{x} + \omega^2 x + 2\gamma \dot{x} = 0 \Leftarrow L = \frac{\dot{x} + \gamma x}{\gamma x} \arctg \frac{\dot{x} + \gamma x}{\omega x} -\frac{\omega}{2\gamma} \ln \left( \omega^2 x^2 + (\dot{x} + \gamma x)^2 \right)\footnote{\emph{H/w} проверить. См. статью \cite{Shalashov_2017}.}
\end{equation}
\begin{equation}
\begin{cases}
x(t) = A e^{-\gamma t} \sin (\omega t + \gamma)\\
p(t)  = \pdv{L}{\dot{x}}\; \text{уходит в бесконечность за конечное время,}
\end{cases}
\end{equation}
движение инфинитно.
\end{ex}

\documentclass[12pt, a4paper]{article}
\usepackage[utf8]{inputenc}
 \usepackage[T1, T2A]{fontenc}
\usepackage[english, russian]{babel}
\usepackage{caption}
\usepackage{indentfirst}
\usepackage{graphicx, xcolor}
\usepackage{cmap}

\usepackage[unicode, pdftex]{hyperref}
\hypersetup{linkcolor=blue, urlcolor=blue, colorlinks=true}
\usepackage{hyphenat}
\hyphenation{объект}
\usepackage{wrapfig}
\usepackage[left=1.4cm,right=1.4cm,top=1.5cm,bottom=1.5cm,bindingoffset=0cm]{geometry}
\usepackage{tocloft}    
\usepackage{titlesec} \titlelabel{\thetitle.\quad} 
\frenchspacing
\makeatletter
\renewcommand{\@biblabel}[1]{#1.} % Заменяем библиографию с квадратных скобок на точку:
\makeatother
\makeindex

\usepackage[]{mathtools}
\renewcommand{\theequation}{\thesection.\arabic{equation}}
    
\parindent=1.25cm
%\parskip=0.1cm

\usepackage{physics} 
\usepackage{siunitx} % typesets numbers with units very nicely
\usepackage{amssymb,amsfonts,amsmath,mathtext,cite,enumerate,float}
\DeclareMathOperator{\sign}{sgn}
\usepackage{amsthm}

%\usepackage[dvips]{graphicx}

\usepackage{multicol}

\bibliographystyle{unsrt}


\begin{document}
\renewcommand{\cftsecaftersnum}{.}
\renewcommand{\cftsubsecaftersnum}{.}

\renewcommand\refname{Список литературы}

\theoremstyle{plain}
\newtheorem{thm}{Теорема}[section]
\newtheorem{lem}[thm]{Лемма}
\newtheorem{pst}{Постулат}[section]

\theoremstyle{definition}
\newtheorem{dfn}{Определение}[section]
\newtheorem{cns}[thm]{Следствие}

\theoremstyle{remark}
\newtheorem{task}{Задача}[section]
\newtheorem{ex}{Пример}[subsection]
\newtheorem{cex}[ex]{Контрпример}
\newtheorem{rmk}{Замечание}[subsection]

\newcommand*{\eqdef}{\stackrel{\mathrm{def}}{=}}
\newcommand*{\is}[1]{\stackrel{\mathrm{\eqref{#1}}}{=}}
\newcommand*{\eqq}[1]{\stackrel{\mathrm{#1}}{=}}
\newcommand*{\hlf}{\frac{1}{2}}

\columnseprule = 0.4pt

\mathtoolsset{showonlyrefs=true}
\mathtoolsset{showmanualtags=true}

\begin{center}
\Huge{\textbf{Теоретическая механика}}
\end{center}
\tableofcontents
\newpage
\input{section1.tex}
\section{Механика Лагранжа}
Стартуем с механики Ньютона:
\[m_i \Ddot{\vec{r}}_i = \Vec{F}_i (\vec{r}_1, \ldots, \vec{r}_N, \vec{v}_1, \ldots, \vec{r}_N, t),\; i=\overline{1,N}, \] но беда в том, что для ряда сил мы знаем результат их действия, а не сами силы.
\begin{dfn}
Связи \index{Связи!} --- не вытекающие из уравнения движения ограничения на положения точек $\lbrace\vec{r}_i, \vec{v}_i\rbrace$.
\[m_i \Ddot{\vec{r}}_i = \vec{F}_i^{(a)} + \vec{R}_i \] Так, выше указана несвободная система \index{Несвободная система} --- на неё наложены связи. $\vec{F}_i^{(a)}$ --- активные силы (их знаем), $\vec{R}_i$ --- силы реакции (знаем связи)\footnote{Силы, с которыми  тела, осуществляющие связи, действуют на точки системы называются реакциями связей.}.
\end{dfn}
\subsection{Связи и их классификация.}
Различают голономные и неголономные, удерживающие и неудерживающие, стационарные и нестационарные связи.
\begin{dfn}
Голономными (или интегрируемыми) связями называют связи, уравнения которых всегда можно свести к уравнениям вида
\begin{equation}
f(\vb{r}_1, \ldots, \vb{r}_N, t) = 0,
\end{equation}
где $f$ является функцией только координат точек и времени. Эти связи накладывают ограничения не только на положение, но и на скорости и ускорения точек системы.
\end{dfn}
\begin{dfn}
Неголономными  (неинтегрируемыми) связями называют связи, уравнения которых нельзя свести к уравнениям, содержащим только координаты точек и время. Неголономной, например, является связь, налагаемая на шар, катящийся по шероховатой поверхности.
\end{dfn}

\[f(t, \lbrace\vec{r}_i\rbrace, \lbrace\vec{v}_i \rbrace) = 0,\] связи в виде равенств --- удерживающие связи. \index{Связи! удерживающие}
\[f(\dots) \geqslant 0 \text{--- неудерживающиие, ненапряжённые связи.}\]  Неудерживающие связи впервые появились только в теории атомного ядра; математически их можно представить в виде удерживающей связи, подставив в степ-функцию.
...\footnote{см. \cite{OlhTM}, c. 204, \cite{GantnakherTM}, с. 201.}
\begin{gather}
f(t, \lbrace\vec{r}_i\rbrace) = 0 \Rightarrow \notag\\
\pdv{f}{t} + \sum_{i=1}^N \pdv{f}{\vec{r}_i} \vec{v}_i = 0. \label{golonom}
\end{gather}
Конечные, дифференцируемые, недифференцируемые, интегрируемые, голономные \eqref{golonom}.
Стационарные (склерономные)($\not t$), нестационарные (реаномные)($t$) связи.
\begin{ex}

\end{ex}

\subsection{Основная задача механики. Идеальные связи}
\subsection{Идеальные связи и уравнения Лагранжа первого рода}
Есть $\{\vb{r}_i\}$. Договорились, что 
\begin{equation}
m_i \ddot{\vb{r}}_i = \vb{F}_i + \vb{R}, \quad i = \overline{1, N},
\end{equation}
то есть поделили на активные силы и силы реакции связей, но плохо то, что знаем не все силы правой части. Из-за $R_i$-ых возникают $3N$ новых величин, связи дают лишь $K$ величин, и мы хотим выяснить, когда у нас задача согласована.
Вспомним про голономные связи, которые могут быть представлены в виде \begin{equation}
f_j (\{\vb{r}_i\}, t) = 0, \quad j = \overline{1, K}.
\end{equation} 
\begin{dfn}
Возможное перемещение --- произвольное бесконечно малое перемещение точек системы, которое согласовано со связями. Формально
\begin{gather}
\left. \sum_i \pdv{f_j}{\vb{r}_i}\vb{v}_i + \pdv{f_j}{t} \equiv 0 \right| \cdot \dd{t},\\
\intertext{то есть умножаем на $\dd{t}$ интегрируемую связь, тогда}
\sum_i \pdv{f_j}{\vb{r}_i} \dd{\vb{r}_i} + \pdv{f_j}{t} \equiv 0,
\end{gather}
и решение этой системы $K$ уравнений является совокупностью всех возможных перемещений.
\end{dfn}

\begin{dfn}
Действительное перемещение --- бесконечно малое перемещение, совместимое со связями и уравнениями движения (и оно единственно как решение задачи Коши).
\end{dfn}

Если <<заморозить>> время, то есть <<забыть>> про частную производную $\pdv{f}{t}$,  то 
\[\left. \sum_i \pdv{f_j}{\vb{r}_i} \vb{v}_i = 0 \right| \cdot \dd{t} \quad \Rightarrow \sum_i \pdv{f_j}{\vb{r}_i} \var \vb{r}_i = 0 \footnote{$\var \vb{r}_i = \vb{v}_i \dd{t}$ --- дифференциал при замороженном времени.}\] 

\begin{dfn}
Виртуальное перемещение --- бесконечно малое перемещение, совместимое со связями при замороженном времени.
\end{dfn}

Виртуальному перемещению можем сопоставить виртуальную работу:
\[\{\var \vb{r}_i\} \longrightarrow \var A = \sum_i \vb{F}_i \var \vb{r}_i = \sum_i \vb{F}_i^{(a)} \var \vb{r}_i + \sum_i \vb{R}_i \var \vb{r}_i.\]
И оказывается, что почти в любом идеализированном механизме без трения $\var A_R = 0$.

\begin{dfn}
Связь называется идеальной, если $\var A_R = 0 \quad \forall \{\var \vb{r}_i\},$ то есть если виртуальная работа сил реакции связей равна нулю при любом виртуальном перемещении.
\end{dfn}
\noindent Примером идеальной связи может служить движение по гладкой неподвижной поверхности.


\textit{Эмпирическое утверждение.} Почти все связи в механике являются идеальными. Но трение  (попытка учесть немеханическое явление в механике) разрушает идеальность, и мы не знаем, как устроены связи, то есть мы обычно идеализируем задачи и почти всегда угадываем, но при этом сядем в лужу, если уйдём в очень большие масштабы --- в космологию, или в очень малые...

\begin{thm}
Пусть дана идеальная связь:
\begin{equation}
\begin{cases}
\sum_i \vb{R}_i \var \vb{r}_i = 0,\\
\sum_i \pdv{f_j}{\vb{r}_i} \var \vb{r}_i = 0 
\end{cases} \Longleftrightarrow
\exists\; \lambda_j (t): \; \vb{R}_i = \sum_{j=1}^K \lambda_j \pdv{f_j}{\vb{r}_i},
\end{equation}
то есть решили основную задачу механики.
\end{thm}
Перед доказательством применим сформулированную теорему, рассмотрим следствие из неё. 
\begin{cns} Посмотрим, как будут записываться уравнения движения.
\begin{equation}
\begin{cases}
m_i \ddot{r}_i = \vb{F}_i^{(a)} + \sum_{j=1}^K \lambda_j \pdv{f_j}{\vb{r}_i}\\
f_j (t, \{\vb{r}_i\}) = 0,
\end{cases}
\end{equation}
и эта задача уже математически корректна: в ней число переменных соответствует числу уравнений: $3N+K$ неизвестных, $3N+K$ соотношений. Уравнения движения в такой форме для несвободной системы называют \textit{уравнениями Лагранжа первого рода}. \index{Уравнения Лагранжа! первого рода}

И как эти уравнения можно решать?
Метод Лагранжа.
\begin{enumerate}
\item $\forall \lambda(t) \Rightarrow \vb{r}(t, \lambda)$ из уравнений движения.
\item $f(\{\vb{r}(t, \lambda\}, t) \equiv 0 \Rightarrow \lambda.$
\end{enumerate}
\end{cns}
\begin{rmk}
$\pdv{f}{t} = 0 \Rightarrow \pdv{\lambda}{f} = 0$
\end{rmk}

\newpage
\subsection{Линейные колебания в лагранжевых системах}
Начнём потихонечку искать решения уравнений Лагранжа.
\subsubsection{Одномерное движение}
Пусть нам дана одномерная (s =1) натуральная лагранжева система. Будем временно использовать букву $x $ вместо $q$. Имеем полином не выше второй степени (в силу натуральности системы):
\[L(t, x, \dot{x})  = \frac{1}{2} \alpha(x) \dot{x}^2 + \beta(x) \dot{x} - U(x),\]
уже есть некоторая нестыковка, потому что $\alpha, \beta$ могут зависеть от времени --- сделаем упрощения.
\begin{enumerate}
\item $\pdv{L}{t} = 0.$
\item Линейный по скорости член --- гироскопическая сила, но в одномерном случае никаких гироскопических сил не существует, поэтому можем этот член выкинуть, поскольку всегда найдётся $\beta,$ т. ч.
\[\beta \dot{x} = \dv{t} \int \beta \dd{x}.\]
\begin{dfn}[Состояние равновесия]
Состояние равновесия --- решение уравнения движения --- тождественная константа.
$x = x_0 = const\; (\dot{x} = \ddot{x} = \dots = 0).$
\end{dfn}
\item Диссипативных сил нет, то есть $\dv{t} \pdv{L}{\dot{x}} = \pdv{L}{x}$.
\end{enumerate}
Тогда 
\[\pdv{L}{\dot{x}} = \alpha(x) \dot{x} \Rightarrow \alpha \ddot{x} + \alpha' \dot{x}^2 = \pdv{L}{x} = \frac{1}{2} \alpha' \dot{x}^2 - U' \Rightarrow \]
\begin{gather}
 \alpha(x) \dot{x} + \hlf \pdv{\alpha}{x} \dot{x}^2 = - \pdv{U}{x} \Rightarrow \pdv{U(x_0)}{x} =0.\\ \intertext{Используем ещё существования стационарного решения в точке $x_0$:}
 \pdv{U(x_0)}{x} = 0\\
 \intertext{Опишем формально двжиение в окрестности точки $x_0$:}
x = x_0 + q \Rightarrow \alpha(x) = \alpha(x_0) + \dots; \quad \alpha(x_0) = m\\
U(x) = U(x_0) + U'(x_0)q + \hlf U'' (x_0)q^2 + \approx \hlf k q^2; \quad k = U'' (x_0) \Rightarrow\\
 L = \hlf m \dot{q}^2 - \hlf kq^2\\
 \intertext{Возникает вопрос, что значит <<мало>>, когда говорим о малости отклонения от положения равновесия. Поэкспериментировав, можно проверить, что неважно, где делать разложение: в функции Лагранжа или в уравнениях движения. Если будем учитывать следующие поправки, то у нас буду появляться следующие поправки к силе, которая уже учтена, и они по сравнению с ней должны быть малы. Второе замечание: коэфффициенты $k$ и $q$ должны быть невырожденными, иначе должны учитывать следующие члены в разложении, но тогда колебания уже будут нелинейными. Перепишем лагранжиан в эквивалентной форме:}
L = \hlf m \dot{q}^2 - \hlf kq^2 = \frac{m}{2} (\dot{q}^2 - \omega^2 q^2), \quad \omega^2 = k/m,\\
\intertext{соответствующее уранение движения}
\ddot{q} + \omega^2 q =0\\
\intertext{ линейное ОДУ с постоянными коэффициентами, порождаемое квадратичной формой. Есть стандартный способ решения таких уравнений:}
q = Ce^{\lambda t} \Rightarrow \lambda^2 C e^{\lambda t} + \omega^2 C e^{\lambda  t} = 0\\ \Rightarrow \lambda^2 + \omega^2 = 0 \Leftrightarrow \lambda = \pm i \omega \Rightarrow\\
q = C_1 e^{i \omega t} + C_2 e^{-i \omega t}\\  
\intertext{потребуем, чтобы}
q \in \mathbb{R} \Rightarrow C_2 = C_1^* \Leftrightarrow q = C_1 e^{i \omega t} + \text{к. с.} = 2\Re C_1 e^{i\omega t} = \Re C e^{i\omega t}, \quad C \in \mathbb{C}\\
C = c e^{i \varphi} \text{ --- комплексная амплитуда} \Rightarrow q = c \cos(\omega t + \varphi)
\end{gather}
\begin{thm}
Пусть $\Hat{D}$ --- дифференциальный оператор, имеем
\[\Hat{D} q = 0, \quad \Hat{D} \in \mathbb{R},\]
тогда мы всегда можем рассмотреть некое решение $X \in \mathbb{C},\; \Hat{D} X = 0$, автоматически
\begin{equation}
\begin{cases}
\Hat{D} (\Re X) = 0\\
\Hat{D} (\Im X) = 0
\end{cases}
\end{equation}
\end{thm}
\begin{proof}
\begin{align}
\Hat{D}& (\Re X + i \Im X) = 0 \Rightarrow\\
\Hat{D}& (\Re X) + i \Hat{D} (\Im X) = 0
\end{align}
\end{proof}
\begin{gather}
q= Ce^{\lambda t}\\
\ddot{q} + \omega^2 q = Ce^{\lambda t} (\lambda^2 + \omega^2) = 0 \Rightarrow q = Ce^{i \omega t} \Rightarrow q = \Re C e^{i \omega t}\\
\omega^2 > 0\quad q = c \cos (\omega t + \varphi)  = a \sin \omega t + b \cos \omega t,\\
\intertext{ через комплексные амплитуды:}
C = c e^{i  \varphi}.
\intertext{Квадрат $\omega$ больше нуля, когда $k$ больше нуля, потому что $m$ мы получаем из законов Ньютона и оно больше нуля, а вот $k$ <<выползает>> из потенциала, и может быть меньше нуля. Положительный квадрат частоты отвечает минимуму потенциальной энергии. }
\omega^2 < 0 \quad \lambda = \pm \sqrt{\abs{\omega}} \in \mathbb{R}\\
q = c_1 e^{\lambda t} + c_2 e^{- \lambda t} = a \sh \lambda t + b \ch \lambda t \underset{H/w}{\eqq{?}} c \sh(\lambda t + \varphi) \underset{H/w}{\eqq{?}} \tilde{c} \ch (\lambda t + \varphi)\\
\omega = 0 \quad q = c_1 t + c_2 \quad \ddot{q} = 0
\end{gather}
Продемонстрируем всю мощь метода комплексных амплитуд. В этих нескольких примерах будем выходить за рамки консервативных ($L = \frac{m}{2} (\dot{q}^2 - \omega^2 q^2),\; H = \frac{m}{2}(\dot{q} + \omega^2 q^2) = const$) систем. 
\begin{ex}[Осциллятор с трением.]
\begin{gather}
\ddot{q} + \omega_0 q + 2\gamma \dot{q} = 0,\\ \intertext{то есть рассматриваем линейный осциллятор с трением.}
q = C e^{\lambda t} \Rightarrow (\underbrace{\lambda^2 + \omega_0^2 + 2\gamma \lambda}_{0}) C e^{\lambda t} = 0\\
\lambda = -\gamma \pm \sqrt{\gamma^2 - \omega_0^2} = -\gamma \pm i\sqrt{\omega_0^2 - \gamma^2},
\intertext{в комплексном виде записали, чтобы был переход к незатухающему осциллятору.}
q = c e^{-\lambda t} \cos(\sqrt{\omega_0^2 - \gamma^2}t + \varphi),\\
\intertext{получили общее решение. $H/w\; \omega_0 = \gamma,\; \omega_0 < \gamma,\; \omega_0 > \gamma\, \text{(движение в меду).}$}
\end{gather}
\end{ex}
\begin{ex}[Осциллятор, на который действует внешняя сила.]
\begin{gather}
\ddot{q} + \omega^2 q = f(t)\\
\dot{q} + i\omega t = a(t) e^{i \omega t},\label{pl_h_1}\\
 a(t) \in \mathbb{C} \Rightarrow q(t) = \frac{1}{\omega} \Im a e^{i \omega t}\\
\begin{cases}
\dv{t}\left(\dot{q} + i\omega t \right) = (\dot{a} +  i \omega a)e^{i \omega t}\\
\eqref{pl_h_1}* i\omega: \quad - i\omega \dot{q} + \omega^2 q = -i \omega a e^{i \omega t}
\end{cases}
\Rightarrow \ddot{q} + \omega^2 q = \dot{a} e^{i \omega t} = f(t) \Rightarrow a(t) = \int^t f(t) e^{-i \omega t} \dd{t}\\
\intertext{Обратим внимание, что $a(t)$ очень похоже на преобразование Фурье.}
\end{gather}
\end{ex}

\begin{thm}[Появляющаяся сила.]
Пусть в некоторый момент на осцилллятор подействовала сила с конечным спектром (см. рисунок) $\int\limits_{-\infty}^{+\infty} f e^{-i \omega t} \dd{t} = F,$ тогда $H(+\infty) - H(-\infty) = \frac{m}{2} \abs{F}^2,$\\ где~$F$~--- спектральная компонента силы.
\end{thm}
\begin{proof}
H/w
\end{proof}
\begin{task}
\begin{gather}
\ddot{q} + 2\gamma \dot{q} + \omega_0^2 q = f(t)
\end{gather}
В частности, когда сила сама осциллирует: $f (t) = A \cos \omega t \Rightarrow q(t) = ?$.
\end{task}

\subsubsection{Многомерные системы}
\begin{gather}
L = \sum_{i, j} \hlf \alpha_{ij}(x) \dot{x}_i \dot{x}_j + \sum_i \beta_i(x)\dot{x}_i - U(x)
\end{gather}
Построим уравнение движения. Скажем, что $x$ --- тождественная константа --- отвечает случаю локального экстремума функции $U(x)$:
\[x = x_0 \equiv const \Leftrightarrow \pdv{U(x_0)}{x_i} = 0, \quad \forall i =\overline{1,s}\]
\begin{align}
x = x_0 + q \Rightarrow\; & \alpha_{ij}(x) \approx m_{ij} = \alpha_{ij} (x_0) \\
&U(x) \approx \sum_{i, j} \hlf k_{ij} q_i q_j; \quad k_{ij} = \pdv{U(x_0)}{x_i}{x_j}
\end{align}
Что можем сказать про коэффициенты $m_{ij}, k_{ij}$?
\begin{enumerate}
\item $m_{ij}$ симметричная ($m_{ij} = m{ji}$) и положительно определённая, $\alpha$ --- положительно определённая квадратичная форма, потому что произошла из кинетической энергии, и матрица постоянных коэффициентов $m$ унаследовала эти свойства.
\item $k_{ij} = k_{ji}:$ появилась по определению как смешанная производная, положительная определённость не гарантируется (достигается в случае локального минимума потенциала). 
\end{enumerate}
Осталось рассмотреть гироскопические силы. К полной производной, как в одномерном случае они сводиться не обязаны. Заметим, что 
\begin{gather}
\sum_i \beta_i(x_0) \dot{x}_i = \dv{t} \left(\sum \beta_i (x_0) x_i\right),\; \text{поэтому}\\
\beta_i(x) \approx \not{\beta_i(x_0)} + \sum_j \pdv{\beta_i(x_0)}{x_j} q_j; \quad g_{ij} = \pdv{\beta_i(x_0)}{x_j},
\end{gather}
подставим это всё в лагранжиан:
\begin{equation}
\boxed{L = \sum_{i, j = 1}^s \left\lbrace \hlf m_{ij} \dot{q}_i \dot{q}_j + g_{ij} q_j \dot{q}_i - \hlf k_{ij} q_i q_j \right\rbrace}. \label{Lagr_osc}
\end{equation}
\begin{ex}[$g_{ij} = 0$]
Рассмотрим лагранжиан \eqref{Lagr_osc} в частном случае, когда нет гиротропных сил, то есть $g_{ij} = 0$:
\begin{equation}
L = \sum_{i, j = 1} \left\lbrace \hlf m_{ij} \dot{q}_i \dot{q}_j  - \hlf k_{ij} q_i q_j \right\rbrace.
\end{equation}
Его можно рассматривать как две квадратичные формы, соответствующие двум слагаемым: первая симметричная и положительно определённая, вторая симметричная. Теорема из линейной алгебры утверждает, что такие кв. формы диагонализируемы одновременно.
\begin{thm}
$$
\left\{
\begin{array}{rcl}
\text{симм.$(+)$}\\
\text{симм.}
\end{array}
\right.
\Rightarrow \text{всегда диагонализуемы одновременно!}
$$
То есть $\exists\; \text{линейное преобразование}\; a_{ik}\; |\; q_i = \sum_i a_{ik} \theta_k, \dot{q}_i = \sum a_{ik} \dot{\theta}_k.$
\end{thm}
 тогда
\begin{equation}
L = \sum^s_k \lbrace \hlf m_k \dot{\theta}_k^2 - \hlf k_k \theta_k^2 \rbrace = \sum_k \frac{m_k}{2} \lbrace \dot{\theta}_k^2 - \omega^2_k \theta_k^2 \rbrace, \label{Lagr_ex_osc}
\end{equation}
где $\omega_k^2 = k_k / m_k,$ то есть система распадается на $s$ штук невзаимодействующих подсистем, каждая из которых есть  одномерный гармонический осциллятор.
Уравнение движения соответствующее \eqref{Lagr_ex_osc}:
\begin{gather}
\ddot{\theta}_k + \omega^2_k \theta_k = 0 \Rightarrow\\
\theta(t) = C_k \cos (\omega t + \varphi_k)  \approx \Re C_k e^{i \omega_ kt}
\end{gather}
\begin{dfn}
$\{\omega_k\}$ --- спектр нормальных частот $\omega_k,\; k = \overline{1, s}.$
\end{dfn}
\begin{dfn}
$\{\theta_k\}$ --- нормальные координаты.
\end{dfn}
Как выглядит решение? 
\begin{dfn}
Частное решение при $\theta_k = 0,$ кроме $k = k^* \Rightarrow$
\begin{equation}
q_j = a_{jk^*} \theta_{k^*}(t) 
\end{equation}
называют нормальными колебаниями.
\end{dfn}
Общее решение:
\begin{equation}
q_j (t) = \sum_{k=1}^s a_{jk} \theta_k (t).
\end{equation}
\begin{rmk}
Если мы возбудили одно нормальное колебание, то каждая степень свободы колеблется в одной и той же фазе. То есть, если у нас есть сложная многомерная система, и одна степень свободы проходит через ноль или экстремум, то остальные степени свободы тоже проходят через ноль или экстремум соответственно.
\end{rmk}
\begin{rmk}
Если мы живём на дне потенциального рельефа, то в \eqref{Lagr_ex_osc} две положительно определённые квадратичные формы, значит, $\omega_k^2 > 0\; \forall k,$ и у нас действительно колебания, то есть можем получить решения в виде синусов и косинусов, а не только экспонент.
\end{rmk}
\end{ex}
Построим уравнение движения для лагранжиана \eqref{Lagr_ex_osc}:
\begin{gather}
\pdv{L}{\dot{q}_k} =  \{ \frac{1}{2} m_{ik} \dot{q}_i + \frac{1}{2} m_{kj} \dot{q}_j + g_{kj}q_j \} = \sum_i \{ m_{ik} \dot{q}_i + g_{ki} q_i\}, \label{pdv_L_dotq_osc}\\
\pdv{L}{q_k} = \sum \{ g_{ik} \dot{q}_i - k_{ik} q_i \} \stackrel{!!!!!}{\Rightarrow} \label{pdv_L_qk}\\
\dv{t}\pdv{L}{\dot{q}_j} - \pdv{L}{q_j} = 0 \Rightarrow \sum_i \{ m_{ij} \ddot{q}_i + (g_{ij} - g_{ji})\dot{q}_i + k_{ij} q_i \} = 0.
\end{gather}
\begin{rmk}
$G_{ij} = -G_{ji}.$
\end{rmk}
Ищем решение в виде
\begin{equation}
q_i = \Re C_i e^{\lambda t}, 
\end{equation}
тогда
\begin{gather}
\Re \sum_i \{m_{ij} \lambda^2 + G_{ij} \lambda + k_{ij} \} C_i e^{\lambda t} = 0 \Leftrightarrow\\
\sum_i \{m_{ij} \lambda^2 + G_{ij} \lambda + k_{ij} \} C_i = 0. \label{eq_lambda}\\
\intertext{Эта линейная однородная алгебраическая система имеет невырожденное решение, когда детерминант матрицы её коэффициентов равен нулю:}
\det \left( m_{ij} \lambda^2 + G_{ij} \lambda + k_{ij} \right) = 0\; \text{--- характеристическое уравнение.}
\end{gather}
Поразмышляем о структуре решения. По размерности $P_{2s} (\lambda) =0,$ плюс, если $\lambda$ --- корень, то $\lambda^*$~--- тоже корень, так как $P_{2s} \in \mathbb{R},$ то есть все коэффициенты этого полинома действительные. На самом деле, уравнение движения консервативной системы накладывает ещё одно ограничение, и если расписать детерминант, то  можно получить, что решение имеет вид
$P_s (\lambda^2) = 0.$ Свойство чётности степеней --- свойство обратимости времени. Покажем, что решения действительно идут парами. Пусть $\lambda$ --- корень исходного характеристического уравнения, рассмотрим это же уравнение относительно $-\lambda$:
\begin{gather}
\det \left(m_{ij} (-\lambda)^2 +  G_{ij}(- \lambda) + k_{ij} \right) \Leftrightarrow\\
\det \left(m_{ij} \lambda^2 +  G_{ji} \lambda + k_{ij} \right) \Leftrightarrow\\
\det \left(m_{ji} \lambda^2 +  G_{ji} \lambda + k_{ji} \right) \label{new_det},\\
\intertext{поскольку $m_{ji}$ и $k_{ji}$ симметричные, и мы получили уравнение, выполняющееся тождественно, потому что $\lambda$ --- корень --- свойство антисимметричности члена, отвечающшего за гиротропию. А это означает, что $-\lambda$ --- тоже корень, что в точности и означает, что характеристическое уравнение имеет вид}
P_s(\lambda^2) =0.
\end{gather}
Для консервативной системы каждый корень порождает ещё три:
\begin{equation}
\lambda \longrightarrow \lambda^*, -\lambda, -\lambda^*.
\end{equation}

Поразмышляем, при каких условиях реализуются устойчивые колебания, а не какие-то экспоненты, описывающие неустойчивые состояния равновесия. Вернёмся к \eqref{eq_lambda}. Для анализа таких уравнений существует стандартный приём: умножим каждое уравнение на $C_j^*$, учтём, что
\begin{gather}
C_i C_j^* = (c_i' + i c_i'')(c_j' - i c_j'') = (\underbrace{c_i' c_j' + c_i''c_j''}_{S_{ij}}) + i(\underbrace{c_i'' c_j' - c_i' c_j''}_{A_{ij}}),\\
\intertext{•:}
\sum_{i, j} \boxed{m_{ij} S_{ij} \lambda^2 + iG_{ij} A_{ij} \lambda + k_{ij} S_{ij} = 0}
\end{gather}
\begin{enumerate}
\item гиротропии нет $k_{ij} (+); G_{iJ} = 0 \Rightarrow \lambda^2 = - \frac{k_{ij} S_{ij}}{m_{ij} S_{iJ}} < 0 \Rightarrow \lambda = \pm i\omega$ --- ситуация, когда существуют нормальные частоты и нормальные колебания в смысле именно колебаний.
\item Нет $(+) k_{ij} \Rightarrow \lambda^2 > 0 \Rightarrow \pm \lambda \Rightarrow c_1 e^{\lambda t} + c_2 e^{-\lambda t}$ --- состояние равновесия типа седло.
\item Можно показать, что гиротропия не может разрушить устойчивое состояние равновесия. $k_{ij}(+) \& G_{iо} \neq 0 \Rightarrow \lambda^2 < 0,$ то есть колебания устойчивые.
\end{enumerate}

Допустим, что мы научились решать характеристическое уравнение. Получим общее решение уравнения движения лагранжевой системы вблизи положения равновесия.
\begin{align}
P_s (\lambda^2) = 0 \Rightarrow \{&\lambda^2_k\} k=\overline{1, s}\\
&\lambda^2_k < 0 \Rightarrow \lambda_k = \pm i\omega_k\\
q_j^{(k)} &= \Re C_{jk} e^{i\omega_k t}\\
\end{align}
\begin{align}
\sum_{i=1}^s \left( m_{ij} \lambda^2 + G_{ij} \lambda + k_{ij} \right) C_i = 0 \quad j=\overline{1, s} \Rightarrow C_{jk} = a_{jk} e^{i \varphi_{jk}} &B_k\\
&\forall B_k = b_k e^{i\varphi_{0_k}}
\end{align}
Строим общее решение для $q,$ которое есть сумма всех нормальных колебаний:
\begin{align}
q_ j= \sum_{k=1}^s q_j^{(k)} &= \sum_k \Re b_k a_{jk} e^{i\omega_k t + i\varphi_{jk} + i \varphi_{0_k}}\\
\intertext{$\varphi_{jk} = 0,$ если $G = 0$ (нет гиротропии), тогда}
q_j &= \sum_k a_{jk} \underbrace{\Re B_k e^{i \omega_k t + i \varphi_{0_k}}}_{\theta_k (t)},\\
\intertext{то есть свели ответ к предыдущему.}
\end{align}
Вообще,
\begin{equation}
q_j = \sum_k \Re \left\{B_k C_{jk} e^{i\omega_k t} \right\}.
\end{equation}
Рассмотрим пару простых примеров.
\begin{ex}[Чашечка]
Пусть у нас есть движение в поле тяжести в окрестности минимума какой-то ямки $z = h(x, y).$ Заметим, что если $x =y = 0$ отвечают $\min h(x, y),$ то 
\begin{gather}
z = h(x, y) = \frac{x^2}{2\rho_1^2} + \frac{y^2}{2\rho_2^2} + \dots \quad \rho_{1, 2}\;\text{--- главные радиусы кривизны.}\\
\intertext{Составим лагранжиан:}
L = T - U = \frac{m}{2} \left(\dot{x}^2 + \dot{y}^2 + \dot{z}^2\right) - mg\left( \frac{x^2}{2\rho_1^2} + \frac{y^2}{2\rho_2^2}\right), \label{lagrangian_yamka}\\
\intertext{$\dot{z}^2$ нас не интересует, потому что речь идёт о малых колебаниях, поэтому \eqref{lagrangian_yamka} можно переписать в виде}
L = \frac{m}{2} \left\{\dot{x}^2 - \Omega_1^2 x^2 + \dot{y}^2 - \Omega_2^2 y^2 \right\}, \quad \Omega_{1, 2} = \frac{g}{\rho_{1, 2}^2}.\\
x = a \cos (\Omega_1 t + \varphi_1),\\
y = b \cos (\Omega_2 t + \varphi_2), \text{и эти колебания независимые.}
\end{gather}
\end{ex}
\begin{ex}[Вращающаяся чашечка]
Перейдём в систему координат $x, y$, которая прибита к чашке, и в ней уравнение чашки не изменится, но при этом в подвижной системе координат появятся дополнительные члены, связанные с вращением:
\begin{gather}
\vb{v}_{co} = [\vb*{\Omega}, \vb{r}] \Rightarrow\\
\begin{cases}
v_x = \dot{x} - \Omega y\\
v_y = \dot{y} + \Omega x
\end{cases}
\Rightarrow L = \frac{m}{2} \left\{(\dot{x} - \Omega y)^2 + (\dot{y} + \Omega x)^2 - \Omega_1^2 x^2 - \Omega_2^2 y^2 \right\},
\intertext{этот лагранжиан квадратичен по всем координатам и скоростям, и он содержит гироскопически члены (вида произведение координаты на скорость), а мы его перепишем:}
L = \frac{m}{2} \big\{ \dot{x}^2 + \dot{y}^2 + 2\Omega (x\dot{y} - y\dot{x}) - (\underbrace{\Omega_1^2 - \Omega^2}_{\widetilde{\Omega}_1^2})x^2 - (\underbrace{\Omega_2^2 - \Omega^2}_{\widetilde{\Omega}_2^2})y^2\big\}.\\
\begin{rcases}
\dv{t}\pdv{L}{\dot{x}}= m\ddot{x} - m\Omega\dot{y} = \pdv{L}{x} = m\Omega \dot{y} - m\widetilde{\Omega}_1^2 x\\
\dv{t}\pdv{L}{\dot{y}} = m\ddot{y} + m\Omega\dot{x} = \pdv{L}{y} = - m\Omega \dot{x} - m \widetilde{\Omega}_2^2 y
\end{rcases}
\Rightarrow\\
\begin{cases}
\ddot{x} - 2\Omega \dot{y} + \widetilde{\Omega}_1^2 x = 0\\
\ddot{y} + 2\Omega \dot{x} +  \widetilde{\Omega}_2^2 y =0
\end{cases}\\
x = C_1 e^{i \Omega t}\\
y = C_2 e^{i\Omega t}\\
\begin{cases}
\left(-\omega^{2}+\widetilde{\Omega}_1^2\right) C_{1}-2 \Omega i \omega C_{2}=0\\
2 \Omega i \omega  C_{1}+\left(-\omega^{2} + \widetilde{\Omega}_2^2 \right) C_{2}=0, \label{C1_C2_sys}
\end{cases}
\intertext{система \eqref{C1_C2_sys} имеет нетривиальное решение, когда определитель матрицы коэффициентов перед искомыми $C_1$ и $C_2$ равен нулю, тогда}
\boxed{ \left(\omega^{2}-\widetilde{\Omega}_{1}^{2}\right)\left(\omega^{2}-\widetilde{\Omega}_{2}^{2}\right)=4 \Omega^{2} \omega^{2}}.
\end{gather}
 Получили биквадратное уравнение относительно нормальных частот $\omega$. Проанализируем случай, когда
 \[\omega \ll \widetilde{\Omega}_1, \widetilde{\Omega}_2.\]
\end{ex}

До сих пор у нас был консервативный случай, и обобщённая энергия сохранялась:
\[H = const \quad H = \sum \frac{1}{2} \left\{m_{iJ} \dot{q}_i \dot{q}_j + k_{ij} q_i q_j\right\}.\]
\[H/w \quad \dv{H}{t} = 0 \Leftrightarrow P_s(\lambda^2) = 0\; \text{или}\; \lambda\; \text{и}\; -\lambda\; \text{--- корни одновременно.} \]

\subsubsection{Малые колебания в диссипативных системах}
\paragraph{Диссипативная функция Рэлея.} \index{Диссипативная функция Рэлея}
Линейное трение в лагранжевых системах обычно вводится следующим образом:
\[\vb{F}_i = -\sum_j \mu_{ij} \vb{v}_j,\]
и такие силы можно пересчитать в обобщённые силы, которые войдут в уравнение Лагранжа, с помощью этакого потенциала в пространстве скоростей, с помощью функции Рэлея $R(t, \{\vb{r}_i\}, \{\vb{v}_i\},$ например, такой функции:
\[\vb{F}_i = -\sum_j \mu_{ij} \vb{v}_j = - \pdv{R}{\vb{r}_i} \quad R = \frac{1}{2} \sum_{i, j} \mu_{ij} \vb{v}_i \vb{v}_j = \frac{1}{2} \gamma_{iJ} \dot{q}_i \dot{q}_j,\]
и в этом случае
\[Q_j = \sum_{i=1}^N \vb{F}_i \pdv{\vb{r}_i}{q_j} = - \sum_{i=1}^N \pdv{R}{\vb{v}_i} \pdv{\vb{v}_i}{\dot{q}_j} = - \pdv{R}{\dot{q}_j} = - \sum_{i=1}^n \gamma_{ij} \dot{q}_i,\]
и отличие от гироскопических сил только в том, что матрица коэффициентов здесь симметричная (по построению):
\[\gamma_{ij} = \gamma_{ji}.\]
А уравнение Лагранжа выглядит следующим образом:
\[\boxed{\frac{d}{d t} \frac{\partial L}{\partial \dot{q}_{j}}-\frac{\partial L}{\partial q_{j}}+\frac{\partial R}{\partial \dot{q}_{j}}=0}.\]
\[H/w \Rightarrow \sum_{j=1}^{s}\left\{m_{i j} \dot{q}_{j}+\left(G_{i j}+\gamma_{i j}\right) \dot{q}_{j}k_{ij} q_j\right\} = 0, \]
причём $G_{ij}$ --- антисимметричная часть (порождается функцией Лагранжа), гарантирует,\\ 
что $P_s (\lambda^2) =0;\; \dv{H}{t} =0;$ $\lambda_{ij}$ --- симметричная часть (порождается функцией Рэлея, и $P_{2s}(\lambda) =0$ --- есть нечётные степени, диссипация, и направления времени не эквиваленты, диссипация работает в обе стороны, система не может двигаться <<по кругу>>, $\dv{H}{t} = \sum Q_j \dot{q}_j = - \sum \pdv{R}{\dot{q}_j} \dot{q}_j = -2R,$ то есть физический смысл функции Рэлея в том, что она отвечает мощности потерь на соответствующей силе, которую она определяет, и в этом случае мы будем получать решения, как для осциллятора с трением, в виде
\begin{gather}
e^{\lambda t};\; \lambda = \lambda' + i\lambda'' \Rightarrow\\
e^{\lambda' t} \cos (\lambda'' t + \varphi),
\end{gather}
действительная часть корня характеристического уравнения описывает затухание в случае диссипации, а мнимая часть --- действительную часть частоты, свойство одновременной принадлежности к корням $\lambda$ и $\lambda^*$ сохраняется (потому что это свойство действительности коэффициентов), а $\lambda$ и $-\lambda$ --- нет.
\subsection{Вариационная форма механики Лагранжа}
\subsubsection{Введение в принцип наименьшего действия}
Раньше, получая уравнения Ньютона, мы исходили из принципа малых шажков. Математически это выражалось тем, что мы рассматривали дифференциальные уравнения Ньютона, и получали решения (уравнения движения как бесконечную сумму бесконечно малых шажков). Введение принципа уравнения Лагранжа и понятия идеальных связей позволили нам исключить из уравнений Ньютона силы реакции связей, которые мы не знаем, и мы получили уравнения Лагранжа второго рода. Это был дифференциальный подход.

Вариационная формулировка подразумевает рассмотрение траекторий как некое целое --- "интегральный подход".

\[L(t, q, \dot{q}, q = (q_1, \dots, q_s), s= 3N-k,\]
и нет непотенциальных обобщённых сил
\[Q^{\text{НП}} =0, \]
то есть наша система полностью описывается уравнением
\[\left(\frac{d}{d t} \frac{\partial L}{\partial \dot{q}_{j}} - \frac{\partial L}{\partial q_{j}}=0\right).\]
\begin{dfn}
$\{q\}$ --- конфигурационное пространство.
\end{dfn}
Линия $q(t)$ в конфигурационном пространстве --- то, что мы ищем --- траектория системы, эволюция её состояний во времени. Принцип наименьшего действия позволяет отсортировать настоящие и ненастоящие траектории.  Истинная траектория --- <<прямой путь>>.
\begin{gather}
q_1 = q(t_1),\\
q_2 = q(t_2).
\end{gather}
Как отличить истинную траекторию, которая отвечает уравнения Лагранжа, от всех остальных? В фазовом пространстве траектории не пересекаются нигде, кроме особых точек, пересечения (самопересечения) кривых в конфигурационном пространстве ничему не противоречат.

\subsubsection{Вариационный принцип Гамильтона для обобщенно-потенциальных систем}
Давайте каждой из траекторий по какому-то закону припишем число, а потом скажем, что истинной траектории отвечает конкретное число.

Отображение функций в числа \[q(t) \stackrel{S}{\rightarrow} \mathbb{R}\] называют функционалом.Чтобы не путать с функциями, пишут аргумент в квадратных скобках
\[S[q(t)] \rightarrow \mathbb{R}.\]

Длина кривой не позволяет выделять истинные траектории. Рассмотрим функционал
\begin{equation}
 \boxed{S[q(t)] = \int\limits L(t, q, \dot{q}(t)) \dd{t}}, \label{deystv}
 \end{equation}
 этот функционал называют функционалом \index{Функционал действия} действия (иногда просто действием). Конфигурационное пространство является общим для большого семейства систем с разными лагранжианами, но общими обобщёнными координатами.
 
 \begin{pst}
 Пусть
 \begin{enumerate}
 \item $q(t_1) = q_1, q(t_2) = q_2,$
 \item $\exists M:\; \abs{q_1 - q_2} < M$, 
 \end{enumerate}
 тогда $q(t),$ удовлетворяющее уравнению движения, отвечает наименьшему значению функционала действия $S[q(t)] \rightarrow min$ --- аналог того, что первая производная равна нулю, вторая производная знакоопределена. 
 \end{pst}
 
 \begin{pst}[Вариационный принцип Гамильтона]
 Пусть $q(t_1) = q_1, q(t_2) = q_2,$ тогда между этими положениями система движется так, что ФЛ принимает стационарные значения $S[q(t)] \rightarrow stat$ --- аналог того, что первая производная равна нулю.
 \end{pst}

Давайте рассмотрим малое возмущение (бесконечно близкую траекторию с невозмущённым  концами) $q(t) + \delta q(t),$ где $\delta q(t_1) =0,\; \delta q(t_2) = 0,\; \forall \delta q(t)$.
\begin{gather}
S[q(t)+\delta q(t)]-S[q(t)] \geqslant 0
\end{gather}
Мы не умеем находить экстремумы в пространстве функций, но умеем в пространстве чисел. Предположим, что
\begin{gather}
\delta q(t) = \alpha \cdot h(t),\\
h(t_1) = h(t_2) = 0,
\end{gather}
введём понятие 
\begin{equation}
S(\alpha) = S[q(t) + \alpha \cdot h(t) ] \geqslant,
\end{equation}
последнее неравенство ... то есть переформулировали ПНД в терминах задачи на экстремум в обычных переменных. На физическом уровне строгости напишем необходимое условие существования экстремума в точке $\alpha = 0$
\begin{gather}
S(\alpha) \rightarrow \pdv{S}{\alpha} = 0\; |_{\alpha = 0}.
\end{gather}
Распишем
\begin{gather}
\pdv{\alpha} \int\limits_{t_1}^{t_2} L (t, q + \alpha h, \dot{q} + \alpha \dot{h}) \dd{t} = \int\limits_{t_1}^{t_2} \left\{ \pdv{L}{q} h + \pdv{L}{\dot{q}}\dot{h} \right\}_{|_{\alpha = 0}} \dd{t} = \\
= \int \limits_{t_1}^{t_2} \left\{ \pdv{L}{q} - \dv{t} \left(\pdv{L}{\dot{q}}\right) \right\} h(t) \dd{t} +  \underbrace{\left. \pdv{L}{\dot{q}}\, h \right| _{t_1}^{t_2}}_{=0} = \\
=\int\limits_{t_1}^{t_2} \sum_{j=1}^{s}\left\{\frac{\partial L}{\partial q_{j}}-\dv{t} \left(\frac{\partial L}{\partial \dot{q}_j}\right)\right\} h_{j}(t) d t=0 \quad \text{по ПНД для $\forall h_j (t)$.} \Rightarrow\\
\forall\, j =1,s \quad \boxed{\dv{t} \left(\pdv{L}{\dot{q}_j} \right) - \pdv{L}{q_j}=0},
\end{gather}
то есть мы получили, что принцип наименьшего действия гарантирует, что движение по истинным траекториям удовлетворяет уравнению Лагранжа (или, что то же самое, что необходимое условие экстремума --- в точности то же, что уравнение Лагранжа).

Займёмся тем же, что проделали только что, однако не переходя к дифференцированию в числах,
\begin{equation}
S[q + \delta q] - S[q] = \int \limits_{t_1}^{t_2} \left\{ L(t, q + \delta q, \dot{q} + \delta \dot{q}) - L(t, q, \delta{q}) \right\} \dd{t},
\end{equation}
зафиксируем момент времени и применим разложение в ряд Тейлора, тогда
\begin{gather}
S[q + \delta q] - S[q] = \int\limits_{t_1}^{t_2} \left\{\pdv{L}{q} \delta q + \pdv{L}{\dot{q}} \delta \dot{q} + \ldots \right\} \dd{t} =,\\
\intertext{и проинтегрируем по частям}
= \int\limits_{t_1}^{t_2} \left\{ \pdv{L}{q} - \dv{t} \pdv{L}{\dot{q}} \right\} \delta q \dd{t} + \ldots .
\end{gather}
Написанное нами слагаемое называют \textit{первой вариацией функционала действия,} \index{Вариация} обозначают как $\var{S[q, \delta q]}$.


Согласно ВПГ
\begin{equation}
\var{S} = 0  \Longleftrightarrow \; \text{ур-ю Лагранжа II рода при фиксированных концах траектории.}
\end{equation}


А ПНД говорит, что
\begin{equation}
\text{ур-e Лагранжа II рода} \Longrightarrow \begin{cases}
\var S = 0,\\
\var^2{S} \geqslant 0 
\end{cases}
\text{при фиксированных концах, близости}\; q_1, q_2.
\end{equation}

\begin{rmk}[Обобщение на случай систем, не являющихся обобщённо-потенциальными]
Хотим, чтобы первая вариация приводила к уравнению $\frac{d}{d t} \frac{\partial T}{\partial \dot{q}}-\frac{\partial T}{\partial q}=Q$. Первые два слагаемые получаются из кинетической энергии $T(t, q, \dot{q}),$ второе --- из $\var{S} = \int\limits_{t_1}^{t_2} Q \dd{t},$ при этом \[Q = \sum_{j=1}^s Q_j \delta q_j = \sum_{i=1}^N \vb{F}_i^{(a)} \delta \vb{r}_i = \delta A^{(a)},\] значит,
\begin{equation}
S[q(t)] = \int\limits_{t_1}^{t_2} \left(T + A^{(a)}\right) \dd{t}.
\end{equation}

Частный случай, когда мы имеем дело с обобщённо-потенциальными силами в натуральных системах, в качестве работы активных сил может выступать обобщённый потенциал, то есть
\[S = \int\limits_{t_1}^{t_2} (T- U) \dd{t}.\] \textit{H/w показать, что, варьируя такой функционал, получим то же, что при варьировании такого функционала с функцией с Лагранжа в качестве аргумента.}
\end{rmk}

\subsubsection{Вариационный принцип для гармонического осциллятора}
Почему выбрали именно гармонический осциллятор? Задача на равенство нулю первой производной гораздо проще задачи на определение знака второй производной...

Пусть у нас одномерная система, которая задана лагранжианом
\[L = \frac{1}{2} (\dot{q}^2 - \omega^2 q.\]
Для это этой системы мы всё знаем:
\begin{gather}
\ddot{q} = -\omega^2 q,\\
q = a\sin (\omega t + \varphi).
\end{gather}
Рассмотрим малое возмущение $q+\delta q,$ посчитаем
\begin{equation}
S[q+\delta q] - S[q]. \label{functional_1}
\end{equation}
Будем считать $t_1 = 0,$ $t_2 = \tau,$ и если $\tau \neq n T/2,$ то по $q_1$ и $q_2,$ соответствующим заданным моментам времени мы траекторию однозначно определяем, иначе, при времени $\tau = n T/2$ ($T = \frac{2\pi}{\omega}$), называемом \textit{кинетическим фокусом} \index{Кинетический фокус},  получим бесконечно много решений, и все они будут удовлетворять уравнению движения. 
\begin{dfn}
Кинетический фокус сопряжённый некой начальной точке --- точка, в которую сходятся  две любые бесконечно близкие (пущенные с разными скоростями) траектории.
\end{dfn}

\begin{ex}[Кинетический фокус на сфере]
На глобусе между двумя точками есть наикратчайшее расстояние, но если мы рассмотрим две точки на диаметрально противоположные точки , то таких траекторий с наименьшей длиной будет бесконечно много, и диаметрально противоположная точка будет кинетическим фокусом.
\end{ex}

Нам надо определить
\begin{equation}
\begin{cases}
\delta q(0) =0\\
\delta q(\tau) = 0,
\forall \delta q(t)
\end{cases}
\Rightarrow \delta q(t) = \sum_{n=1}^\infty a_n \sin \left(n \frac{\pi}{\tau} t \right), \; \forall a_n \in \mathbb{R}.
\end{equation}
Вернёмся к \eqref{functional_1}
\begin{gather}
S[q+\delta q] - S[q] = \int\limits_0^\tau \frac{1}{2} \left( (\dot{q} + \delta \dot{q})^2 - \omega^2 (q +\delta q)^2  - \dot{q}^2 + \omega^2 q\right) \dd{t} =\\
=\int\limits_0^\tau \frac{1}{2} \left\lbrace \underbrace{2\dot{q} \delta \dot{q} - 2\omega^2 q \delta q}_{\dot{q}\delta \dot{q} + \ddot{q} \delta q = \dv{t}(\dot{q} \delta q)} + \delta \dot{q}^2 - \omega^2 \delta q^2 \right\rbrace \dd{t} =\\
= \int\limits_0^\tau \frac{1}{2} \sum_{n=1}^\infty a_n^2  \left\lbrace (n \frac{\pi}{\tau})^2 \cos^2 (n \frac{\pi}{\tau} t) - \omega^2 \sin^2 (\frac{n \pi}{\tau} t) \right\rbrace \dd{t} =\\
= \frac{\pi}{4} \sum_{n=1}^\infty a_n^2 (n^2 \Omega^2 - \omega^2), 
 \end{gather}
 и вся эта штука отвечает минимуму, если $\Omega > \omega,$ что равносильно тому, что $\tau < T/2.$ То есть, если траектория не содержит кинетического фокуса, то реализуется принцип наименьшего действия\footnote{Подробнее, но без доказательств, см. \cite{Atherman}, \S 5, Гл. 7.}.

Повторим общее утверждение. Если на действительной траектории не лежит кинетический фокус исходной точки, то функционал действия отвечает наибольшему (или наименьшему\footnote{Зависит от того, какой знак мы поставили перед лагранжианом.}) значению.

Если мы рассматриваем траекторию с кинетическим фокусом, то её можно разбить на кусочки без кинетических фокусов, и на каждом куске будет реализован минимум.

Если проводить вариацию только по траекториям, проходящим через кинетический фокус, то будет принцип наименьшего действия.

\subsubsection{Следствия вариационного принципа}
Чем же так хорош вариационный принцип?
У вариационного принципа есть два аспекта, которые сделали его в некоторой мере центральным для современной теоретической физики.

Сформулировав вариационный принцип, мы сумели абстрагироваться от системы координат --- технической вспомогательной конструкции, которая нужна уже при решении уравнений движения. Неизменность уравнений Лагранжа при изменении координат --- следствие вариационного принципа.

\begin{enumerate}
\item Пусть у нас есть $L(t, q, \dot{q})$, рассмотрим $L' = L + \dv \Phi (t, q)$. $L$ порождает функционал действия
\[S[q] = \int_{t_1}^{t_2} L \dd{t} \rightarrow \var{S [q, \delta q]} = 0 \Leftrightarrow \text{ур. Лагранжа на $q(t).$}\]
А теперь для $L'$
\[S'[q] = \int_{t_1}^{t_2} L' \dd{t} = S[q] + \left. \Phi (t, q) \right|_{t_1}^{t_2} \rightarrow \var{S'} = \var{S},\]
поскольку последняя подстановка не даёт вклада в вариацию, и вариации зануляются на одних и тех же значения $q$ и $t$.

\item Рассмотрим преобразование координат в уравнении Лагранжа.
\begin{gather}
L(t, q, \dot{q}) \rightarrow\\
\dv{t} \frac{\partial L}{\partial \dot{q}}-\frac{\partial L}{\partial q}=0 \stackrel{q^* = \varphi(t, q)}{\longrightarrow} \text{как будет выглядеть уравнение $q^* (t)$?}\\
\left(\dv{\tilde{\varphi}}{t} = \pdv{\tilde{\varphi}}{t} + \pdv{\tilde{\varphi}}{q^*}\dot{q}^*, \quad q = \tilde{\varphi}(t, q^*) \right)\\
S[q] = \int_{t_1}^{t_2} L \dd{t} \equiv \int_{t_1}^{t_2} \underbrace{L \left(t, \tilde{\varphi}(t, q^*), \dv{\tilde{\varphi}}{t}\right)}_{L^* (t, q^*, \dot{q}^*)} \dd{t} = S^* [q^* (t)]\\
\delta S[q] = \delta S^* [q^*] =0
\end{gather}
\begin{rmk}
Если система была натуральной $L = L_0 + L_1 + L_2,$ то она останется натуральной при замене координат.
\end{rmk}

\item Замена координат и времени. Была Лагранжева задача
\begin{equation}
L(t, q, \dot{q}) \rightarrow \text{ур. Лагранжа} \rightarrow \begin{cases}
q^* = \varphi(t, q)\\
t^* = \psi(t, q)
\end{cases}
\rightarrow \text{? ур-я $q^*(t^*)$}
\end{equation}
\begin{equation}
\begin{cases}
q=\tilde{\varphi}\left(t^{*}, q^{*}\right)\\
t = \widetilde{\psi}(t^*, q^*)
\end{cases}
\end{equation}

\begin{equation}
S[q(t)] = \int_{t_1}^{t_2} L(t, q, \dot{q}) \dd{t} = \int_{t_1^*}^{t_2^*} \underbrace{L\left(\widetilde{\psi}, \tilde{\varphi}, \dv{\tilde{\varphi}}{\widetilde{\psi}}\right) \dv{\widetilde{\psi}}{t^*}}_{L^* \left(t^*, q^*, \dv{q^*}{t^*}\right)} \dd{t^*} = S^* [q^*(t^*)],
\end{equation}
но этот функционал тождественно связан с исходным, поэтому
\[\var{S} \equiv \var{S^*},\]
следовательно, $L^{*}$ порождает уравнения Лагранжа, которые в точности равны исходным, в которых произведена замена <<$*$>>.
\begin{equation}
\begin{rcases}
\dd{q}=\pdv{\tilde{\varphi}}{t^*} \dd{t^{*}}+\pdv{\tilde{\varphi}}{q^*}  \dd{q^{*}}\\
\dd{t} = \pdv{\widetilde{\psi}}{t^*} \dd{t^*} + \pdv{\widetilde{\psi}}{q^*}  \dd{q^{*}}
\end{rcases}
\Rightarrow
L^{*} = L \left( \widetilde{\psi}, \tilde{\varphi}, \frac{\pdv{\tilde{\varphi}}{t^*} +\pdv{\tilde{\varphi}}{q^*} \dv{q^*}{t^*}}{\pdv{\widetilde{\psi}}{t^*} + \pdv{\widetilde{\psi}}{q^*} \dv{q^*}{t^*}}\right) \cdot \Bigg(\underbrace{\pdv{\widetilde{\psi}}{t^*} + \pdv{\widetilde{\psi}}{q^*} \dv{q^*}{t^*}}_{\dv{t}{t^*}} \Bigg) \label{monster}
\end{equation}
\begin{rmk}
Система не остаётся натуральной!
\end{rmk}
\end{enumerate}

\begin{ex}
\begin{multicols}{2}
Натуральная система (Ньютоновская механика).
\[\vb{p}= m \vb{v} = \pdv{L}{\vb{v}} \Rightarrow L = \frac{mv^2}{2}.\]
\begin{equation}
\begin{rcases}
H = \vb{p} \cdot \vb{v} -L = L\\
H = T 
\end{rcases}
\Rightarrow L = T\; (\text{натуральность})
\end{equation}

Не натуральная система (релятивистская механика). $H/w$
\begin{gather}
\vb{p} = \frac{m\vb{v}}{\sqrt{1- v^2/c^2}} = \pdv{L}{\vb{v}} \Rightarrow L = - mc^2 \sqrt{1 - v^2/c^2}\\
H = \vb{p} \cdot \vb{v} - L = \frac{mv^2 + mc^2}{\sqrt{\ldots}} = \frac{mc^2}{\sqrt{1 - v^2/c^2}}\\
H = T \Rightarrow L \neq T\; (\text{ненатур.})
\end{gather}
\end{multicols}
\end{ex}

\subsubsection{Вариационный принцип и законы сохранения (теорема Нётер)} \index{Теорема! Нётер}
Связь уравнений, которые следуют из вариационного принципа, с законами сохранения --- вторая глобальная фишка, делающая ВП центральным в современной теоретической физике, первая --- уравнения, следующие из ВП ковариантны, то есть не зависят от ввода системы координат. 

Существует неразрывная связь симметрий в системе и уравнений движения. Теорема Нётер позволяет связать симметрии и следующие их них интегралы движения способом, не зависящим от способа ввода координат.

\begin{thm}[Теорема Нётер]
Каждому преобразованию координат и времени, оставляющему функционал действия инвариантным, отвечает первый интеграл уравнения движения.

Математически, если дано преобразование координат и времени
\begin{gather}
q^* = \varphi(t, q, \alpha)\\
t^* = \psi (t, q, \alpha),
\end{gather}
которое удовлетворяет следующим трём пунктам  условий,
\begin{enumerate}
\item $q = q^*,$ $t = t^*$ при $\alpha = 0.$

\item При $\abs{\alpha} < \varepsilon$ существуют обратные преобразования $\tilde{\varphi}(t^*, q^*, \alpha) = q,$ $\widetilde{\psi}(t^*, q^*, \alpha) = t.$

\item  (Симметрии.) $\Phi$-инвариантность\footnote{Используемая дальше функция $L^*$ определяется уравнением \eqref{monster}.} $L^* \left(t^*, q^*, \dv{q^*}{t^*}\right) = L \left(t^*, q^*, \dv{q^*}{t^*}\right) + \dv{t^*} \Phi(t^*, q^*, \alpha)$. \footnote{Эквивалентное утверждение для действия $S[q] = S^*[q^*] = S[q^*] + \left.\Phi \right|_{t_1^*}^{t_2^*}$.}
\end{enumerate}

то на решениях системы существует первый интеграл уравнения Лагранжа
\[I(t, q, \dot{q}) \left\{ \sum_{j=1}^s \pdv{L}{\dot{q}_j} \pdv{\varphi_j}{\alpha} - H \pdv{\psi}{\alpha} + \pdv{\Phi}{\alpha} \right\}_{\alpha = 0} = const.\]
\end{thm}

\begin{ex}[Преобразование сдвига]
\begin{equation}
x^* = x + \alpha, \Phi = 0 \Rightarrow I = \pdv{L}{\dot{x}} \left(\pdv{X^*}{x}\right)_{\alpha = 0} = p_x.
\end{equation}
\end{ex}
\begin{ex}[Поворот]
\begin{equation}
\begin{cases}
x^* = x \cos \alpha + y \sin \alpha\\
y^* = -x\sin \alpha + y \cos \alpha,
\end{cases}
\Phi = 0 \Rightarrow I = p_x \pdv{x^*}{\alpha} + p_y \pdv{y^*}{\alpha} = yp_x - xp_y = M_z
\end{equation}
\end{ex}
\begin{ex}[Винтовая симметрия]
К повороту из предыдущего примера добавим сдвиг $z^* = z + \frac{h}{2\pi} \alpha$, тогда
$I = M_z + p_z \frac{h}{2\pi}.$
\end{ex}
\begin{ex}
\[t^* = t + \alpha \Rightarrow I = -H.\]
\end{ex}
\begin{ex}[Преобразования Галилея]
\begin{equation}
\begin{cases}
x^* = x - \alpha t \quad(\alpha = u)\\
t^* = t
\end{cases}
\Rightarrow \pdv{x^*}{\alpha} = -t,
\end{equation}
при этом 
\begin{gather}
L = \frac{m}{2} \dot{x}^2,\; \dot{x} = \dot{x}^* + \alpha,\\
L^* = L(\dot{x}^* + \alpha) = \underbrace{\frac{m}{2} \dot{x}^*}_{L^*(\dot{x}^*)} + \underbrace{m \alpha \dot{x}^* + \frac{m}{2} \alpha^2}_{\dv{t}\left(\alpha mx^* + \frac{m\alpha^2}{2}t\right)} \Rightarrow\\
\Rightarrow I = p_x (-t) + mx = const\; (= 0),\\
p_x = m \frac{x}{t}.
\end{gather}

\end{ex}



\begin{proof}[Доказательство теоремы Нётер]
Попробуем свойства $\Phi$-инвариантности записать в пределе $\alpha \to 0$. Разложим в ряды преобразования координат  и времени
\begin{align}
q^* &= \varphi(t, q, \alpha) = q + \alpha \cdot Q + \ldots & Q_j &= \left. \pdv{\varphi_j}{\alpha} \right|_{\alpha = 0}\\
t^* &= \psi (t, q, \alpha) =  t + \alpha \cdot T + \ldots & T &= \left. \pdv{\psi}{\alpha} \right|_{\alpha = 0}.
\end{align}
Заметим, что 
\begin{equation}
\dv{q^*}{t^*} = \frac{\dot{q} + \alpha \dot{Q} + \ldots}{1 + \alpha \dot{T} + \ldots} = \dot{q} + \alpha (\dot{Q} - \dot{q} \dot{T}) + \ldots.
\end{equation}
По формуле \eqref{monster} получим выражение для $L^*$
\begin{gather}
L^* = L\left(t^* - \alpha T, q^* - \alpha Q, \dv{q^*}{t^*} - \alpha(\dot{Q} - \dot{q} \dot{T}) \right) \cdot \left(1 - \alpha \dot{T} \right) =\\
= L(t^*, q^*, \dv{q^*}{t^*}) - \alpha \bigg\{ \underbrace{\pdv{L}{t}}_{= - \dot{H} T} T + \underbrace{\pdv{L}{q} Q}_{\dot{p} Q} + \pdv{L}{\dot{q}} \left(\dot{Q} - \dot{q} \dot{T}\right) + L\dot{T}  \bigg\} =\\
= L(*)  - \alpha \left\{-\dv{t} (HT) + \dv{t}(pQ)\right\} + \ldots, \label{pre_phi_inv}
\end{gather}
так как $\pdv{L}{\dot{q}} \dot{Q} = p \dot{Q},$ а $-\pdv{L}{\dot{q}} \dot{q} \dot{T} + L\dot{T} = \dot{T} (L - \dot{q} p) = - H\dot{T}$. А дальше воспользуемся $\Phi$-инвариантностью, и перепишем полученное для $L^*$ выражение \eqref{pre_phi_inv}
\begin{equation}
 L^* = L(*) + \underbrace{\dv{t^*} \Phi(t^*, q^*, \alpha)}_{\dv{t}\left(\alpha \left. \pdv{\Phi}{\alpha}\right|_{\alpha = 0} + \ldots\right)}.
 \end{equation} 
 Приравняем одинаковые порядки, тогда
 \begin{equation}
 \dv{t} \left\{ pQ - HT + \left. \pdv{\Phi}{\alpha} \right|_{\alpha = 0} \right\} = 0,
 \end{equation}
а дифференцируемое выражение в точности и есть сохраняющийся интеграл движения.
\end{proof}

\subsubsection{Как вариационная формулировка работает в приближенных к реальным условиях?}
\[L(t, q, \dot{q}) \Rightarrow S[q(t)] = \int_{t_1}^{t_2} L \left(t, q(t), \dot{q}(t)\right) \dd{t}\]
\begin{gather}
\text{ур. Лагранжа}\; \Leftrightarrow \begin{cases}
\var{S} = 0\\
\var{q(t_1)} = \var{q(t_2)} = 0
\end{cases}
\text{(ВПГ)}\quad
\oplus \quad
\begin{cases}
q_1, q_2\; \text{близки так, что нет кин. фокусов,}\\
\text{сопряжённых с началом (концом) траектории}\\
S \to min\; (max)
\end{cases}
\end{gather}

Что мы выигрываем, зная, что уравнения Лагранжа следуют из решения вариационной экстремальной задачи?
\begin{enumerate}
\item Ковариантность относительно замены координат.
\item Теорема Нётер. 
\begin{gather}
\begin{cases}
q^* = q + \alpha \cdot Q(t, q) + O(\alpha^2)\\
t^* = t + \alpha \cdot T(t, q) + O(\alpha^2)
\end{cases}
+ \text{$\Phi$-инфариантность}\; L^{*} = L \left(t^*, q^*, \dv{q^*}{t^*}\right) + \dv{t^*} \Phi (t^*, q^*, \alpha),\\
\left(t\; \text{и}\;q\; \text{не равны нулю одновременно,}\;\Phi = \Phi_0 + \alpha \cdot \Xi + O(\alpha^2)\right),
\end{gather}
тогда существует интеграл движения
\begin{gather}
\pdv{L}{\dot{q}}Q + H \cdot T + \Xi = const\; \text{на реш. ур. Лагранжа.}
\end{gather}
\begin{rmk}
\begin{gather}\begin{cases}
q^* = \varphi(q, t)\\
t^* = \psi(q, t)\\
\ldots
\end{cases} \Rightarrow
\begin{cases}
Q = \left. \pdv{\varphi}{\alpha} \right|_{\alpha= 0},\\
T = \left. \pdv{\psi}{\alpha} \right|_{\alpha = 0},\\
\Xi = \left. \pdv{\Phi}{\alpha} \right|_{\alpha= 0}
\end{cases}
\end{gather}
\end{rmk}
\end{enumerate}

\subsubsection{Лагранжиан свободной материальной точки.}\index{Лагранжиан! свободной материальной точки}
\paragraph{В Ньютоновской механике}\! из дифференциального подхода для свободной материальной точки мы знаем
\begin{equation}
L = \frac{mv^2}{2},
\end{equation}
но можно стартовать с вариационного принципа, и последний не утверждает, что $L = T -U,$ стартуем с того, что система описывается $L(t, q, \dot{q}),$ выведем механику сил.
Воспользуемся теми же принципами, что и для Ньютоновской механики.
\begin{enumerate}
\item Пространство однородно и изотропно. Время однородно. $\Rightarrow L (\not{t}, \not{\vb{r}}, \vb{v}) = L(v^2).$
Зная, что у нас $L(v^2),$ можем доказать первый закон Ньютона:
\begin{gather}
\vb{p} = \pdv{L}{\vb{v}} = \pdv{L}{v^2} \pdv{v^2}{\vb{v}} = \pdv{L}{v^2} \cdot 2\vb{v} = const\; (\text{$\vb{r}$  --- цикл.}) \Rightarrow v =const, \vb{v} = const\\
H = \vb{p}\cdot \vb{v} - L = 2v^2 \pdv{L(v^2)}{v^2} - L = const\; (\text{$t$  --- цикл.}),
\end{gather}
двумя способами получили :)
\item Чтобы воспользоваться теоремой Нётер, надо придумать какое-то преобразование, оставляющее уравнение движения инвариантным, вспомним про преобразование Галилея, которое тоже как бы свойство пространства-времени, делающее эквивалентными все инерциальные системы отсчёта в Ньютоновской механике. 
\begin{gather}
\begin{cases}
t^* = t\\
r^* = \vb{r} - \vb{u} \cdot t;\; \vb{u} = \alpha \cdot \vb{n}_{const}
\end{cases}
\stackrel{Th.\, Noether}{\longrightarrow} 
\begin{cases}
T = 0\\
Q = - \vb{n}\cdot t
\end{cases} \Rightarrow
\boxed{-\pdv{L}{v^2} 2(\vb{n} \cdot \vb{v})t = F (t, \vb{r})}\dv{t} \rightarrow\\
\pdv{L}{v^2} \cdot 2(\vb{n} \cdot \vb{v}) = \pdv{F(t, \vb{r})}{t} + \pdv{F(t, \vb{r})}{\vb{r}}\vb{v},\\
\intertext{но левая часть полученного равенства не зависит от времени и координат, поэтому правая тоже от них не зависит, значит,}
\pdv{L}{v^2} \cdot 2(\vb{n} \cdot \vb{v}) = \pdv{F(t, \vb{r})}{t} + \pdv{F(t, \vb{r})}{\vb{r}}\vb{v} = a + (\vb{b}, \vb{n}) \stackrel{\vb{b} =\vb{n}\cdot m}{=} m (\vb{n}, \vb{v})\\
\left(H/w\quad (\vb{n}, \vb{v}) = a + (\vb{b}, \vb{v}) \Rightarrow a =0, \vb{b} = \vb{n} \right)
\end{gather}
и мы, хитро подобрав константы, получили то, что нам надо в ответе:
\[\pdv{L}{v^2} = \frac{m}{2} \Rightarrow L = \frac{mv^2}{2} + const.\]
Ландау и Лифшиц дальше для несвободной материальной точки постулируют силы, то есть к лагранжиану свободной материальной точки аддитивно добавляют взаимодействие с внешними полями
\[L = \frac{mv^2}{2} - U(t, \vb{r}, \vb{v}) \ldots,\]
возможность так делать постулируется.
\end{enumerate}

\paragraph{В СТО}\! то же, что в предыдущем пункте, но с преобразованиями Лоренца
\begin{equation}
\begin{cases}
\vb{r}^* = \frac{\vb{r}-\vb{u} t}{\sqrt{1-u^{2} / c^{2}}}\\
t^* = \frac{t-(\vb{u}, \vb{r}) / c^{2}}{\sqrt{1-u^{2} / c^2}}.
\end{cases}
\stackrel{\vb{u} = \alpha \cdot \vb{n}}{\Longrightarrow}
\boxed{
\begin{cases}
\vb{r}^* = \vb{r} - \alpha \vb{n} \cdot t + \ldots\\
t^* = t - \alpha (\vb{n}, \vb{r})/c^2 + \ldots
\end{cases}}.
\end{equation}
По-прежнему время и пространство однородны, пространство изотропно. Значит, как и раньше,
\[\vb{p} = 2\vb{v} \pdv{L}{t^2}; H = 2v^2 \pdv{L}{v^2} - L \Rightarrow \vb{p} = const, \vb{v} = const, v = const.\]
Чтобы узнать форму $L(v^2)$, воспользуемся теоремой Нётер
\begin{gather}
\vb{p} Q - H \cdot T = - F(t, \vb{r})\\
-2(\vb{n}, \vb{v}) \pdv{L}{v^2} \cdot t + \left(2v^2 \pdv{L}{v^2} - L\right) \frac{(\vb{n}, \vb{r})}{c^2} =  F(t, \vb{r})\; \left| \dv{t} \right.\\ 
(\vb{n}, \vb{v}) \left\{ \left(2v^2 \pdv{L}{v^2} - L\right)\frac{1}{c^2} - 2\pdv{L}{v^2}\right\} = \underbrace{\pdv{F}{t}}_{=0} + \underbrace{\pdv{F}{\vb{r}}\vb{v}}_{\parallel \vb{n}} = (\vb{n}, \vb{v}) \cdot \underbrace{\frac{L_0}{c^2}}_{const}\\
-\pdv{L}{v^2}\left(1-\frac{v^{2}}{c^{2}}\right)=\frac{1}{2 c^{2}}\left(L-L_{0}\right)
\end{gather}
\begin{gather}
\pdv{\ln (L-L_0)}{v^2} = \hlf \frac{1}{v^2 - c^2} = \hlf \pdv{v^2} \ln \abs{v^2 - c^2}\\
L = L_0 + A \sqrt{c^2 - v^2} \to \frac{mv^2}{2}\; \text{при $v \to 0$}\; \stackrel{cA= - mc^2}{\Rightarrow} \boxed{L = - mc^2 \sqrt{1 - v^2 / c^2} + mc^2} \Rightarrow\\
H = \frac{mc^2}{\sqrt{1- v^2/c^2}} \neq L,
\end{gather}
то есть $L \neq T,$ система ненатуральная; $\vb{p} = \frac{m\vb{v}}{\sqrt{1- v^2/c^2}}.$

\subsection{Электромеханические аналогии}
\textbf{(Смешанные электромеханические системы в механике Лагранжа)}
\begin{ex}
\begin{equation}
U = \mathcal{E},\; U = \frac{q}{C},\; \mathcal{E} = -L\dot{I} = - L \ddot{q} \Rightarrow
\end{equation}
\begin{gather}
L \ddot{I} + \frac{1}{C} q = 0\; \text{ --- гармонический осциллятор с $\omega^2 = \frac{1}{LC},$}\\
L \ddot{I} + \frac{1}{C} I = 0\; \left| \cdot CL \right. \Longrightarrow CL^2 \ddot{I} + L I = 0.
\end{gather}
Будем рассматривать заряд на обкладках в качестве обобщённой координаты, тогда, рассматривая индуктивность в роли массы, запишем функцию Лагранжа
\begin{equation}
\mathcal{L} = \hlf L\dot{q}^2 - \frac{1}{2C} q^2 = T -U: 
\begin{cases}
T = \hlf LI^2\; \text{--- энергия магнитного поля в катушке,}\\
U = \frac{1}{2C} q^2\; \text{--- энергия энергия электрического поля в конденсаторе.}
\end{cases}
\end{equation}
Точно так же в качестве обобщённой координаты можно рассматривать ток (тогда роль массы играет выражение $CL^2$):
\begin{equation}
\mathcal{L} = \hlf CL^2 \dot{I}^2 - \hlf LI^2 = T - U,
\begin{cases}
T = \hlf CU^2,\\
U = \frac{1}{2} LI^2,
\end{cases}
\end{equation}
и, когда контур замкнутый, разницы никакой нет. Разница возникает, если контур разомкнуть. $H/w$
\end{ex}

\begin{ex}
\begin{gather}
\mathcal{L}_{\text{мех}} = \frac{m \dot{x}^{2}}{2}-\frac{k x^{2}}{2}-\operatorname{mg} x\\
\mathcal{L}_{\text{эл}} = \frac{1}{2} L(x) \dot{q}^{2}-\frac{1}{2 C(x)} q^{2},\\
\left(C(x) = \frac{S_C}{4\pi x}, L(x)  = 4\pi \frac{N^2 S_L}{L-x} \right)\\
\mathcal{L} = \mathcal{L}_{\text{мех}} + \mathcal{L}_{\text{эл}}
\end{gather}
Получим уравнения Лагранжа.
\begin{equation}
\dv{t} \pdv{\mathcal{L}}{\dot{x}} = m\ddot{x} = \pdv{\mathcal{L}}{x} = - kx -mg + \frac{1}{2}\pdv{L}{x} \cdot \dot{q}^2 - {\frac{q^2}{2} \pdv{x} \frac{1}{C}}\footnote{$-\frac{q^2}{2} \frac{4\pi}{S_C} = -2\pi \sigma q = Eq$ --- в точности электрическая сила взаимодействия двух пластин конденсатора. Можно показать, что в точности совпадает с силой сопротивления сжатию соленоида предыдущий член, да и все слагаемые могут быть получены из первых принципов.}
\end{equation}
\end{ex}
У нас теперь две степени свободы. Найти $q$  и $\dot{q}$ можно из второго уравнения движения:
\begin{equation}
\dv{t} \pdv{\mathcal{L}}{\dot{q}} = \dv{t} (L(x) \dot{q}) = L \ddot{q} + \pdv{L}{x} \dot{x} \dot{q} = \pdv{\mathcal{L}}{q} = \frac{q}{C}.
\end{equation}

\section{Интегрируемые задачи механики}
\subsection{Интегрирование уравнения движения систем с одной степенью свободы}
Мы будем интересоваться решением ДУ вида $\ddot{x} = F(t, x, \dot{x}).$\footnote{$x= 1;\; \dim x =1, s =1$}
\subsubsection{Классификация состояний равновесия автономной системы на плоскости}
\begin{equation}
\ddot{x} = F(x, \dot{x}) \Rightarrow \begin{cases}
\dot{x} = f(x, y)\\
\dot{y} = g(x, y)
\end{cases}
\end{equation}
Состояние равновесия
\begin{equation}
\begin{cases}
\dot{x} = 0\\
\dot{y} = 0
\end{cases}
\Leftrightarrow
\begin{cases}
f(x_0, y_0) =0\\
g(x_0, y_0) = 0.
\end{cases}
\end{equation}
Рассматриваемое малое возмущение
\begin{equation}
\begin{cases}
x = x_ 0 + \xi\\
y = y_0 + \eta 
\end{cases}
\Rightarrow
\begin{cases}
\dot{xi} = \alpha \xi + \beta \eta\\
\dot{\eta} = \gamma \xi + \delta \eta,
\end{cases}
\alpha = \left. \pdv{f}{x} \right|_{x_0, y_0}, \beta = \left. \pdv{f}{y} \right|_{x_0, y_0}, \ldots, \delta = \left. \pdv{g}{y} \right|_{x_0, y_0}.\footnote{Рассматриваем линейное состояние равновесия.}
\end{equation}
Будем искать решение в виде 
\begin{gather}
\begin{pmatrix}
\xi\\
\eta
\end{pmatrix} = 
\begin{pmatrix}
a\\
b
\end{pmatrix}
e^{\lambda t} \Rightarrow \begin{cases}
\lambda a = \alpha a + \beta\\
\lambda b = \gamma a + \delta b
\end{cases} \Rightarrow 
\det \begin{pmatrix}
\alpha - \gamma & \beta\\
\gamma & \delta - \lambda
\end{pmatrix}
= 0 \Leftrightarrow\\
\Leftrightarrow (\lambda - \alpha)(\lambda - \delta) = \gamma \beta,\\
\lambda_{1,2} = \frac{\alpha + \beta}{2} \pm \sqrt{\left(\frac{\alpha - \delta}{2}\right)^2 - \gamma \beta}.
\end{gather}
\begin{enumerate}
\item $\lambda = \pm i \omega$ --- состояние равновесия типа центр, устойчивое, не асимптотически устойчивое. Фазовые кривые --- эллипсы.
\item $\lambda p \pm i \omega$ --- состояние равновесия типа фокус, может быть в зависимости от знака $p$ как устойчивым, так и неустойчивым\footnote{Устойчиво при $p < 0$.}, устойчивое равновесие асимптотически устойчиво. Фазовые кривые <<сжимаются>> или <<раскручиваются>>.
\item $\lambda \in \mathbb{R}, \lambda_1 \cdot \lambda_2 > 0$ ---состояние равновесия типа узел --- равновесие типа фокус при очень сильном трении. Асимптотически устойчиво либо неустойчиво.
\item $\lambda \in \mathbb{R}, \lambda_1 \cdot \lambda_2 < 0$ --- состояние равновесия типа седло, неустойчиво.
\end{enumerate}
$H/w$ Как найти направления асимптот?

\begin{ex}
\begin{gather}
\ddot{x} = f(x, v), v =0, m = 1 \\
\begin{cases}
\dot{x} = v\\
\dot{v} = f_0(x) - \mu \cdot v = - \pdv{U(x)}{x} - \mu \cdot x
\end{cases}\\
\pdv{U(x_0)}{x} = 0; \pdv[2]{U(x_0)}{x} = U'' \neq 0; \begin{cases}
x = x_0 + \xi\\
v = \eta
\end{cases}
\Rightarrow\\
\begin{cases}
\dot{\xi} = v\\
\dot{v} = -U'' \cdot \xi - \mu \cdot v
\end{cases} \Rightarrow 
\det \begin{pmatrix}
-\lambda & 1\\
-U'' & -\lambda - \mu
\end{pmatrix} = 0 \Rightarrow \boxed{\lambda^2 + \mu \lambda + U'' = 0} \Rightarrow \\
\lambda_{1, 2} = - \frac{\mu}{2} \pm \sqrt{\frac{\mu^2}{4} - U''}
\end{gather}
Чередуются состояние равновесия типа седло, узел и фокус, центры превращаются в диссипативные состояния равновесия, фокусы отвечают случаям малого трения, при её увеличении превращаются в узлы.
\end{ex}

\subsubsection{Интегрирование уравнения движения консервативных одномерных систем}
Будем рассматривать лагранжеву задачу, в которой существует интеграл энергии.
\[L(x, \dot{x})\; \text{--- уравнения Лагранжа решаются в квадратурах.}\]
\begin{equation}
\pdv{L}{t} = 0 \Rightarrow H(x, \dot{x}) = const\; \text{на решениях}.
\end{equation}
Фазовый портрет --- линии уровня $H(x, \dot{x})$.\index{Фазовый портрет}
\begin{gather}
H(x, \dot{x}) = E = const \Rightarrow \dot{x} = V(x, E)\\
\dv{x}{t} = V(x, E) \Rightarrow \dd{t} = \frac{\dd{x}}{V(x, E)} \Rightarrow \boxed{t= \int^x \frac{\dd{x}}{V(x, E)}},
\end{gather}
аддитивная константа в нижнем пределе интегрирования отвечает заданному $E$, его линии уровня, и такие точки надо различать, поэтому на самом деле везде в решении вместо $V(x, E)$ должно быть $V_i (x, E),$ где $i$ определяет \textit{ветку однозначности}.\index{Ветка однозначности}

\begin{dfn}
$x$ --- ограниченная область --- движение финитное; $x$ --- неограниченная область --- движение инфинитное.
\end{dfn}
\begin{pst}
Всякое финитное движение в одномерной консервативной системе обязательно является периодическим. 
\end{pst}

Частный случай --- натуральная система.
\begin{gather}
L(x, \dot{x}) = \frac{1}{2} m(x) \dot{x}^2 - U(x)\\
H(x, \dot{x}) = \frac{1}{2} m \dot{x}^2 + U(x) = E \Rightarrow \dot{x} = \pm \sqrt{\frac{2}{m}\left(E- U(x)\right)}\\
\boxed{t = \pm \int^x \sqrt{\frac{m(x)}{2} \frac{\dd{x}}{\sqrt{E - U(x)}}}},
\end{gather}
знак в начале определяется начальными условиями, меняется в точке разворота, где знаменатель обращается в нуль, то есть $U(x) = E$. 
\begin{gather}
p(x) = \pdv{L}{\dot{x}} = m(x) \dot{x} = \pm \sqrt{2m (E - U)}\\
\pdv{p}{E} = \sqrt{\frac{m}{2}}\frac{1}{\sqrt{E- U}} \Rightarrow t = \pm \pdv{E} \int^x p (x, E) \dd{x},
\end{gather}
а интеграл --- площадь под графиком $p(x)$.
\begin{dfn}
$U(x) < E $ --- область возможного движения, иначе мы переходим в комплексное время, что запрещено в классической механике.
\end{dfn}
\begin{dfn}
$U(x) = E$ --- точки разворота.
\end{dfn}

Фазовый портрет обладает зеркальной симметрией относительно оси $x$.
$[x_3, \infty]$ --- инфинитное движение.

\[\dot{x} \sim \sqrt{E - U(x)} \sim \pm \sqrt{x - x_1}\]
$\dot{x} \sim \pm \sqrt{(x - x_0)^2} = \pm \abs{x-x_0}$ --- в состоянии равновесия с асимптотами $\left(\pdv{U(x_0)}{x} =0\right)$.

\subsubsection{Диссипативные одномерные системы}
В общем случае способа решения в квадратурах для диссипативных систем нет. И оказывается, что все известные интегрируемые диссипативные системы на самом деле не по-настоящему диссипативные. Поясним, о чём идёт речь. Вспомним консервативную задачу $L = \frac{1}{2} m(x) \dot{x}^2 - U(x)$ и посмотрим, какому уравнению движения такой лагранжиан соответствует (а интересует нас уравнение вида $\ddot{x} = F(t, x, \dot{x})$).
\begin{equation}
L = \frac{1}{2} m(x) \dot{x}^2 - U(x) \Rightarrow m\ddot{x} + \frac{1}{2} \pdv{m}{x} \dot{x}^2 = -  \pdv{U}{x} \Rightarrow \begin{cases}
\ddot{x} = f(x) - \mu (x) \dot{x}^2,\; \text{где}\\
f(x) = - \frac{1}{m(x)} \pdv{U}{x}, \; \text{а} \\
\mu({x}) = \frac{1}{2m} \pdv{m}{x}, 
\end{cases}
\end{equation}
и мы можем получить решение в квадратурах для любых $f(x)$ и $\mu(x)$:
\begin{align} 
\mu({x}) &= \frac{1}{2m} \pdv{m}{x} \Rightarrow \mu = \pdv{x}(\ln \sqrt{m}) \Rightarrow m = \left( e^{\int \mu \dd{x}}\right)^2,\\
f(x) &= - \frac{1}{m(x)} \pdv{U}{x} \Rightarrow U = - \int m f \dd{x}.
\end{align}
Задачи вида $\ddot{x} + \omega_0^2 x + \mu \dot{x}^n =0,$ где $n = 2k$, допускают решение в квадратурах, в системе <<псевдотрение>>. При $n = 2k +1$ трение уже истинное, поскольку <<штука>> $\dot{x} \cdot \mu \dot{x}^n$ знакоопределена, а она управляет скоростью вытекания энергии:
\begin{equation}
 H_0 = \frac{1}{2} \dot{x}^2 + \frac{\omega^2}{2}x^2 \Rightarrow \dv{H_0}{t} = - \mu \dot{x}^{n+1}.
 \end{equation} 
Но можно и при $n = 2k$ сделать трение: достаточно добавить сигнум $\ddot{x} + \omega_0^2 x + \mu \sign \dot{x} \cdot \dot{x}^n = 0,$ при $n = 2k$ получаем трение, при $n = 2k +1$ --- псевдотрение.
\begin{ex}[n= 0]
Как выглядит фазовый потрет системы $\ddot{x} + \omega_0^2 x + \mu = 0$? А если добавить сигнум, как увидеть, что задача стала диссипативной?
Как выглядит настоящее диссипативное решение? Сшиваем по непрерывности два решения. Зона застоя.

Когда рисуем сигнум скорости, говорим, что фазовой фазовый портрет системы получается из фазового портрета системы без сигнума путём склейки: разрезаем по горизонтальной прямой, с помощью сигнума разворачиваем и склеиваем по непрерывности.

\emph{H/w} $\sign \dot{x} \cdot \dot{x}$ --- задача с трением --- нарисовать фазовый портрет, зная ФП консервативной системы.
\end{ex}

\paragraph{Тривиальные случаи}
$\dot{x} = F(v) \Rightarrow \dot{v} = F(v) \Rightarrow v(t) \Rightarrow x(t) \int v \dd{t}$

\subsubsection{Качественный анализ на фазовой плоскости}
...
\begin{ex}[Линейное трение в системе можно учитывать с помощью чисто лагранжевой задачи с нестационарным лагранжианом]
\begin{align}
L &= \frac{1}{2} \dot{x}^2 - U(x) &  &\longrightarrow &  \ddot{x} &= - \pdv{U}{x} = f(x),\\
\intertext{и в эту задачу мы можем добавить линейное трение:}
L &= \left(\frac{1}{2} \dot{x}^2 - U\right) e^{2\gamma t} &  &\longrightarrow & p &= \pdv{L}{\dot{x}} = \dot{x} e^{2 \gamma t}\\
& & & & \dot{p} &=(\ddot{x} + 2\gamma \dot{x})e^{2\gamma t} = \pdv{L}{x} = -\pdv{U}{x} e^{2\gamma t} 
\end{align}
И последнее равенство, которое можно сократить на $e^{2\gamma t}$ является следствием одной теоремы\dots
\end{ex}
\begin{pst}
\begin{equation}
\begin{rcases}
\ddot{x} = F(t, x, \dot{x})\\
s =1
\end{rcases}
\Rightarrow
 \exists \left.
\begin{matrix} 
L(t, x, \dot{x})\\
\mu (t, x, \dot{x})\neq 0
\end{matrix}
\right|\; (\ddot{x} - F) \cdot \mu = \dv{t} \pdv{L}{\dot{x}} - \pdv{L}{x}
\end{equation}
\end{pst}

\begin{pst}
\begin{equation}
\ddot{x} = F(x, \dot{x}) \Rightarrow
\exists \;
\begin{matrix} 
L(x, \dot{x})\\
\mu (x, \dot{x})
\end{matrix}
\quad \ldots
\end{equation}
\end{pst}
\begin{ex}
\begin{equation}
\ddot{x} + \omega^2 x + 2\gamma \dot{x} = 0 \Leftarrow L = \frac{\dot{x} + \gamma x}{\gamma x} \arctg \frac{\dot{x} + \gamma x}{\omega x} -\frac{\omega}{2\gamma} \ln \left( \omega^2 x^2 + (\dot{x} + \gamma x)^2 \right)\footnote{\emph{H/w} проверить. См. статью \cite{Shalashov_2017}.}
\end{equation}
\begin{equation}
\begin{cases}
x(t) = A e^{-\gamma t} \sin (\omega t + \gamma)\\
p(t)  = \pdv{L}{\dot{x}}\; \text{уходит в бесконечность за конечное время,}
\end{cases}
\end{equation}
движение инфинитно.
\end{ex}

\documentclass[12pt, a4paper]{article}
\usepackage[utf8]{inputenc}
 \usepackage[T1, T2A]{fontenc}
\usepackage[english, russian]{babel}
\usepackage{caption}
\usepackage{indentfirst}
\usepackage{graphicx, xcolor}
\usepackage{cmap}

\usepackage[unicode, pdftex]{hyperref}
\hypersetup{linkcolor=blue, urlcolor=blue, colorlinks=true}
\usepackage{hyphenat}
\hyphenation{объект}
\usepackage{wrapfig}
\usepackage[left=1.4cm,right=1.4cm,top=1.5cm,bottom=1.5cm,bindingoffset=0cm]{geometry}
\usepackage{tocloft}    
\usepackage{titlesec} \titlelabel{\thetitle.\quad} 
\frenchspacing
\makeatletter
\renewcommand{\@biblabel}[1]{#1.} % Заменяем библиографию с квадратных скобок на точку:
\makeatother
\makeindex

\usepackage[]{mathtools}
\renewcommand{\theequation}{\thesection.\arabic{equation}}
    
\parindent=1.25cm
%\parskip=0.1cm

\usepackage{physics} 
\usepackage{siunitx} % typesets numbers with units very nicely
\usepackage{amssymb,amsfonts,amsmath,mathtext,cite,enumerate,float}
\DeclareMathOperator{\sign}{sgn}
\usepackage{amsthm}

%\usepackage[dvips]{graphicx}

\usepackage{multicol}

\bibliographystyle{unsrt}


\begin{document}
\renewcommand{\cftsecaftersnum}{.}
\renewcommand{\cftsubsecaftersnum}{.}

\renewcommand\refname{Список литературы}

\theoremstyle{plain}
\newtheorem{thm}{Теорема}[section]
\newtheorem{lem}[thm]{Лемма}
\newtheorem{pst}{Постулат}[section]

\theoremstyle{definition}
\newtheorem{dfn}{Определение}[section]
\newtheorem{cns}[thm]{Следствие}

\theoremstyle{remark}
\newtheorem{task}{Задача}[section]
\newtheorem{ex}{Пример}[subsection]
\newtheorem{cex}[ex]{Контрпример}
\newtheorem{rmk}{Замечание}[subsection]

\newcommand*{\eqdef}{\stackrel{\mathrm{def}}{=}}
\newcommand*{\is}[1]{\stackrel{\mathrm{\eqref{#1}}}{=}}
\newcommand*{\eqq}[1]{\stackrel{\mathrm{#1}}{=}}
\newcommand*{\hlf}{\frac{1}{2}}

\columnseprule = 0.4pt

\mathtoolsset{showonlyrefs=true}
\mathtoolsset{showmanualtags=true}

\begin{center}
\Huge{\textbf{Теоретическая механика}}
\end{center}
\tableofcontents
\newpage
\input{section1.tex}
\section{Механика Лагранжа}
\input{bridging}
\newpage
\input{oscillation}
\input{var}

\section{Интегрируемые задачи механики}
\input{integration_eq_of_motion}

\input{abridgement.ind}
\bibliography{library}
\end{document}

\bibliography{library}
\end{document}

\bibliography{library}
\end{document}

\bibliography{library}
\end{document}
