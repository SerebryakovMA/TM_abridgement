\documentclass[12pt]{article}
\usepackage[utf8]{inputenc}
\usepackage[T2A]{fontenc}
\usepackage[english,russian]{babel}
\usepackage{caption}
\usepackage{indentfirst}
\usepackage{graphicx, xcolor}

\usepackage[unicode, pdftex]{hyperref}
\hypersetup{linkcolor=blue, urlcolor=blue, colorlinks=true}
\usepackage{hyphenat}
\hyphenation{объект}
\usepackage{wrapfig}
\usepackage[left=1.4cm,right=1.4cm,top=1.5cm,bottom=1.5cm,bindingoffset=0cm]{geometry}
\usepackage{tocloft}    
\usepackage{titlesec} \titlelabel{\thetitle.\quad} 
\frenchspacing
\makeatletter
\renewcommand{\@biblabel}[1]{#1.} % Заменяем библиографию с квадратных скобок на точку:
\makeatother
\makeindex
\renewcommand{\theequation}{\thesection.\arabic{equation}}
    
\parindent=1.25cm
%\parskip=0.1cm

\usepackage{physics} 
\usepackage{siunitx} % typesets numbers with units very nicely
\usepackage{amssymb,amsfonts,amsmath,mathtext,cite,enumerate,float}
\usepackage{amsthm}

%\usepackage[dvips]{graphicx}

\newcommand{\executeiffilenewer}[3]{
	\ifnum\pdfstrcmp{\pdffilemoddate{#1}}
	{\pdffilemoddate{#2}} > 0 {\immediate\write18{#3}}\fi}
\newcommand{\includesvg}[1]{
	\executeiffilenewer{#1.svg}{#1.pdf}
	{inkscape -z -D --file=#1.svg 
	 --export-pdf=#1.pdf --export-latex}
	\input{#1.pdf_tex}%
}

\usepackage{multicol}

\bibliographystyle{unsrt}


\begin{document}
\renewcommand{\cftsecaftersnum}{.}
\renewcommand{\cftsubsecaftersnum}{.}

\renewcommand\refname{Список литературы}

\theoremstyle{plain}
\newtheorem{thm}{Теорема}[section]

\theoremstyle{definition}
\newtheorem{dfn}{Определение}[section]
\newtheorem{cns}[thm]{Следствие}

\theoremstyle{remark}
\newtheorem{task}{Задача}[section]
\newtheorem{ex}{Пример}[subsection]
\newtheorem{cex}[ex]{Контрпример}
\newtheorem{rmk}{Замечание}[subsection]

\newcommand{\eqdef}{\stackrel{\mathrm{def}}{=}}

\begin{center}
\Huge{\textbf{Теоретическая механика}}
\end{center}
\tableofcontents
\newpage
\section{Обзор основных понятий и законов классической механики}
\subsection{Механика как часть теоретической физики}
Общая (экспериментальная) физика основана на принципе индукции и поставляет теоретической физике математические модели (физические законы). Теоретическая физика использует дедукцию, и возвращает общей физике предсказания результатов экспериментов.

\begin{dfn} Механика --- наука о движении материальных объектов в пространстве и времени. \end{dfn}
Откажемся от определения, в котором сложно объяснить слова в правой части, и будем перечислять...
\begin{enumerate}
\item Пространство и время независимы.
\item Пространство трёхмерно, евклидово, однородно, изотропно.
\item Время однородно, однонаправленно (?).
\end{enumerate}
С точки зрения современной физики эти положения неверны, но это представления классической физики. А как сделать их более корректными?
\begin{enumerate}
\item СТО $v \ll c$.
\item КМ $\Delta l \gg \frac{\hbar}{p}$ --- длина волны де Бройля.
\item ОТО $m \varphi \ll mc^2$, $\varphi$ --- потенциал.
\item ВВ $t \gg t_{BB}$.
\end{enumerate}

\begin{dfn}
Материальная точка --- тело, размерами которого при рассмотрении данного класса движение модно пренебречь.
\end{dfn}
\begin{dfn}
Абсолютно твёрдое тело --- набор материальных точек, расстояния между которыми не изменяются.
\end{dfn}

\begin{task}[Задача Кэлли (?)] Какое минимальное количество связей между материальными точками необходимо установить, чтобы их система обрела жёсткость? \end{task}

\subsection{Механика системы материальных точек}
\begin{equation*}
\vb{r}(t) = \{x(t), y(t), z(t) \}; \thickspace \mathop{\{\vb{r}_i(t)\}}\limits_{i = \overline{1{,}N}}
\end{equation*}  

\paragraph{Кинематика материальной точки.}
$\vb{r}(t)$ --- закон движения (иногда знаем траекторию и положение точки для любого момента времени). 
\begin{align*}
\vb{r} &= r \vb{e}_r = \sqrt{x^2 + y^2 + z^2} \vb{e}_r, \\
\vb{e}_r &= \frac{\vb{r}}{r} = \left\{ \frac{x}{\sqrt{x^2 + y^2 + z^2}}, \frac{y}{\sqrt{x^2 + y^2 + z^2}}, \frac{z}{\sqrt{x^2 + y^2 + z^2}} \right\};\\
\vb{v} &= \dv{\vb{r}}{t} = \left\{\dot x, \dot y, \dot z\right\},\\
\vb{v} &= v \vb*{\tau}.
\end{align*}

\paragraph{Криволинейные координаты.}
\begin{align*}
q &= (q_1, q_2, q_3), x =X(q, t), y = Y(q, t), z =Z(q, t) \Rightarrow \\
q(t) &\Rightarrow \vb{r}(t) = \vb{r}{\vb{q}(t), t} = \left\{X(q(t), t), Y(q(t), t), Z(q(t), t)\right\}; \\
\vb{v}(t) &= \dv{\vb{r}}{t} = \sum \pdv{\vb{r}}{q_i} \dot q_i + \pdv{\vb{r}}{t}; \\
\vb{v}_i &= \left\{\pdv{X}{q_i} \dot q_i, \pdv{Y}{q_i} \dot q_i, \pdv{Z}{q_i} \dot q_i \right\}.
\end{align*}

\begin{equation*}
\vb{v} = \sum v_i \vb{e}_i; (e_i, e_j) = \delta_{ij} \Rightarrow v^2 = v_1^2 + v_2^2 + v_3^2
\end{equation*}

\begin{ex}[Цилиндрические координаты] Пусть $q = (\varphi, \rho, z),$ и
\begin{equation*}
\begin{cases}
x = \rho \cos \varphi   \\
y = \rho \sin \varphi \\
z,
\end{cases}
\end{equation*}
тогда
\begin{equation*}
\vb{v} = \dv{t} \left\{\rho \cos \varphi, \rho \sin \varphi, z \right\} \Rightarrow
\end{equation*}Alt+E, Alt+Q
\begin{equation*}
\begin{cases}
\vb{v}_{\varphi} = \rho \dot \varphi \underbrace{\{-\sin \varphi, \cos \varphi, 0\}}_{\vb{e}_\varphi} \\
\vb{v}_\rho = \dot \rho \underbrace{\{\cos \varphi, \sin \varphi, 0\}}_{\vb{e}_\rho} \\
\vb{v}_z = \dot z \{0, 0, 1\},
\end{cases}
\end{equation*}
значит, 
\begin{equation*}
(\vb{e}_\rho, \vb{e}_\varphi) = 0.
\end{equation*}
\end{ex}

\paragraph{Ускорение.}
\begin{align*}
\vb{a} & \eqdef \dv{\vb{v}}{t} = \dv[2]{\vb{r}}{t} = \left\{\ddot{x}, \ddot{y}, \ddot{z} \right\} \\
\vb{a} &= \dv{t} (v \vb*{\tau}) = \dot{v} \vb*{\tau} + \underbrace{v \dot{\vb*{\tau}}}_{= \frac{v^2}{R} \vb{n}} \\
\vb{n} &= \frac{\dot{\vb{r}}}{\dot{\vb*{\tau}}}\\
\abs{\dot{\vb*{\tau}}} &= \frac{v}{R}\\
[\vb*{\tau}, \vb{n}] &= \vb{b} \thickspace \text{--- бинормаль --- по ней не может быть направлено ускорение.}
\end{align*}


\paragraph{Основные постулаты ньютоновской механики.}
Сразу отметим, что количество точек в системе может быть бесконечным, или даже несчётным.

\begin{enumerate}
\item[0.] Пространство $+$ Время классические (однонаправленность времени не учитываем).
\item Первый закон Ньютона: существуют инерциальные системы отсчёта, в которых изолированная точка движется равномерно и прямолинейно.

Добавим принцип относительности Галилея, чтобы не возникало выделенных направлений из-за введения системы отсчёта: законы движения инвариантны относительно преобразования Галилея в замкнутой системе. Преобразование Галилея:
\begin{equation}
\begin{cases}
t = t'\\
\vb{r'} = \vb{r} + \vb{u}t, \qquad \vb{u} = const.
\end{cases}
\label{Galilean_tr}
\end{equation}
Система \eqref{Galilean_tr} порождает бесконечный класс инерциальных систем, и однородность восстанавливается.

\item Второй закон Ньютона. Принцип детерминизма Ньютона <<сидит>> в порядке дифференциальных уравнений для описания динамики системы.
\begin{equation*}
\Ddot{\vb{r}}_i = \vb{f}_i (\vb{r}_1, \dots, \vb{r}_N, \vb{v}_1, \dots, \vb{v}_N, t) 
\end{equation*}
$6N$ свободных констант --- по две на каждую степень свободы.
\begin{equation*}
\begin{cases}
\vb{r}_i (t_0) = {\vb{r}_i}_0\\
\vb{v}_i (t_0) = {\vb{v}_i}_0,
\end{cases}
\end{equation*}
существует, правда, множество меры нуль всяких исключений: диссипативные состояния равновесия, например.

Второй закон Ньютона:
\begin{equation}
m_i \ddot{\vb{r}}_i = \vb{F}_i \left(\{\vb{r}_i\}, \{\dot{\vb{r}}_i, t\right).
\label{Newton's_2nd_law}
\end{equation}
Отметим, что уравнение \eqref{Newton's_2nd_law} работает в инерциальных системах отсчёта, $m_i$ --- характеристика материальной точки, не зависящая от движения.

Экспериментальный способ измерения:
\begin{equation*}
\begin{cases}
m_1 a_1 = F\\
m_2 a_2 = F
\end{cases}
\Rightarrow \; \frac{m_1}{m_2} = \frac{a_2}{a_1}, \qquad
\begin{cases}
ma_1 = F_1\\
ma_2 = F_2
\end{cases}
\Rightarrow \; \frac{F_1}{F_2} = \frac{a_1}{a_2}.
\end{equation*}
\paragraph{Импульс.}
\begin{align*}
\vb{p}_i & \eqdef m_i \vb{v}_i && \text{--- момент импульса,}\\
\vb{M}_i & \eqdef m_i [\vb{r}_i, \vb{v}_i] && \text{--- момент силы,}\\
\vb{N}_i & \eqdef [\vb{r}_i, \vb{F}_i].
\end{align*}

\begin{cex}
Могут быть патологические случаи.
$m(t)$ --- ракета и реактивная сила или, например, тележка с тающим льдом: включаем массу в систему --- система с сохраняющейся массой, а потом смотрим на динамику подсистем. $\vb{F}$ может зависеть от разных других вещей, когда мы пытаемся описать механическим немеханические явления: $m(\dot{\vb{r}})$ --- квазиклассические частицы в твёрдом теле, тела в СТО, $m(\ddot{\vb{r}})$ --- присоединённая масса в гидродинамике, $F(\dddot{\vb{r}})$~--- радиационное трение~--- электрон летает вокруг ядра по боровской  орбите, излучает электромагнитные волны как любой движущийся заряд, теряя таким образом энергию, и падает на ядро, нарушается принцип детерминизма Ньютона --- пытаемся описать электромагнитную задачу механическим языком.
\end{cex}

\item
Третий закон Ньютона. Чтобы его определить, придётся все силы разбить на внутренние и внешние.
\begin{equation*}
\vb{F}_i = \vb{F}_i^{(e)} (t, \vb{r}_i) + \sum \vb{F}_{ij}^{(i)} (t, \vb{r}_i, \vb{r}_j), 
\end{equation*}
внешняя сила может зависеть только от координаты точки (и времени), а все оставшиеся силы~--- внутренние. Для <<всех оставшихся>>  предположили, что каждая из точек действует на $i$-ю точку независимо от всех остальных. И обрываем ряд, то есть считаем, что силы, в которой есть $\vb{r}_i, \vb{r}_j$ и $\vb{r}_k$ уже нет~--- приближение парных взаимодействий, и третий закон Ньютона справедлив только при  его применении. Формулировка третьего закона Ньютона:

\begin{figure}[h]\centering
\def\svgwidth{9cm}
\includesvg{pic3}
\end{figure}
                                                                      
\begin{equation}
\begin{cases}
\vb{F}_{ij}^{(i)} + \vb{F}_{ji}^{(i)} = 0 \label{3d_N's-law}\\
[\vb*{\rho}_{ij}, \vb{F}_{ij}] = 0 
\end{cases}
\end{equation}


\begin{cex}
Третий закон Ньютона справедлив только для объектов, которые являются материальными точками, но существуют объекты нулевого размера, не являющиеся материальными точками: например, электрический диполь, у которого есть вращающий объект, который может ещё и энергию забирать...
\end{cex}
\begin{cns}


\begin{enumerate}
\item \begin{equation*}
 \vb{F} = \sum \limits_{i = 1}^N \vb{F}_i^{(e)} = \vb{F}^{(e)} \Rightarrow \left( \sum m_i \right) \cdot \vb{\ddot{R}} = \vb{F} = \vb{F}^{(e)}, \vb{R} = \frac{\sum m_i \vb{r}_i}{\sum m_i},
\end{equation*}
то есть центр масс системы материальных точек  с попарным взаимодействием движется так же, как одна точка с суммарной массой системы в поле равнодействующей всех внешних сил.
\begin{ex}
Центр масс двигавшегося по параболе и разорвавшегося в некоторой точке снаряда продолжит движение по параболе.
\end{ex}
\begin{cex}
У Карлсона, летящего по параболе, раскрылся парашют...
\end{cex} 
\item \begin{equation*}
\vb{N} = \sum \vb{N}_i = \sum \vb{N}_i^{(e)},
\end{equation*}
то есть суммарный момент сил, действующих на систему с попарным взаимодействием, равен суммарному моменту внешних сил. Как это можно доказать?
\begin{equation}
[\vphantom{\vb{F}_{j}}\vb{r}_i, \vb{F}_{ij}] + [\vphantom{\vb{F}_{j}}\vb{r}_j, \vb{F}_{ji}] = [\vphantom{\vb{F}_{j}}\vb{r}_i - \vb{r}_j, \vb{F}_{ij}] = 0, \label{ftr_3d_N's-law}
\end{equation}
то есть сначала воспользовались первой часть третьего закона Ньютона \eqref{3d_N's-law}, чтобы поменять знак, а потом --- второй, чтобы приравнять к нулю.
\end{enumerate}
\end{cns}
\end{enumerate}
\newpage
\paragraph{Законы сохранения в механике Ньютона.}
\subparagraph{Закон сохранения импульса.}

\begin{gather*}
\vb{p} \eqdef \sum m_i \vb{v}_i \Rightarrow\\
\dot{\vb{p}} = \vb{F} = \vb{F}^{(e)},
\end{gather*}
первый переход осуществлён в силу II закона Ньютона, второй --- III закона Ньютона.
\begin{equation*}
\vb{p} = const \Leftarrow \sum \vb{F}_i^{(e)} = 0,
\end{equation*}
согласно III закону Ньютона.

\subparagraph{Закон сохранения момента импульса.}
\begin{gather*}
\vb{M} \eqdef \sum m_i [\vb{r}_i, \vb{v}_i];\\
\dot{\vb{M}} = \sum m_i \bigl\{ [\vb{r}_i, \dot{\vb{v}}_i] + [\dot{\vb{r}_i}, \vb{v}] \bigr\} = \vb{N} = \vb{N}^{(e)};\\
\begin{cases}
\sum \vb{N}_i^{(e)} = 0,\\
\text{III закон Ньютона}
\end{cases}
\Rightarrow \vb{M} = const
\end{gather*}
\subparagraph{Закон сохранения энергии.}
\begin{gather*}
m \ddot{\vb{r}} = \vb{F} \qquad |\cdot \vb{v} \Rightarrow\\
m \vb{v} \dv{\vb{v}}{t} = m \left( v_x \dot{v}_x + v_y \dot{v}_y + v_z \dot{v}_z \right) = \frac{m}{2} \cdot2 \dv{t} \left( v_x^2 + v_y^2 + v_z^2 \right) = \dv{T}{t} \Rightarrow\\
T = \frac{mv^2}{2} 
\end{gather*}
\begin{figure}[h]\centering
\def\svgwidth{7cm}
%LaTeX with PSTricks extensions
%%Creator: inkscape 0.92.3
%%Please note this file requires PSTricks extensions
\psset{xunit=.5pt,yunit=.5pt,runit=.5pt}
\begin{pspicture}(793.7007874,1122.51968504)
{
\newrgbcolor{curcolor}{0 0 0}
\pscustom[linewidth=0.60732901,linecolor=curcolor]
{
\newpath
\moveto(189.77664889,195.44104651)
\curveto(192.96828537,207.40297161)(194.93820634,223.75863739)(211.72998819,227.53531623)
\curveto(244.2983589,234.86035102)(277.71617316,209.194393)(305.94640956,217.66014962)
\curveto(346.45433384,229.80781159)(332.4532171,287.32880867)(369.0622661,298.30724928)
\curveto(405.39902765,309.20404491)(441.05372138,273.90866077)(482.48784121,292.54674993)
\curveto(496.69834581,298.93896367)(506.39595921,359.32339155)(521.82091916,374.01676563)
}
}
{
\newrgbcolor{curcolor}{0 0 0}
\pscustom[linestyle=none,fillstyle=solid,fillcolor=curcolor]
{
\newpath
\moveto(190.06957919,196.53891914)
\curveto(190.75695096,196.39047927)(191.18090692,195.7681217)(191.01590944,195.14972814)
\curveto(190.85091197,194.53133459)(190.15913402,194.14992289)(189.47176225,194.29836276)
\curveto(188.78439048,194.44680263)(188.36043453,195.0691602)(188.525432,195.68755376)
\curveto(188.69042948,196.30594731)(189.38220742,196.68735901)(190.06957919,196.53891914)
\closepath
}
}
{
\newrgbcolor{curcolor}{0 0 0}
\pscustom[linewidth=0.32390881,linecolor=curcolor]
{
\newpath
\moveto(190.06957919,196.53891914)
\curveto(190.75695096,196.39047927)(191.18090692,195.7681217)(191.01590944,195.14972814)
\curveto(190.85091197,194.53133459)(190.15913402,194.14992289)(189.47176225,194.29836276)
\curveto(188.78439048,194.44680263)(188.36043453,195.0691602)(188.525432,195.68755376)
\curveto(188.69042948,196.30594731)(189.38220742,196.68735901)(190.06957919,196.53891914)
\closepath
}
}
{
\newrgbcolor{curcolor}{0 0 0}
\pscustom[linestyle=none,fillstyle=solid,fillcolor=curcolor]
{
\newpath
\moveto(522.6826305,374.83760712)
\curveto(523.19655466,374.40094243)(523.21972735,373.67130608)(522.73435524,373.20895454)
\curveto(522.24898314,372.74660301)(521.43796009,372.72575572)(520.92403593,373.16242041)
\curveto(520.41011176,373.5990851)(520.38693907,374.32872145)(520.87231118,374.79107298)
\curveto(521.35768328,375.25342451)(522.16870633,375.2742718)(522.6826305,374.83760712)
\closepath
}
}
{
\newrgbcolor{curcolor}{0 0 0}
\pscustom[linewidth=0.32390881,linecolor=curcolor]
{
\newpath
\moveto(522.6826305,374.83760712)
\curveto(523.19655466,374.40094243)(523.21972735,373.67130608)(522.73435524,373.20895454)
\curveto(522.24898314,372.74660301)(521.43796009,372.72575572)(520.92403593,373.16242041)
\curveto(520.41011176,373.5990851)(520.38693907,374.32872145)(520.87231118,374.79107298)
\curveto(521.35768328,375.25342451)(522.16870633,375.2742718)(522.6826305,374.83760712)
\closepath
}
}
{
\newrgbcolor{curcolor}{0 0 0}
\pscustom[linestyle=none,fillstyle=solid,fillcolor=curcolor]
{
\newpath
\moveto(175.56272854,169.83544104)
\lineto(174.31213131,169.83544104)
\lineto(174.29962534,173.22198888)
\curveto(173.42420727,173.2369902)(172.54878921,173.32699811)(171.67337115,173.49201263)
\curveto(170.79795309,173.6645278)(169.91836637,173.91955023)(169.03461099,174.25707991)
\lineto(169.03461099,176.28225802)
\curveto(169.88501711,175.8022158)(170.74376054,175.43843381)(171.61084129,175.19091204)
\curveto(172.48625935,174.95089093)(173.38668936,174.82713005)(174.31213131,174.81962939)
\lineto(174.31213131,179.9500806)
\curveto(172.46958472,180.22010435)(171.12727703,180.67764459)(170.28520822,181.32270132)
\curveto(169.45147674,181.96775805)(169.03461099,182.85283589)(169.03461099,183.97793484)
\curveto(169.03461099,185.20054237)(169.48899465,186.16437713)(170.39776198,186.86943914)
\curveto(171.3065293,187.57450115)(172.61131907,187.97953677)(174.31213131,188.08454601)
\lineto(174.31213131,190.72852854)
\lineto(175.56272854,190.72852854)
\lineto(175.56272854,188.11829898)
\curveto(176.33809882,188.08829634)(177.08845716,188.01328974)(177.81380356,187.89327919)
\curveto(178.53914995,187.78076929)(179.24782171,187.62325544)(179.93981885,187.42073763)
\lineto(179.93981885,185.45181447)
\curveto(179.24782171,185.76684217)(178.53498129,186.01061361)(177.80129758,186.18312878)
\curveto(177.07595119,186.35564395)(176.32976151,186.45690286)(175.56272854,186.4869055)
\lineto(175.56272854,181.68273298)
\curveto(177.45529902,181.42020989)(178.8476306,180.95141866)(179.73972329,180.27635929)
\curveto(180.63181598,179.60129992)(181.07786233,178.67871879)(181.07786233,177.50861588)
\curveto(181.07786233,176.24100439)(180.60263538,175.23966633)(179.65218149,174.50460168)
\curveto(178.71006491,173.77703769)(177.34691392,173.35700075)(175.56272854,173.24449086)
\closepath
\moveto(174.31213131,181.88525079)
\lineto(174.31213131,186.49815649)
\curveto(173.34500278,186.40064791)(172.60715042,186.15312614)(172.09857421,185.75559118)
\curveto(171.589998,185.35805622)(171.3357099,184.82925971)(171.3357099,184.16920166)
\curveto(171.3357099,183.52414493)(171.56915472,183.02160073)(172.03604435,182.66156907)
\curveto(172.5112713,182.3015374)(173.26996695,182.04276465)(174.31213131,181.88525079)
\closepath
\moveto(175.56272854,179.72506081)
\lineto(175.56272854,174.85338236)
\curveto(176.62156753,174.98089357)(177.4177811,175.25091732)(177.95136925,175.6634536)
\curveto(178.49329472,176.07598988)(178.76425745,176.61978771)(178.76425745,177.29484708)
\curveto(178.76425745,177.95490513)(178.50580069,178.4799513)(177.98888717,178.86998561)
\curveto(177.48031096,179.26001991)(176.67159142,179.54504498)(175.56272854,179.72506081)
\closepath
}
}
{
\newrgbcolor{curcolor}{0 0 0}
\pscustom[linestyle=none,fillstyle=solid,fillcolor=curcolor]
{
\newpath
\moveto(186.39290132,175.13465709)
\lineto(190.51987218,175.13465709)
\lineto(190.51987218,187.94953413)
\lineto(186.03022812,187.13946289)
\lineto(186.03022812,189.20964496)
\lineto(190.49486024,190.0197162)
\lineto(193.02106664,190.0197162)
\lineto(193.02106664,175.13465709)
\lineto(197.14803751,175.13465709)
\lineto(197.14803751,173.22198888)
\lineto(186.39290132,173.22198888)
\closepath
}
}
{
\newrgbcolor{curcolor}{0 0 0}
\pscustom[linestyle=none,fillstyle=solid,fillcolor=curcolor]
{
\newpath
\moveto(208.17830584,169.83544104)
\lineto(206.92770861,169.83544104)
\lineto(206.91520264,173.22198888)
\curveto(206.03978458,173.2369902)(205.16436652,173.32699811)(204.28894846,173.49201263)
\curveto(203.41353039,173.6645278)(202.53394367,173.91955023)(201.6501883,174.25707991)
\lineto(201.6501883,176.28225802)
\curveto(202.50059441,175.8022158)(203.35933785,175.43843381)(204.22641859,175.19091204)
\curveto(205.10183666,174.95089093)(206.00226666,174.82713005)(206.92770861,174.81962939)
\lineto(206.92770861,179.9500806)
\curveto(205.08516203,180.22010435)(203.74285433,180.67764459)(202.90078553,181.32270132)
\curveto(202.06705404,181.96775805)(201.6501883,182.85283589)(201.6501883,183.97793484)
\curveto(201.6501883,185.20054237)(202.10457196,186.16437713)(203.01333928,186.86943914)
\curveto(203.9221066,187.57450115)(205.22689638,187.97953677)(206.92770861,188.08454601)
\lineto(206.92770861,190.72852854)
\lineto(208.17830584,190.72852854)
\lineto(208.17830584,188.11829898)
\curveto(208.95367613,188.08829634)(209.70403447,188.01328974)(210.42938086,187.89327919)
\curveto(211.15472725,187.78076929)(211.86339902,187.62325544)(212.55539615,187.42073763)
\lineto(212.55539615,185.45181447)
\curveto(211.86339902,185.76684217)(211.1505586,186.01061361)(210.41687489,186.18312878)
\curveto(209.69152849,186.35564395)(208.94533881,186.45690286)(208.17830584,186.4869055)
\lineto(208.17830584,181.68273298)
\curveto(210.07087632,181.42020989)(211.46320791,180.95141866)(212.3553006,180.27635929)
\curveto(213.24739329,179.60129992)(213.69343963,178.67871879)(213.69343963,177.50861588)
\curveto(213.69343963,176.24100439)(213.21821269,175.23966633)(212.26775879,174.50460168)
\curveto(211.32564221,173.77703769)(209.96249123,173.35700075)(208.17830584,173.24449086)
\closepath
\moveto(206.92770861,181.88525079)
\lineto(206.92770861,186.49815649)
\curveto(205.96058009,186.40064791)(205.22272772,186.15312614)(204.71415151,185.75559118)
\curveto(204.20557531,185.35805622)(203.9512872,184.82925971)(203.9512872,184.16920166)
\curveto(203.9512872,183.52414493)(204.18473202,183.02160073)(204.65162165,182.66156907)
\curveto(205.1268486,182.3015374)(205.88554425,182.04276465)(206.92770861,181.88525079)
\closepath
\moveto(208.17830584,179.72506081)
\lineto(208.17830584,174.85338236)
\curveto(209.23714483,174.98089357)(210.0333584,175.25091732)(210.56694656,175.6634536)
\curveto(211.10887202,176.07598988)(211.37983476,176.61978771)(211.37983476,177.29484708)
\curveto(211.37983476,177.95490513)(211.121378,178.4799513)(210.60446447,178.86998561)
\curveto(210.09588827,179.26001991)(209.28716872,179.54504498)(208.17830584,179.72506081)
\closepath
}
}
{
\newrgbcolor{curcolor}{0 0 0}
\pscustom[linestyle=none,fillstyle=solid,fillcolor=curcolor]
{
\newpath
\moveto(537.79285197,356.64043999)
\lineto(536.54225474,356.64043999)
\lineto(536.52974877,360.02698783)
\curveto(535.65433071,360.04198915)(534.77891265,360.13199706)(533.90349458,360.29701158)
\curveto(533.02807652,360.46952675)(532.1484898,360.72454918)(531.26473443,361.06207886)
\lineto(531.26473443,363.08725697)
\curveto(532.11514054,362.60721475)(532.97388398,362.24343276)(533.84096472,361.99591099)
\curveto(534.71638278,361.75588988)(535.61681279,361.632129)(536.54225474,361.62462834)
\lineto(536.54225474,366.75507955)
\curveto(534.69970816,367.0251033)(533.35740046,367.48264354)(532.51533166,368.12770027)
\curveto(531.68160017,368.772757)(531.26473443,369.65783484)(531.26473443,370.78293379)
\curveto(531.26473443,372.00554132)(531.71911809,372.96937609)(532.62788541,373.67443809)
\curveto(533.53665273,374.3795001)(534.84144251,374.78453572)(536.54225474,374.88954496)
\lineto(536.54225474,377.53352749)
\lineto(537.79285197,377.53352749)
\lineto(537.79285197,374.92329793)
\curveto(538.56822226,374.89329529)(539.3185806,374.81828869)(540.04392699,374.69827814)
\curveto(540.76927338,374.58576824)(541.47794515,374.42825439)(542.16994228,374.22573658)
\lineto(542.16994228,372.25681342)
\curveto(541.47794515,372.57184112)(540.76510473,372.81561256)(540.03142102,372.98812773)
\curveto(539.30607462,373.16064291)(538.55988494,373.26190181)(537.79285197,373.29190445)
\lineto(537.79285197,368.48773193)
\curveto(539.68542245,368.22520885)(541.07775403,367.75641762)(541.96984673,367.08135825)
\curveto(542.86193942,366.40629888)(543.30798576,365.48371774)(543.30798576,364.31361483)
\curveto(543.30798576,363.04600335)(542.83275882,362.04466528)(541.88230492,361.30960063)
\curveto(540.94018834,360.58203664)(539.57703736,360.1619997)(537.79285197,360.04948981)
\closepath
\moveto(536.54225474,368.69024974)
\lineto(536.54225474,373.30315544)
\curveto(535.57512622,373.20564686)(534.83727385,372.9581251)(534.32869764,372.56059013)
\curveto(533.82012144,372.16305517)(533.56583333,371.63425866)(533.56583333,370.97420061)
\curveto(533.56583333,370.32914388)(533.79927815,369.82659968)(534.26616778,369.46656802)
\curveto(534.74139473,369.10653636)(535.50009038,368.8477636)(536.54225474,368.69024974)
\closepath
\moveto(537.79285197,366.53005976)
\lineto(537.79285197,361.65838131)
\curveto(538.85169096,361.78589252)(539.64790453,362.05591627)(540.18149269,362.46845255)
\curveto(540.72341815,362.88098883)(540.99438089,363.42478666)(540.99438089,364.09984603)
\curveto(540.99438089,364.75990408)(540.73592412,365.28495026)(540.2190106,365.67498456)
\curveto(539.71043439,366.06501886)(538.90171485,366.35004393)(537.79285197,366.53005976)
\closepath
}
}
{
\newrgbcolor{curcolor}{0 0 0}
\pscustom[linestyle=none,fillstyle=solid,fillcolor=curcolor]
{
\newpath
\moveto(550.3613549,361.93965604)
\lineto(559.17806538,361.93965604)
\lineto(559.17806538,360.02698783)
\lineto(547.32240363,360.02698783)
\lineto(547.32240363,361.93965604)
\curveto(548.28119484,362.83223454)(549.58598462,364.02858976)(551.23677297,365.5287217)
\curveto(552.89589863,367.03635429)(553.93806299,368.00768972)(554.36326604,368.44272798)
\curveto(555.17198559,369.26029988)(555.73475434,369.95036057)(556.05157231,370.51291004)
\curveto(556.37672759,371.08296018)(556.53930523,371.64175932)(556.53930523,372.18930748)
\curveto(556.53930523,373.08188598)(556.189138,373.80944997)(555.48880355,374.37199944)
\curveto(554.79680642,374.93454892)(553.89220775,375.21582366)(552.77500756,375.21582366)
\curveto(551.98296265,375.21582366)(551.1450625,375.09206277)(550.26130713,374.844541)
\curveto(549.38588906,374.59701923)(548.44794114,374.22198625)(547.44746336,373.71944205)
\lineto(547.44746336,376.01464391)
\curveto(548.46461577,376.38217623)(549.41506967,376.65970064)(550.29882504,376.84721713)
\curveto(551.18258042,377.03473362)(551.99129996,377.12849187)(552.72498367,377.12849187)
\curveto(554.65924072,377.12849187)(556.20164397,376.69345361)(557.35219343,375.82337709)
\curveto(558.50274288,374.95330057)(559.07801761,373.79069832)(559.07801761,372.33557034)
\curveto(559.07801761,371.64550965)(558.9321146,370.98920193)(558.64030858,370.36664718)
\curveto(558.35683987,369.75159309)(557.83575769,369.0240291)(557.07706204,368.18395522)
\curveto(556.86862916,367.96643609)(556.20581263,367.33638067)(555.08861244,366.29378898)
\curveto(553.97141225,365.25869795)(552.39565973,363.8073203)(550.3613549,361.93965604)
\closepath
}
}
{
\newrgbcolor{curcolor}{0 0 0}
\pscustom[linestyle=none,fillstyle=solid,fillcolor=curcolor]
{
\newpath
\moveto(570.40842928,356.64043999)
\lineto(569.15783205,356.64043999)
\lineto(569.14532607,360.02698783)
\curveto(568.26990801,360.04198915)(567.39448995,360.13199706)(566.51907189,360.29701158)
\curveto(565.64365383,360.46952675)(564.76406711,360.72454918)(563.88031173,361.06207886)
\lineto(563.88031173,363.08725697)
\curveto(564.73071785,362.60721475)(565.58946128,362.24343276)(566.45654203,361.99591099)
\curveto(567.33196009,361.75588988)(568.2323901,361.632129)(569.15783205,361.62462834)
\lineto(569.15783205,366.75507955)
\curveto(567.31528546,367.0251033)(565.97297776,367.48264354)(565.13090896,368.12770027)
\curveto(564.29717747,368.772757)(563.88031173,369.65783484)(563.88031173,370.78293379)
\curveto(563.88031173,372.00554132)(564.33469539,372.96937609)(565.24346271,373.67443809)
\curveto(566.15223003,374.3795001)(567.45701981,374.78453572)(569.15783205,374.88954496)
\lineto(569.15783205,377.53352749)
\lineto(570.40842928,377.53352749)
\lineto(570.40842928,374.92329793)
\curveto(571.18379956,374.89329529)(571.9341579,374.81828869)(572.65950429,374.69827814)
\curveto(573.38485069,374.58576824)(574.09352245,374.42825439)(574.78551959,374.22573658)
\lineto(574.78551959,372.25681342)
\curveto(574.09352245,372.57184112)(573.38068203,372.81561256)(572.64699832,372.98812773)
\curveto(571.92165193,373.16064291)(571.17546225,373.26190181)(570.40842928,373.29190445)
\lineto(570.40842928,368.48773193)
\curveto(572.30099975,368.22520885)(573.69333134,367.75641762)(574.58542403,367.08135825)
\curveto(575.47751672,366.40629888)(575.92356307,365.48371774)(575.92356307,364.31361483)
\curveto(575.92356307,363.04600335)(575.44833612,362.04466528)(574.49788222,361.30960063)
\curveto(573.55576564,360.58203664)(572.19261466,360.1619997)(570.40842928,360.04948981)
\closepath
\moveto(569.15783205,368.69024974)
\lineto(569.15783205,373.30315544)
\curveto(568.19070352,373.20564686)(567.45285115,372.9581251)(566.94427495,372.56059013)
\curveto(566.43569874,372.16305517)(566.18141064,371.63425866)(566.18141064,370.97420061)
\curveto(566.18141064,370.32914388)(566.41485545,369.82659968)(566.88174509,369.46656802)
\curveto(567.35697203,369.10653636)(568.11566769,368.8477636)(569.15783205,368.69024974)
\closepath
\moveto(570.40842928,366.53005976)
\lineto(570.40842928,361.65838131)
\curveto(571.46726827,361.78589252)(572.26348184,362.05591627)(572.79706999,362.46845255)
\curveto(573.33899546,362.88098883)(573.60995819,363.42478666)(573.60995819,364.09984603)
\curveto(573.60995819,364.75990408)(573.35150143,365.28495026)(572.83458791,365.67498456)
\curveto(572.3260117,366.06501886)(571.51729216,366.35004393)(570.40842928,366.53005976)
\closepath
}
}
{
\newrgbcolor{curcolor}{0 0 0}
\pscustom[linestyle=none,fillstyle=solid,fillcolor=curcolor]
{
\newpath
\moveto(424.36721769,261.18058613)
\lineto(423.11662046,261.18058613)
\lineto(423.10411449,264.56713397)
\curveto(422.22869643,264.58213529)(421.35327837,264.67214321)(420.4778603,264.83715772)
\curveto(419.60244224,265.00967289)(418.72285552,265.26469532)(417.83910015,265.60222501)
\lineto(417.83910015,267.62740312)
\curveto(418.68950626,267.1473609)(419.5482497,266.7835789)(420.41533044,266.53605714)
\curveto(421.2907485,266.29603603)(422.19117851,266.17227514)(423.11662046,266.16477448)
\lineto(423.11662046,271.29522569)
\curveto(421.27407387,271.56524944)(419.93176618,272.02278968)(419.08969738,272.66784641)
\curveto(418.25596589,273.31290314)(417.83910015,274.19798099)(417.83910015,275.32307994)
\curveto(417.83910015,276.54568746)(418.29348381,277.50952223)(419.20225113,278.21458424)
\curveto(420.11101845,278.91964625)(421.41580823,279.32468187)(423.11662046,279.4296911)
\lineto(423.11662046,282.07367364)
\lineto(424.36721769,282.07367364)
\lineto(424.36721769,279.46344407)
\curveto(425.14258798,279.43344143)(425.89294631,279.35843484)(426.61829271,279.23842428)
\curveto(427.3436391,279.12591439)(428.05231087,278.96840053)(428.744308,278.76588272)
\lineto(428.744308,276.79695956)
\curveto(428.05231087,277.11198727)(427.33947045,277.35575871)(426.60578674,277.52827388)
\curveto(425.88044034,277.70078905)(425.13425066,277.80204796)(424.36721769,277.83205059)
\lineto(424.36721769,273.02787808)
\curveto(426.25978817,272.76535499)(427.65211975,272.29656376)(428.54421245,271.62150439)
\curveto(429.43630514,270.94644502)(429.88235148,270.02386388)(429.88235148,268.85376097)
\curveto(429.88235148,267.58614949)(429.40712453,266.58481142)(428.45667064,265.84974678)
\curveto(427.51455406,265.12218279)(426.15140308,264.70214585)(424.36721769,264.58963595)
\closepath
\moveto(423.11662046,273.23039589)
\lineto(423.11662046,277.84330158)
\curveto(422.14949194,277.74579301)(421.41163957,277.49827124)(420.90306336,277.10073628)
\curveto(420.39448715,276.70320131)(420.14019905,276.17440481)(420.14019905,275.51434676)
\curveto(420.14019905,274.86929003)(420.37364387,274.36674583)(420.8405335,274.00671416)
\curveto(421.31576045,273.6466825)(422.0744561,273.38790974)(423.11662046,273.23039589)
\closepath
\moveto(424.36721769,271.0702059)
\lineto(424.36721769,266.19852745)
\curveto(425.42605668,266.32603866)(426.22227025,266.59606241)(426.7558584,267.00859869)
\curveto(427.29778387,267.42113498)(427.5687466,267.9649328)(427.5687466,268.63999217)
\curveto(427.5687466,269.30005022)(427.31028984,269.8250964)(426.79337632,270.2151307)
\curveto(426.28480011,270.605165)(425.47608057,270.89019007)(424.36721769,271.0702059)
\closepath
}
}
{
\newrgbcolor{curcolor}{0 0 0}
\pscustom[linestyle=none,fillstyle=solid,fillcolor=curcolor]
{
\newpath
\moveto(434.1468888,281.3648613)
\lineto(440.6499944,262.42944597)
\lineto(438.52397911,262.42944597)
\lineto(432.0208735,281.3648613)
\closepath
}
}
{
\newrgbcolor{curcolor}{0 0 0}
\pscustom[linestyle=none,fillstyle=solid,fillcolor=curcolor]
{
\newpath
\moveto(441.43787182,277.16824221)
\lineto(443.87653642,277.16824221)
\lineto(448.25362673,266.59231208)
\lineto(452.63071704,277.16824221)
\lineto(455.06938164,277.16824221)
\lineto(449.81687327,264.56713397)
\lineto(446.69038019,264.56713397)
\closepath
}
}
{
\newrgbcolor{curcolor}{0 0 0}
\pscustom[linestyle=none,fillstyle=solid,fillcolor=curcolor]
{
\newpath
\moveto(468.30069939,270.8564371)
\curveto(468.30069939,272.37907102)(467.95053217,273.5716759)(467.25019772,274.43425177)
\curveto(466.55820058,275.30432829)(465.60357803,275.73936655)(464.38633006,275.73936655)
\curveto(463.16908208,275.73936655)(462.21029087,275.30432829)(461.50995642,274.43425177)
\curveto(460.81795929,273.5716759)(460.47196072,272.37907102)(460.47196072,270.8564371)
\curveto(460.47196072,269.33380319)(460.81795929,268.13744797)(461.50995642,267.26737145)
\curveto(462.21029087,266.40479559)(463.16908208,265.97350766)(464.38633006,265.97350766)
\curveto(465.60357803,265.97350766)(466.55820058,266.40479559)(467.25019772,267.26737145)
\curveto(467.95053217,268.13744797)(468.30069939,269.33380319)(468.30069939,270.8564371)
\closepath
\moveto(460.47196072,275.255574)
\curveto(460.95552499,276.00563997)(461.56414897,276.56068878)(462.29783268,276.92072044)
\curveto(463.0398537,277.28825277)(463.92360908,277.47201893)(464.94909881,277.47201893)
\curveto(466.64991105,277.47201893)(468.02973666,276.8644655)(469.08857565,275.64935863)
\curveto(470.15575195,274.43425177)(470.6893401,272.83661126)(470.6893401,270.8564371)
\curveto(470.6893401,268.87626295)(470.15575195,267.27862244)(469.08857565,266.06351558)
\curveto(468.02973666,264.84840871)(466.64991105,264.24085528)(464.94909881,264.24085528)
\curveto(463.92360908,264.24085528)(463.0398537,264.42087111)(462.29783268,264.78090277)
\curveto(461.56414897,265.1484351)(460.95552499,265.70723424)(460.47196072,266.45730021)
\lineto(460.47196072,264.56713397)
\lineto(458.15835585,264.56713397)
\lineto(458.15835585,282.07367364)
\lineto(460.47196072,282.07367364)
\closepath
}
}
{
\newrgbcolor{curcolor}{0 0 0}
\pscustom[linestyle=none,fillstyle=solid,fillcolor=curcolor]
{
\newpath
\moveto(485.18375956,262.42944597)
\lineto(485.18375956,260.80930348)
\lineto(484.40838927,260.80930348)
\curveto(482.33239787,260.80930348)(480.94006628,261.08682789)(480.23139452,261.6418767)
\curveto(479.53106007,262.19692552)(479.18089285,263.30327282)(479.18089285,264.96091861)
\lineto(479.18089285,267.6499051)
\curveto(479.18089285,268.78250471)(478.95578534,269.56632364)(478.50557034,270.0013619)
\curveto(478.05535534,270.43640016)(477.23829848,270.65391929)(476.05439977,270.65391929)
\lineto(475.29153546,270.65391929)
\lineto(475.29153546,272.26281079)
\lineto(476.05439977,272.26281079)
\curveto(477.24663579,272.26281079)(478.06369265,272.47657959)(478.50557034,272.90411719)
\curveto(478.95578534,273.33915545)(479.18089285,274.11547373)(479.18089285,275.23307202)
\lineto(479.18089285,277.9333095)
\curveto(479.18089285,279.59095529)(479.53106007,280.69355226)(480.23139452,281.24110041)
\curveto(480.94006628,281.79614923)(482.33239787,282.07367364)(484.40838927,282.07367364)
\lineto(485.18375956,282.07367364)
\lineto(485.18375956,280.46478214)
\lineto(484.33335344,280.46478214)
\curveto(483.15779204,280.46478214)(482.39075907,280.29976763)(482.03225453,279.9697386)
\curveto(481.67374999,279.63970957)(481.49449772,278.94589856)(481.49449772,277.88830554)
\lineto(481.49449772,275.09806015)
\curveto(481.49449772,273.92045658)(481.30273948,273.06538138)(480.919223,272.53283454)
\curveto(480.54404383,272.0002877)(479.89790192,271.64025604)(478.98079729,271.45273955)
\curveto(479.90623924,271.25022174)(480.5565498,270.88268941)(480.93172897,270.35014258)
\curveto(481.30690814,269.81759574)(481.49449772,268.96627087)(481.49449772,267.79616796)
\lineto(481.49449772,265.00592256)
\curveto(481.49449772,263.94832955)(481.67374999,263.25451853)(482.03225453,262.92448951)
\curveto(482.39075907,262.59446048)(483.15779204,262.42944597)(484.33335344,262.42944597)
\closepath
}
}
{
\newrgbcolor{curcolor}{0 0 0}
\pscustom[linestyle=none,fillstyle=solid,fillcolor=curcolor]
{
\newpath
\moveto(498.92782578,275.23307202)
\curveto(498.66936902,275.36808389)(498.38590032,275.46559247)(498.07741967,275.52559775)
\curveto(497.77727633,275.59310368)(497.44378374,275.62685665)(497.07694188,275.62685665)
\curveto(495.77632076,275.62685665)(494.77584298,275.24432301)(494.07550853,274.47925572)
\curveto(493.38351139,273.7216891)(493.03751282,272.63034312)(493.03751282,271.20521778)
\lineto(493.03751282,264.56713397)
\lineto(490.72390795,264.56713397)
\lineto(490.72390795,277.16824221)
\lineto(493.03751282,277.16824221)
\lineto(493.03751282,275.21057004)
\curveto(493.52107709,275.97563733)(494.15054436,276.54193713)(494.92591464,276.90946946)
\curveto(495.70128493,277.28450244)(496.64340151,277.47201893)(497.75226439,277.47201893)
\curveto(497.91067337,277.47201893)(498.08575698,277.46076794)(498.27751522,277.43826596)
\curveto(498.46927347,277.42326464)(498.68187499,277.39701233)(498.91531981,277.35950904)
\closepath
}
}
{
\newrgbcolor{curcolor}{0 0 0}
\pscustom[linestyle=none,fillstyle=solid,fillcolor=curcolor]
{
\newpath
\moveto(502.15436262,262.42944597)
\lineto(503.02978069,262.42944597)
\curveto(504.19700477,262.42944597)(504.95570042,262.59071015)(505.30586765,262.91323852)
\curveto(505.66437219,263.23576688)(505.84362446,263.93332823)(505.84362446,265.00592256)
\lineto(505.84362446,267.79616796)
\curveto(505.84362446,268.96627087)(506.03121404,269.81759574)(506.40639321,270.35014258)
\curveto(506.78157238,270.88268941)(507.43188294,271.25022174)(508.35732489,271.45273955)
\curveto(507.43188294,271.64025604)(506.78157238,272.0002877)(506.40639321,272.53283454)
\curveto(506.03121404,273.06538138)(505.84362446,273.92045658)(505.84362446,275.09806015)
\lineto(505.84362446,277.88830554)
\curveto(505.84362446,278.95339921)(505.66437219,279.64721023)(505.30586765,279.9697386)
\curveto(504.95570042,280.29976763)(504.19700477,280.46478214)(503.02978069,280.46478214)
\lineto(502.15436262,280.46478214)
\lineto(502.15436262,282.07367364)
\lineto(502.94223888,282.07367364)
\curveto(505.01823028,282.07367364)(506.40222455,281.79614923)(507.09422169,281.24110041)
\curveto(507.79455614,280.69355226)(508.14472336,279.59095529)(508.14472336,277.9333095)
\lineto(508.14472336,275.23307202)
\curveto(508.14472336,274.11547373)(508.36983086,273.33915545)(508.82004587,272.90411719)
\curveto(509.27026087,272.47657959)(510.08731773,272.26281079)(511.27121644,272.26281079)
\lineto(512.04658672,272.26281079)
\lineto(512.04658672,270.65391929)
\lineto(511.27121644,270.65391929)
\curveto(510.08731773,270.65391929)(509.27026087,270.43640016)(508.82004587,270.0013619)
\curveto(508.36983086,269.56632364)(508.14472336,268.78250471)(508.14472336,267.6499051)
\lineto(508.14472336,264.96091861)
\curveto(508.14472336,263.30327282)(507.79455614,262.19692552)(507.09422169,261.6418767)
\curveto(506.40222455,261.08682789)(505.01823028,260.80930348)(502.94223888,260.80930348)
\lineto(502.15436262,260.80930348)
\closepath
}
}
{
\newrgbcolor{curcolor}{0 0 0}
\pscustom[linestyle=none,fillstyle=solid,fillcolor=curcolor]
{
\newpath
\moveto(523.20191668,282.05117166)
\curveto(522.08471649,280.32601993)(521.25515366,278.61961986)(520.71322819,276.93197143)
\curveto(520.17130273,275.24432301)(519.90033999,273.53417261)(519.90033999,271.80152022)
\curveto(519.90033999,270.06886784)(520.17130273,268.35121677)(520.71322819,266.64856703)
\curveto(521.26349097,264.95341795)(522.0930538,263.24701787)(523.20191668,261.52936681)
\lineto(521.20096111,261.52936681)
\curveto(519.95036388,263.29202183)(519.01241596,265.02467421)(518.38711734,266.72732396)
\curveto(517.77015604,268.4299737)(517.46167539,270.12137246)(517.46167539,271.80152022)
\curveto(517.46167539,273.47416733)(517.77015604,275.15806542)(518.38711734,276.85321451)
\curveto(519.00407864,278.54836359)(519.94202657,280.28101598)(521.20096111,282.05117166)
\closepath
}
}
{
\newrgbcolor{curcolor}{0 0 0}
\pscustom[linestyle=none,fillstyle=solid,fillcolor=curcolor]
{
\newpath
\moveto(529.95514607,280.74605688)
\lineto(529.95514607,277.16824221)
\lineto(534.69490957,277.16824221)
\lineto(534.69490957,275.55935072)
\lineto(529.95514607,275.55935072)
\lineto(529.95514607,268.7187491)
\curveto(529.95514607,267.69115872)(530.10938639,267.03110067)(530.41786704,266.73857495)
\curveto(530.73468501,266.44604922)(531.37248959,266.29978636)(532.33128081,266.29978636)
\lineto(534.69490957,266.29978636)
\lineto(534.69490957,264.56713397)
\lineto(532.33128081,264.56713397)
\curveto(530.55543274,264.56713397)(529.32984745,264.86341003)(528.65452495,265.45596214)
\curveto(527.97920244,266.05601492)(527.64154119,267.14361057)(527.64154119,268.7187491)
\lineto(527.64154119,275.55935072)
\lineto(525.95323493,275.55935072)
\lineto(525.95323493,277.16824221)
\lineto(527.64154119,277.16824221)
\lineto(527.64154119,280.74605688)
\closepath
}
}
{
\newrgbcolor{curcolor}{0 0 0}
\pscustom[linestyle=none,fillstyle=solid,fillcolor=curcolor]
{
\newpath
\moveto(537.3711857,282.05117166)
\lineto(539.37214127,282.05117166)
\curveto(540.6227385,280.28101598)(541.55651776,278.54836359)(542.17347906,276.85321451)
\curveto(542.79877768,275.15806542)(543.11142699,273.47416733)(543.11142699,271.80152022)
\curveto(543.11142699,270.12137246)(542.79877768,268.4299737)(542.17347906,266.72732396)
\curveto(541.55651776,265.02467421)(540.6227385,263.29202183)(539.37214127,261.52936681)
\lineto(537.3711857,261.52936681)
\curveto(538.48004857,263.24701787)(539.30544275,264.95341795)(539.84736821,266.64856703)
\curveto(540.39763099,268.35121677)(540.67276239,270.06886784)(540.67276239,271.80152022)
\curveto(540.67276239,273.53417261)(540.39763099,275.24432301)(539.84736821,276.93197143)
\curveto(539.30544275,278.61961986)(538.48004857,280.32601993)(537.3711857,282.05117166)
\closepath
}
}
{
\newrgbcolor{curcolor}{0 0 0}
\pscustom[linestyle=none,fillstyle=solid,fillcolor=curcolor]
{
\newpath
\moveto(553.97911366,261.18058613)
\lineto(552.72851643,261.18058613)
\lineto(552.71601046,264.56713397)
\curveto(551.84059239,264.58213529)(550.96517433,264.67214321)(550.08975627,264.83715772)
\curveto(549.21433821,265.00967289)(548.33475149,265.26469532)(547.45099611,265.60222501)
\lineto(547.45099611,267.62740312)
\curveto(548.30140223,267.1473609)(549.16014566,266.7835789)(550.02722641,266.53605714)
\curveto(550.90264447,266.29603603)(551.80307448,266.17227514)(552.72851643,266.16477448)
\lineto(552.72851643,271.29522569)
\curveto(550.88596984,271.56524944)(549.54366215,272.02278968)(548.70159334,272.66784641)
\curveto(547.86786186,273.31290314)(547.45099611,274.19798099)(547.45099611,275.32307994)
\curveto(547.45099611,276.54568746)(547.90537977,277.50952223)(548.81414709,278.21458424)
\curveto(549.72291442,278.91964625)(551.02770419,279.32468187)(552.72851643,279.4296911)
\lineto(552.72851643,282.07367364)
\lineto(553.97911366,282.07367364)
\lineto(553.97911366,279.46344407)
\curveto(554.75448394,279.43344143)(555.50484228,279.35843484)(556.23018868,279.23842428)
\curveto(556.95553507,279.12591439)(557.66420683,278.96840053)(558.35620397,278.76588272)
\lineto(558.35620397,276.79695956)
\curveto(557.66420683,277.11198727)(556.95136641,277.35575871)(556.2176827,277.52827388)
\curveto(555.49233631,277.70078905)(554.74614663,277.80204796)(553.97911366,277.83205059)
\lineto(553.97911366,273.02787808)
\curveto(555.87168414,272.76535499)(557.26401572,272.29656376)(558.15610841,271.62150439)
\curveto(559.0482011,270.94644502)(559.49424745,270.02386388)(559.49424745,268.85376097)
\curveto(559.49424745,267.58614949)(559.0190205,266.58481142)(558.06856661,265.84974678)
\curveto(557.12645003,265.12218279)(555.76329904,264.70214585)(553.97911366,264.58963595)
\closepath
\moveto(552.72851643,273.23039589)
\lineto(552.72851643,277.84330158)
\curveto(551.7613879,277.74579301)(551.02353554,277.49827124)(550.51495933,277.10073628)
\curveto(550.00638312,276.70320131)(549.75209502,276.17440481)(549.75209502,275.51434676)
\curveto(549.75209502,274.86929003)(549.98553983,274.36674583)(550.45242947,274.00671416)
\curveto(550.92765642,273.6466825)(551.68635207,273.38790974)(552.72851643,273.23039589)
\closepath
\moveto(553.97911366,271.0702059)
\lineto(553.97911366,266.19852745)
\curveto(555.03795265,266.32603866)(555.83416622,266.59606241)(556.36775437,267.00859869)
\curveto(556.90967984,267.42113498)(557.18064257,267.9649328)(557.18064257,268.63999217)
\curveto(557.18064257,269.30005022)(556.92218581,269.8250964)(556.40527229,270.2151307)
\curveto(555.89669608,270.605165)(555.08797654,270.89019007)(553.97911366,271.0702059)
\closepath
}
}
\end{pspicture}

\caption{Траектория материальной точки.}
\end{figure}
\begin{gather}
\dv{T}{t} = \vb{F} \cdot \vb{v}\\
\int \limits_{t_1}^{t_2} \dv{T}{t} \, dt = T(2) - T(1) = \int \limits_{t_1}^{t_2} \vb{F} \cdot \underbrace{\vb{v} \, dt}_{d \vb{r}} = \int \limits_{\vb{r}_1}^{\vb{r}_2} \vb{F} \, d\vb{r} = \int \limits_1^2 \, dA
\end{gather}


\begin{dfn}
Сила является потенциальной, если существует потенциал $U(\vb{r}, t)$.
\begin{gather}
\vb{F} = - \dv{U}{\vb{r}} = \left\{ -\pdv{U}{x}, -\pdv{U}{y}, -\pdv{U}{z} \right\} = -\nabla U\\
rot \, \vb{F} \equiv 0 \Leftrightarrow [\vphantom{\vb{F}} \nabla, \vb{F}] = 0,
\end{gather}
но вся эта наука справедлива в односвязной стягиваемой области. Если сила потенциальная, то 
\begin{equation*}
\oint \vb{F} (\tau, r)\, d\vb{r} = 0
\end{equation*}
\end{dfn}
\begin{dfn}
\begin{equation}
\pdv{U}{t} = 0 \Rightarrow \text{сила консервативная (стационарная потенциальная).}
\end{equation}
\end{dfn}
\begin{dfn}
\begin{gather}
dA = 0,\\
\intertext{значит, сила гироскопическая (сила Кориолиса, магнитная составляющая силы Лоренцa).}
\left(\vb{F}, \vb{v}\right) = 0
\end{gather}
\end{dfn}
\begin{dfn}
Диссипативная сила:
\begin{equation}
dA < 0
\end{equation}
\begin{ex}
\begin{equation}
\vb{F} = -k (t, \vb{r}, \vb{v}) \cdot \vb{v}, \quad k>0
\end{equation}
\end{ex}
\end{dfn}
Запишем выражение для силы $\vb{F}$, используя введённые определения и считая, что они покрывают все силы:
\begin{equation}
\vb{F} = - \nabla  U (\vb{r}, t) + \vb{F}_g + \vb{F}_d
\end{equation}
Введём понятие полной энергии:
\begin{equation}
T + U = E.
\end{equation}
Получим дифференциалы потенциальной и кинетической энергии:
\begin{gather}
\dv{T}{t} = - \pdv{U}{\vb{r}} \vb{v} + \vb{F}_d \cdot \vb{v}; \label{dT}\\
\dv{U}{t} = \pdv{U}{t} + \pdv{U}{\vb{r}} \dot{\vb{r}}; \label{dU}
\end{gather}
Сложим \eqref{dT} и \eqref{dU}:
\begin{equation}
\dv{E}{t} = \pdv{U}{t} + \vb{F}_d \cdot \vb{v},
\end{equation}
то есть для выполнения закона сохранения энергии требуется, чтобы все потенциальные силы были консервативными, и отсутствовали диссипативные силы, а полная энергия материальной точки может изменяться за счёт работы и диссипативных сил и работы потенциальных неконсервативных сил.

\subparagraph{Закон сохранения энергии для системы материальных точек.}
\begin{gather}
m_i \ddot{\vb{v}}_i  = \vb{F}_i \qquad |\cdot \vb{v}_i, \sum\limits_i \quad i = 1, N\\
\dv{T}{t} = \sum \limits_{i=1}^N \vb{F}_i \vb{v}_i ; \quad T = \sum \limits_{i=1}^N \frac{m_i v_i^2}{2}\\
\vb{F}_i = \vb{F}_i^{(e)} + \sum_{i \neq j} \vb{F}_{ij}\\
F_i^{(e)} = - \pdv{U^{(e)} (\vb{r}_i)}{\vb{r}_i} + \vb{F}^{(e)}_{g\;i} + \vb{F}^{(e)}_{d\;i}, 
\end{gather}
то есть для каждой внешней силы проделали то же, что делали в случае одной материальной точки. И посчитаем:
\begin{gather}
\sum  \vb{F}_i^{(e)} \cdot \vb{v}_i = - \sum \pdv{U^{(e)}}{\vb{r}_i} \cdot \dot{\vb{r}}_i + \sum  \vb{F}_{d\;i}^{(e)} \cdot \vb{v}_i, \label{sum_F^e}\\
\intertext{введём $U^{(e)}$:}
U^{(e)} = \sum_{i=1}^N U^{(e)}_i (\vb{r}_i), \text{тогда}\\
\dv{U^{(e)}}{t} = \sum \pdv{U^{(e)}}{\vb{r}_i} \cdot \dot{\vb{r}}_i + \pdv{U^{(e)}}{t}. \label{dU^e}\\
\intertext{Выразим первое слагаемое в правой части \eqref{sum_F^e} из  \eqref{dU^e} и соберём полные производные энергии по времени вместе:}
\dv{t}\left(T + U^{(e)}\right) = \pdv{U^{(e)}}{t} +  \sum  \vb{F}_{d\;i}^{(e)} \cdot \vb{v}_i.
\end{gather}

\begin{gather}
U_{ij} = U_{ij} (\abs{\vb{r}_j - \vb{r}_i}) = U_{ij} \rho_{ij} \Rightarrow U_{ij} = U_{ji};\\
\vb{F}_i = - \pdv{U_{ij}}{\vb*{\rho}_{ij}} \pdv{\vb*{\rho}_{ij}}{\vb{r}_i} = + \pdv{U_{ij}(\vb*{\rho}_{ij})}{\vb*{\rho}_{ij}}, \vb{F}_j = - \pdv{U_{ij}}{\vb*{\rho}_{ij}} \pdv{\vb*{\rho}_{ij}}{\vb{r}_j} = - \pdv{U_{ij}(\vb*{\rho}_{ij})}{\vb*{\rho}_{ij}};\\
\vb{F}_{ij} = - \pdv{\vb{r}_i} U_{ij} (\vb{r}_i, \vb{r}_j);\\
\sum_{\substack{i, j\\ i \neq j}} \vb{F}_{ij} \cdot \vb{v}_i =\sum (\underbrace{\vb{v}_i - \vb{v}_j}_{-\dot{\vb*{\rho}}_{ij}} ) \frac{\partial U_{i j}}{\partial \rho_{i j}} = -\dv{t} U^{(i)};\\
U^{(i)} = \frac{1}{2} \sum_{\substack{i, j = 1\\ i \neq j}}^N U^{(i)} (\vb*{\rho}_{ij}) \Rightarrow\\
\boxed{\dv{t}(T + U^{(e)} + U^{(i)}) = \pdv{U^{(e)}}{t} + \sum \vb{F}_{d\; i}^{(e)}}
\end{gather}
\newpage
З.\,С.\,Э. $E = const$:
\begin{enumerate}
\item Внешние силы консервативные и/или гироскопические.
\item Нет диссипативных сил.
\item Внутренние силы специальной структуры: отвечают глобальным законам симметрии пространства. $U^{(i)}(\abs{\vb{r}_i - \vb{r}_j})$.
\end{enumerate}
\begin{ex}
\begin{equation*}
E_{\text{Л}} = - e \nabla \varphi + \frac{e}{c} [\vb{v}, \vb{B}]
\end{equation*}
\end{ex}
\subparagraph{Интеграл движения.}
\begin{equation}
I(t, {\vb{r}_i}, {\vb{v}_i}) = const
\end{equation}
 (???)
 
 \subparagraph{Теорема Нётер}
 (???)
 
 \begin{ex}
 \begin{gather*}
 \begin{cases}
 \vb{F} = \frac{\alpha}{x^2 + y^2} \vb*{\tau}, \\
 \vb*{\tau} = \{-y, x, 0\},\\
 x = R \cos{\omega t},\\
 y = R \sin{\omega t},\\
 z =0 
 \end{cases}
 \Rightarrow\\
 A = \frac{\alpha}{R^2} \int\limits_0^T(\vb*{\tau}, \vb{v})\, dt =   \frac{\alpha}{R^2} \int\limits_0^T (R^2 \omega \sin^2{\omega t} + R^2 \omega \cos^2{\omega t}) \, dt = \frac{\alpha}{R^2} R^2 \omega \int\limits_0^T \, dt = \alpha \omega T = 2 \pi \alpha
 \end{gather*}
 \end{ex}
\newpage
%\bibliography{library.bib}

\end{document}
